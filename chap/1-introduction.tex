\chapter{Introduction}
\label{cha:Introduction}

Model selection problems arise in various scientific research areas. It is in
essence making decisions about which model should be used to describe the data
observed. Therefore, we insist upon a decision-oriented perspective on the
model selection problem. Two kinds of approaches to model selection are
discussed in this report. The first is the information-theoretic approach, and
the second is the Bayesian approach. The decision-oriented perspective is
applicable to both cases. It shall be noted that the two approaches are
closely related. For example a modest amount of literature on the choice of
priors used in Bayesian statistics results from information theory. And some
information approaches have a Bayesian interpretation (e.g. see
section~\ref{sub:A Bayesian analysis of aic}).

The literature review in this report serves two purposes. First, methods are
compared. We aim to answer questions like ``why the \aic based methods would
not work for this kind of complex models'' or ``why we need Bayesian modeling
for these models''. The second purpose is to review the computation
difficulties and limitations we met in the work of \textcite{Zhou:2011uo} and
therefore motivated our future research on new computation techniques for
Bayesian modeling.

Chapter~\ref{cha:Information-theoretic model selection} reviews
information-theoretic model selection methods. They are still widely used in
many areas due to the simplicity in computation. Chapter~\ref{cha:Bayesian
  model selection} reviews the basic methodology of Bayesian model comparison
and selection. Chapter~\ref{cha:Bayesian computation with Monte Carlo methods}
reviews Monte Carlo methods for Bayesian computation. It will be shown that
the current methods often suffers from either difficulties of implementation
or unsatisfied inference results. We are therefore motivated to seek new
methods that satisfy the following criteria. First, it should be applicable to
a wide range of models, with either high dimensions or complex model
specifications. Second, it should give comparable results compared to some
currently advanced methods. Third, perhaps ambitiously, we will seek general
strategies for implementing the methods for models of interest. The more
general such strategies are, the easier to apply the proposed methods to a
wider range of problems. The proposed method introduced in
chapter~\ref{cha:New method for Bayesian computation} uses sequential Monte
Carlo sampler and path sampling identity to estimate the logarithm of ratio of
normalizing constants.

\section{A word on ``true models''}
\label{sec:A word on true models}

The problem of the existence of a ``true model'' as the data generating
mechanism is often questionable. Even it exists, the applicability of model
selection techniques is often challenged by the fact that the ``true model''
is not one of the candidate models at all.

There are two folds of this problem. First, if the ``true model'' exists but
is not one of the candidate models, is model selection still meaningful? We
believe that, strictly speaking, for real problem, there is no
\emph{parametric} model that can describe the data completely. In this sense,
the ``true model'' is never in the space of model parameterization, except for
the trivial case: for finite data $\bfx = (x_1,\dots,x_n)^T$ and a model $(X_1
= x_1,\dots,X_n = x_n)^T$. However, more than often, some parametric models
serves the purpose as an approximation to the underlying mechanism of the data
generating process well. And model selection techniques provide opportunities
to identify the best approximation among them.

Second, it is often encountered in real problems that, all the candidate
models are ``far'' from the ``truth''. In this case, the effects of
misspecification need to be assessed. Some techniques were argued to be more
robust to these scenarios.  However, there is no simple way to assess this
problem. Instead, these effects often need to be considered on a per problem
basis.

In this report, the focus is on the implementation of model selection
techniques. However, whenever they are applied to real problems, the
applicability of them should be considered carefully.
