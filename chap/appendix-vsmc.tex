\chapter[vSMC: A C++ Library for Parallel SMC]
{\protect\vsmc: A C++ Library for Parallel SMC}
\label{cha:Appendix vSMC}

\section{Classes of parallel computers}
\label{sec:Classes of parallel computers}

There are a few types of parallel computers. Here we introduce the four types
of hardware parallelism that are most commonly seen. Parallel computers can be
nested. In a multicore \cpu, each core can perform instruction level
parallelism. On the other hand, a distributed system can be formed by multiple
multicore \cpu{}s.

\subsection{Instruction level}
\label{sub:Instruction level}

Modern \cpu{}s all implement the so called \simd instructions, short for
\emph{single instruction, multiple data}. The \cpu can execute a single
instruction on different data in a single cycle. However, unlike the higher
level parallelism discussed later, \simd often has strict requirement on the
arrangement of the data. In addition, the implementation often requires using
low level assembly language or intrinsics functions.

Though \vsmc does not directly implement this level of parallelism, it can be
used by the user nonetheless. In addition, many operations within \vsmc can be
performed using libraries that are implemented with \simd parallelization,
such as \mkl. Also note that, most modern \cpp compilers perform \simd
optimizations on simple loop and some of them, such as \clang \cite{clang}
performs \simd optimizations for non-loop structures. This kind of
optimization is also called \emph{vectorization}.

\subsection{Multicore processors and symmetric multiprocessing}
\label{sub:Multicore processors and symmetric multiprocessing}

In the late 1990s, computer \cpu{}s are advanced by increasing the clock
speed. However, this strategy soon hit some bottlenecks, mainly the control of
heat and power. The industry started to develop multicore processors. Each
\cpu has several cores, each running at a modest clock rate. By executing
different threads on different cores, the \cpu can process the same amount of
work with less time without increasing the clock rate.

When a computer has multiple \cpu{}s and each of them has the same speed to
access the memory, the system is often called \emph{symmetric multiprocessing}
(\smp). Most higher end workstations are \smp systems. The programming tools
are usually the same for \smp and multicore processors.

The \vsmc library support various \smp programming models. In addition, \vsmc
allows the same user implementation source code to be compiled into different
parallel samplers using different programming models.

\subsection{Distributed computing}
\label{sub:Distributed computing}

Distributed computing usually refers to the form of computing where both
memory and computing processors are spread among computing nodes. It can take
different forms, such as grids and clusters. The \emph{de facto} programming
model for distributed computing is \mpi. This is also supported by \vsmc. In
addition, the library also allows easy integration of \mpi and various \smp
programming models.

\subsection{Massive parallel computing}
\label{sub:Massive parallel computing}

In recent years, there is a new trend of using specialized massive parallel
devices, such as \gpu{}s for scientific computing. Modern \gpu{}s often have
hundreds or thousands co-processors. The main difference between \gpu and \cpu
is that, \cpu has more logic control units, and thus is more suited for
general programs. In contrast, \gpu are better at applying the same arithmetic
operations on a collection of data. It performs the best if each computing
unit are executing \emph{exactly the same} instructions. In addition, it is
often much more efficient if there are a large amount data to be processed.
Another significant feature of these devices is that they provide much higher
\emph{local} memory bandwidth and can use various technologies to reduce local
data latency than traditional \cpu.

Massive parallel computing is extremely suitable for the \smc algorithms,
which can have a large number of particles, while each of them needs to be
updated using the same \mcmc kernel.

There are two major programming models for general purpose \gpu programming
(\gpgpu), Nvidia's \cuda framework and the \opencl \cite{opencl} standard. The
\vsmc library provides direct support for the \opencl programming model.

\section{Parallel patterns}
\label{sec:Parallel patterns}

In a more micro level, parallelism can be implemented with different patterns.
The term \emph{pattern} in computer science, introduced and popularized by
\cite{software:GoF}, is a way of codifying best practices for software
engineering. We found patterns are more useful to statisticians for reasoning
the parallel structure of a given algorithm. This section is not an exhaustive
discussion of parallel patterns. Instead, we choose some of the most commonly
seen in practice, in particular those relevant to Monte Carlo algorithms.

\subsection{Map}
\label{sub:Map}

This is perhaps the simplest form of parallelism. A function, called
\emph{elemental function}, is replicated for each element of a data collection
concurrently. The elemental function must have no side-effects in order for
the map to be implementable in parallel while achieving deterministic results.
In particular, it cannot modify global data that other instances of that
function depend on.

In \smc algorithm, the updating of particle values is clearly implementable
using a map pattern. The operation of the kernel $K(x_{t-1},x_t)$ depends only
on the history of the particle that it will be used to update, but not other
particles at a given generation.

\subsection{Fork-join}
\label{sub:Fork-join}

This pattern lets control flows fork into multiple parallel flows that rejoin
later. The major difference between fork-join and map is that fork-joint does
not necessarily apply the same function on different data. Instead, usually
different functions are applied to different or the same data. There are
different programming models that implement this pattern. The \openmp
\cppinline{parallel region} fork control into multiple threads that all
execute the \emph{same} statements and use other constructs to determine which
thread does what. The \cilk \cite{cilk} \cppinline{spawn} fork a new thread to
execute the calling function on a new thread and it is later joined with the
callee.

The fork-join pattern are often used by programming models to implement other
patterns and is widely used in practice itself. One example is numerical
integrations, especially for adaptive schemes. Whenever a new segment of the
integral interval is chosen, the program can fork a new thread to compute the
results. And after all segments are computed, the program can join all threads
and sum up the final result.

\subsection{Reduction}
\label{sub:Reduction}

This pattern uses an associative operator to combine every element in a
collection into a single element. Given the associativity of the operator,
many different orderings are possible and hence multiple threads can be used
to parallelize the computation. This is most often used for parallelization of
computations such as summations.

For example, the computation of \ess, \cess, normalizing of weights, etc., are
all parallelized using the reduction pattern within \vsmc.

\subsection{Pipeline}
\label{sub:Pipeline}

A pipeline connects tasks in a \emph{producer-consumer} relationship. A few
computation units are active at the same time. The first one consume the data,
and produce new data to be used by the second, and so on. As the data flows
into the pipeline, each unit has its own work to do and thus computations are
carried out in parallel while the data dependencies are correctly maintained.

There are several applications of pipeline in Monte Carlo computing. For
example, an \mcmc algorithm often needs to compute various convergence
statistics, say $h(X_{0:t})$. Often, this statistic can be written as
$h(X_{0:t}) = h(X_t, h(X_{0:{t-1}}))$. Instead of compute it after all
iterations, one can use one thread to update the \mcmc chain and another one
to compute the statistics, using the pipeline pattern. In this case, the
Markov kernel that update the states is the \emph{producer} and the thread
that update the statistics is the \emph{consumer}.

\section{Modern C++}
\label{sec:Modern C++}

The \cpp programming language \cite{cpp03} was first created to support
object-oriented programming (\oop) on top of the C programming language
\cite{evolutioncpp}.  The features, such as templates, come to the language
fairly late. However, it was found that the \cpp template feature provides a
complete sub-language \cite{cpptemplateturing}. This leads to various new
metaprogramming techniques. Many of them are documented in \cite{moderncpp}
and characterize the modern usage of \cpp. In this section, we introduce two
of these techniques. They are widely used inside the \vsmc library and the
contents here should ease the reading of the following sections for those less
familiar with them. However, we assume the reader has at least some working
knowledge of \cpp, including concepts such as \oop.

\subsection{Templates}
\label{sub:Templates}

\cpp is a static strong type language. It requires the user to declare
variables, functions, and most other kinds of entities using specific types.
However, a lot of code looks the same for different types, especially for
implementation of algorithms. The \cpp template technique allows one to write
generic code to solve a class of problems, while the involved types can be
seamlessly replaced at compile time.

Templates are useful for a few reasons. It reduces duplication of the same
code for multiple types. Though conventional \oop also supports polymorphism
behaviors, they rely on runtime decisions. In contrast, templates rely on
compile time decisions and are more type safe. Templates emphasize that the
same operation can be applied to many types. It allows unlimited extension of
existing functionality. It is possible to have any combination of the allowed
operations on a certain type and the allowed types of a certain operations.
More specifically, given a collection of operations and a collection of types,
each operation may support any subset of the types and each type can support
any subset of the operations. Such combinatorial behavior is difficult to
implement using conventional \oop, where a collection of types have a common
interface that provides a fixed set of operations.

There are two main types of templates in \cpp, function template and class
template.

\subsubsection{Function template}
\label{ssub:Function template}

The following lines define a simple function template,
\cppfile{snippet/function_template_def.cpp}
This template definition specifies a \emph{family} of functions that returns
the maximum of two values, which are passed as function parameters
\cppinline{a} and \cppinline{b}. The type of these parameters is left open as
\emph{template parameter} \cppinline{T}.

In this template definition, the assumption about \cppinline{T} is that the
operator \cppinline{<} is properly defined. It does not matter whether
\cppinline{T} is a fundamental type or a class type with this operator defined
by the user. The actual types are deduced at compile time when the function
template is used.

\subsubsection{Class template}
\label{ssub:Class template}

Class template is similar to function template. A class template define a
family of classes. For example,
\cppfile{snippet/class_template_def.cpp}
defines a \cppinline{Stack} class template. For simplicity, some edge cases
and exceptional situations such as calling \cppinline{top} on an empty
\cppinline{Stack} is not handled here. This class template can be used as the
following,
\cppfile{snippet/class_template_usage.cpp}
As we can see, to use a class template, one \emph{explicitly} supply the type
of the template parameter. Unlike function template, there is no template
parameter deduction here.

\subsection{Callable objects}
\label{sub:Callable objects}

\vsmc is a framework for constructing generic \smc samplers. It relies on the
user to write callback functions to perform application specific operations,
such as updating particles. In this section, we introduce the few forms of
callback that are supported by the library. Collectively, they are also called
\emph{callable objects}, meaning that they support the function calling syntax
though they may not be functions.

A callable object, say \cppinline{callable}, is similar to a function in the
sense that it has a return type and a parameter list as its signature. It can
be used with the syntax,
\begin{cppcode}
callable( /* arguments */ );
\end{cppcode}
However, the object may or may not be a function. There are three ways to
define a callable objects, function pointer, functor and \cppoo lambda
expression. Function pointer is the main way of passing callback in C. The
other two are introduced later.

The library also use type erasures, introduced later in this section, to
enforce certain interfaces. The benefits of techniques introduced below
increases the productivity and flexibility of the library compared to
conventional techniques of passing callback through function pointer.

\subsubsection{Functors}
\label{ssub:Functors}

Consider the simple problem, sort a vector $\{x_i\}_{i=1}^N$ according to the
values $y_i = f(x_i)$. We may define a function template to solve this
problem,
\cppfile{snippet/sort_f.cpp}
The function template \cppinline{sort_f} expects input and output as pointers.
In addition, it expects a callable object \cppinline{f}, which accepts a
variable of type \cppinline{double} as its input and return a number that can
be assigned to a variable of type \cppinline{double}.

One way to define such a callable object is to use \emph{functor}, a class
type with \cppinline{operator()} properly defined. For example,
\begin{cppcode}
struct F
{
    double operator() (double x) const { return x * x; }
};
sort_f(input, output, F());
\end{cppcode}
Here we created this object in the function call of \cppinline{sort_f}. It can
also be used as,
\begin{cppcode}
F f;
double y = f(3); // y <- 9
\end{cppcode}

\subsubsection{Lambda expressions}
\label{ssub:Lambda expressions}

Another way, introduced in \cppoo, is a new feature called \emph{lambda
  expression}. It is also called \emph{local function} or \emph{closure} in
other programming languages. It allows us to define callable object \emph{on
  site}. Here is an example,
\begin{cppcode}
sort_f(input, output, [] (double x) { return x * x; });
\end{cppcode}
The full declaration of a lambda expression is as the following,
\begin{cppcode}
[ /* capture */ ] ( /* parameters */ ) /* mutable */
/* exception specification */
/* attribute specification */
-> /* return type */ { /* body * };
\end{cppcode}
The \cppinline{/* mutable */} part can be either empty or the keyword
\cppinline{mutable}, which allows the body to modify captured parameters
(explained soon). The exception specification is similar to a normal function
and the attribute specification is a new feature for all functions in \cppoo,
that specifies things like parameter passing conventions among other things,
which we will not go into details. The part \cppinline{-> /* return type */}
specifies the return type of the lambda expression. If omitted, it is deduced
from the body. And if the body does contain any \cppinline{return} statement,
it is deduced to be \cppinline{void}. If the expression takes no arguments,
the parameter list can also be omitted.

The \cppinline{/* capture */} specifies which symbols visible at the scope of
the definition of the lambda expression will be visible inside the body. There
are a few forms,
\begin{description}
  \item[\cppinline{[a, &b]}] captures \cppinline{a} by value and \cppinline{b}
    by reference.
  \item[\cppinline{[this]}] captures the \cppinline{this} pointer by value.
  \item[\cppinline{[=]}] captures all automatic variables \emph{used} in the
    body by value.
  \item[\cppinline{[&]}] captures all automatic variables \emph{used} in the
    body by reference.
  \item[\cppinline{[]}] captures nothing.
\end{description}

\subsubsection{Type erasures}
\label{ssub:Type erasures}

In the example above, the function template \cppinline{sort_f} does not
actually enforce the signature of the function or functor, in contrast to the
definition of it that takes a function pointer as an argument. For example,
the following is perfect valid \cpp,
\begin{cppcode}
int h (int x) { return x * x };
sort_f(input, output, &h);
\end{cppcode}
while it may not be what one wants. The use of \cppinline{h} with
\cppinline{sort_f} is perhaps an typo. The type erasure in \cppoo{},
\cppinline{std::function}, provides a solution to this problem. A \emph{type
  erasure} can convert various types of objects into a single type. Below is a
basic usage of \cppinline{std::function},
\begin{cppcode}
#include <functional> // the header that defines std::function

std::function<double (double)> f;
F f_obj;
f = f_obj;  // Correct
f = &h;     // ERROR: h does not has the required signature.
\end{cppcode}
Now we can redefine the function \cppinline{sort_f} as,
\cppfile{snippet/sort_f_erasure.cpp}
The \vsmc library makes extensive use of the type erasure to enforce certain
callback interfaces. When \cppoo features are not available, the \boost
library provides the same functionality through \cppinline{boost::function}.
See \cite{vsmcjss} for details of how \vsmc choose between \cppoo and \boost
libraries.
