\chapter{Sequential Monte Carlo for Bayesian Computation}
\label{cha:Sequential Monte Carlo for Bayesian Computation}

As reviewed in Chapter~\ref{cha:Monte Carlo Methods}, \mcmc algorithms, though widely used for the purpose of Bayesian computation, have many limitations. Algorithms such as \rjmcmc are conceptually appealing, yet often difficult to design in practice. Other algorithms such as the Metropolis-Hastings algorithm and the Gibbs sampling, though provide generic frameworks, within which problems in many fields can be solved, the design of efficient, high performance algorithms still requires considerable expertise and sometimes extensive experience.

In recent years, there is a tendency of considering population based algorithms. A common theme in these algorithms is that, instead of simulating directly from a complex target distribution, related yet simpler distributions are used to ``help'' the simulation. One such algorithm, which is essentially a generalization of the Metropolis-Hastings algorithm, population \mcmc is reviewed in Section~\ref{sub:Population mcmc}, in which the easier to simulate distribution ``lends'' information to the target and accelerates its mixing. Another population based algorithm, \smc sampler, is the central topic of this chapter

Sequential Monte Carlo (\smc) samplers, in various forms \draftinline{have} been around for many years and \draftinline{are} widely used in many fields. Until recently there has been little interest in using them for Bayesian model comparison for a few reasons. One of the more important one is that, when \mcmc algorithms are available, an \smc sampler could cost more computational resources than a well designed \mcmc sampler. However we believe there are at least two important reasons that \smc can be preferable to \mcmc for the purpose of Bayesian model comparison in many interesting problems. First, it provides a generic and robust framework for simulation from complex distributions that are difficult for \mcmc algorithms, especially for high dimensional multimodal ones. Though it is not impossible to design \mcmc algorithms for these problems, it can be hugely difficult in practice. The \smc framework provides an alternative that is easy to use. It has the potential to enable statisticians to construct more realistic, useful models that were previously difficult to use due to the computational complexity. Second, most Monte Carlo algorithms has to be implemented on computers to be useful. Therefore it is \draftinline{realistic to} consider the trend of today's computer technologies, in particular, parallel computing. \smc is much more suitable for this kind of computing than conventional \mcmc. As we will see later, \smc has certain advantages over some other parallelized algorithms.

In this chapter, we first give a review of \smc algorithms in Section~\ref{sec:Sequential Monte Carlo samplers}. It is followed by a section that details the use of \smc in the context of Bayesian model comparison. Next, Section~\ref{sec:Extensions and refinements} develops some extensions and refinements of existing practices. It is followed by a discussion of how the presented framework \draftinline{leads} to automatic and generic algorithms. This chapter is concluded with extensive empirical performance \draftinline{study} of various proposed strategies.

\section{Sequential Monte Carlo samplers}
\label{sec:Sequential Monte Carlo samplers}

\smc samplers allow us to obtain, iteratively, collections of weighted samples from a sequence of distributions $\{\pi_t\}_{t\ge0}$ over essentially any random variables on some spaces $\{E_t\}_{t\ge0}$. It is an extension to the \emph{sequential importance sampling} (\sis) and resampling algorithms. In the remainder of this section, sequential importance sampling and resampling algorithms are introduced. \draftinline{Then,} how they are generalized to \smc samplers for the purpose of the current work \draftinline{is} discussed. To simplify the discussion, we will assume that the distributions are continuous and their density functions will also be denoted by $\{\pi_t\}_{t\ge0}$.

\subsection{Sequential importance sampling and resampling}
\label{sub:Sequential importance sampling and resampling}

Sequential importance sampling (\sis) generalizes the importance sampling (see Section~\ref{sec:Importance sampling}) technique for a sequence of distributions $\{\pi_t\}_{t\ge0}$ defined on spaces $\{\prod_{k=0}^tE_k\}_{t\ge0}$. The algorithm operates as the following.

At time $t = 0$, draw $\{X_0^{(i)}\}_{i=1}^N$ from $\eta_0$ and compute the weights $W_0^{(i)} \propto \pi_0(X_0^{(i)})/\eta_0(X_0^{(i)})$. At time $t\ge1$, each sample $X_{0:t-1}^{(i)}$, usually termed \emph{particles} in the literature, is extended to $X_{0:t}^{(i)}$ ($X_{0:0}^{(i)} = X_0^{(i)}$) by sampling from a proposal distribution $q_t(\cdot|X_{0:t-1}^{(i)})$. The weights are recalculated as $W_t^{(i)} \propto \pi_t(X_{0:t}^{(i)})/\eta_t(X_{0:t}^{(i)})$ where
\begin{equation}
  \eta_t(X_{0:t}^{(i)}) =
  \eta_{t-1}(X_{0:t-1}^{(i)})q_t(X_{0:t}^{(i)}|X_{0:t-1}^{(i)})
\end{equation}
and thus
\begin{align}
  W_t^{(i)} \propto \frac{\pi_t(X_{0:t}^{(i)})}{\eta_t(X_{0:t}^{(i)})}
  &= \frac{\pi_t(X_{0:t}^{(i)})\pi_{t-1}(X_{0:t-1}^{(i)})}
  {\eta_{t-1}(X_{0:t-1}^{(i)})q_t(X_{0:t}^{(i)}|X_{0:t-1}^{(i)})
    \pi_{t-1}(X_{0:t-1}^{(i)})} \notag\\
  &= \frac{\pi_t(X_{0:t}^{(i)})}
  {q_t(X_{0:t}^{(i)}|X_{0:t-1}^{(i)})\pi_{t-1}(X_{0:t-1}^{(i)})}W_{t-1}^{(i)}.
  \label{eq:si}
\end{align}
The importance sampling approximation of $\Exp_{\pi_t}[\varphi_t(X_{0:t})]$ can be obtained using $\{W_t^{(i)},X_{0:t}^{(i)}\}_{i=1}^N$, where $\varphi_t$ is some function of interest.

However, this approach fails as $t$ becomes large. The weights tend to become concentrated on a few particles as the discrepancy between $\eta_t$ and $\pi_t$ becomes larger. Resampling techniques are applied such that, a new particle system $\{\bar{W}_t^{(i)},\bar{X}_{0:t}^{(i)}\}_{i=1}^M$ is obtained with the property,
\begin{equation}
  \Exp\Square[Big]{\sum_{i=1}^M\bar{W}_t^{(i)}\varphi_t(\bar{X}_{0:t}^{(i)})}
  = \Exp\Square[Big]{\sum_{i=1}^NW_t^{(i)}\varphi_t(X_{0:t}^{(i)})}
  \label{eq:resample}
\end{equation}
where $\varphi_t$ is the function of interest and both $\{\bar{W}_t^{(i)}\}_{i=1}^N$ and $\{W_t^{(i)}\}_{i=1}$ are normalized weights, that is they are scaled such that they sum up to one. In other words, the resampling step does not change the expectation of the estimate. In practice, the resampling algorithm is usually chosen such that $M = N$ and $\bar{W}^{(i)} = 1/N$ for $i=1,\dots,N$. Resampling can be performed at each iteration $t$ or adaptively based on some criteria of the discrepancy between the distribution of the particles and the target distribution $\pi_t$, accumulated since the last time resampling was performed. One popular quantity used to monitor this discrepancy is \emph{effective sample size} (\ess), introduced by \cite{Liu:1998iu}, defined as
\begin{equation}
  \ess_t = \frac{1}{\sum_{i=1}^N (W_t^{(i)})^2}
\end{equation}
where $\{W_t^{(i)}\}_{i=1}^N$ are the normalized weights. Resampling can be performed when $\ess\le \alpha N$ where $\alpha\in[0,1]$.

The common practice of resampling is to replicate particles with large weights and discard those with small weights. In other words, instead of generating a random sample $\{\bar{X}_{0:t}^{(i)}\}_{i=1}^N$ directly, a random sample of integers $\{R_t^{(i)}\}_{i=1}^N$ is generated, such that $R_t^{(i)} \ge 0$ for $i = 1,\dots,N$ and $\sum_{i=1}^N R_t^{(i)} = N$. Each particle value $X_{0:t}^{(i)}$ is then replicated $R_t^{(i)}$ times in the new particle system. The distribution of $\{R_t^{(i)}\}_{i=1}^N$ should fulfill the requirement of Equation~\eqref{eq:resample}. One such distribution is a multinomial distribution of size $N$ and weights $(W_t^{(i)},\dots,W_t^{(N)})$ and the resulting algorithm is called the \emph{multinomial resampling}. See \cite{Douc:2005wa} for some widely used resampling algorithms. Here we briefly review some of the most commonly used algorithms besides multinomial resampling.

\paragraph{Residual resampling}

This was introduced in \cite{Liu:1998iu}. In this approach, for $i = 1,\dots,N$, we have
\begin{equation}
  R_t^{(i)} = \Floor{NW_t^{(i)}} + \bar{R}_t^{(i)}
\end{equation}
where $\Floor{}$ denotes the integer part and $\{\bar{R}_t^{(i)}\}_{i=1}^N$ is distributed according to a multinomial distribution with size $N - \bar{N}_t$ and weights $(\bar{W}_t^{(i)},\dots,\bar{W}_t^{(N)})$ with
\begin{equation*}
  \bar{N}_t = \sum_{i=1}^N\Floor{NW_t^{(i)}} \qquad\text{and}\qquad
  \bar{W}_t^{(i)} = \frac{NW_t^{(i)} - \Floor{NW_t^{(i)}}}{N - \bar{N}_t}
\end{equation*}
It was shown that residual resampling can lead to significant variance reduction for the importance sampling estimator when compared to the multinomial resampling \cite{Douc:2005wa}. It is easy to see that, unlike multinomial resampling, in residual resampling, the replication number $R_t^{(i)}$ will be no less than $NW_t^{(i)} - 1$. The next two resampling algorithms also share this property.

\paragraph{Stratified resampling}

This can be seen in \cite{Kitagawa:1996vx}. Let $Q$ denote the generalized inverse function of the cumulative distribution function of a multinomial distribution with size $N$ and weights $(W_1,\dots,W_N)$. That is $Q(x) = i$ for $x\in(\sum_{j=1}^{i-1}W_j,\sum_{j=1}^iW_j]$. The stratified resampling proceeds by first drawing uniform random variates $U_t^{(i)}$ on $((i-1)/N, i/N]$ for $i = 1,\dots,N$, and then setting $I_t^{(i)} = Q(U_t^{(i)})$. The new particles system is formed by $\{X_{0:t}^{(I_t^{(i)})}\}_{i=1}^N$. This algorithm also results in smaller variance of the importance sampling estimator than that of the multinomial resampling \cite{Douc:2005wa}.

\paragraph{Systematic resampling}

This was mentioned in \cite{Whitley:1994vx}. Similar to the stratified resampling, the systematic resampling also uses the inversion method. However, instead of generating $N$ uniform random variates, it only generates one $U_t$ from $(0, 1/N]$ and deterministically set $U_t^{(i)} = (i - 1)/N + U_t$ for $i = 1,\dots,N$. Though it has the most straightforward implementation among all the algorithms introduced so far, it is more complicated to study the behavior of the conditional variance of the generated samples. As shown in \cite{Douc:2005wa}, there exists counter-examples that the systematic resampling does not outperform the multinomial resampling.

\paragraph{Combination of residual and stratified/systematic resampling}

Both the stratified and systematic resampling can be used together with the residual resampling. It operates by first \draftinline{computing} the integer part and the residual of $NW_t^{(i)}$ for $i = 1,\dots,N$, and then stratified or systematic resampling is performed using the residuals as weights. It has the advantage that the resulting algorithm \draftinline{provides} better performance than each of the algorithms involved \cite{Douc:2005wa}.

There are other specialized resampling algorithms. The algorithms shown above have a common drawback. They require the knowledge of all the weights being available before the algorithm can proceed. Therefore, in some situations the performance of \smc algorithms can be limited by the fact that the resampling step cannot be parallelized. Parallelized resampling is an area being actively researched (e.g., \cite{Jun:2011vx,Murray:2013vx}). However, we will not discuss such specialized algorithms in this thesis.

\subsection[SMC samplers]{\protect\smc samplers}
\label{sub:SMC Samplers}

\smc samplers generalize the \sis algorithm for a sequence of distributions $\{\pi_t\}_{t\ge0}$ over essentially any random variables on some spaces $\{E_t\}_{t\ge0}$, by constructing a sequence of auxiliary distributions $\{\tilde\pi_t\}_{t\ge0}$ on spaces of increasing dimensions,
\begin{equation}
  \tilde\pi_t(x_{0:t})=\pi_t (x_t) \prod_{s=0}^{t-1} L_s(x_{s+1},x_s),
\end{equation}
where the sequence of Markov kernels $\{L_s\}_{s=0}^{t-1}$, termed \emph{backward kernels}, is formally arbitrary but critically influences the estimator variance. See \cite{DelMoral:2006hc} for further details and guidance on the selection of these kernels (also see Section~\ref{sub:Optimal and suboptimal backward kernels}).

Standard \sis and resampling algorithms can then be applied to the sequence of synthetic distributions, $\{\tilde\pi_t\}_{t\ge0}$. The calculation of the importance weights is straightforward. At time $t-1$, assume that a set of weighted particles approximating $\tilde\pi_{t-1}$ is available, $\{W_{t-1}^{(i)},X_{0:t-1}^{(i)}\}_{i=1}^N$, then at time $t$, the path of each particle is extended with a Markov kernel say, $K_t(x_{t-1}, x_t)$ and the set of particles $\{X_{0:t}^{(i)}\}_{i=1}^N$ reach the distribution $\eta_t(x_{0:t}^{(i)}) = \eta_0(x_0^{(i)})\prod_{k=1}^tK_t(x_{t-1}^{(i)}, x_t^{(i)})$ (assuming no resampling has occurred), where $\eta_0$ is the initial distribution of the particles. To correct the discrepancy between $\eta_t$ and $\tilde\pi_t$, Equation~\eqref{eq:si} is applied to calculate the new weights,
\begin{equation}
  W_t^{(i)} \propto \frac{\tilde\pi_t(X_{0:t}^{(i)})}{\eta_t(X_{0:t}^{(i)})}
  = \frac{\pi_t(X_t^{(i)})\prod_{s=0}^{t-1}L_s(X_{s+1}^{(i)}, X_s^{(i)})}
  {\eta_0(X_0^{(i)})\prod_{k=1}^tK_k(X_{k-1}^{(i)},X_k^{(i)})}
  \propto \tilde{w}_t(X_{t-1}^{(i)}, X_t^{(i)})W_{t-1}^{(i)}
\end{equation}
where $\tilde{w}_t$, termed the \emph{incremental weights}, are calculated as,
\begin{equation}
  \tilde{w}_t(X_{t-1}^{(i)},X_t^{(i)}) =
  \frac{\pi_t(X_t^{(i)})L_{t-1}(X_t^{(i)}, X_{t-1}^{(i)})}
  {\pi_{t-1}(X_{t-1}^{(i)})K_t(X_{t-1}^{(i)}, X_t^{(i)})}.
\end{equation}
If $\pi_t$ is only known up to a normalizing constant, say $\pi_t(x_t) = \gamma_t(x_t)/Z_t$, then we can use the \emph{unnormalized} incremental weights
\begin{equation}
  w_t(X_{t-1}^{(i)},X_t^{(i)}) =
  \frac{\gamma_t(X_t^{(i)})L_{t-1}(X_t^{(i)}, X_{t-1}^{(i)})}
  {\gamma_{t-1}(X_{t-1}^{(i)})K_t(X_{t-1}^{(i)}, X_t^{(i)})}
\end{equation}
for importance sampling estimation. Further, with the normalized weights of the last generation of the particle system, $\{W_{t-1}^{(i)}\}_{i=1}^N$, we can estimate the ratio of normalizing constant $Z_t/Z_{t-1}$ by
\begin{equation}
  \frac{\hat{Z}_t}{Z_{t-1}} =
  \sum_{i=1}^N W_{t-1}^{(i)}w_t(X_{t-1}^{(i)},X_t^{(i)}).
  \label{eq:ratio normalized}
\end{equation}
Iteratively, the ratio of the normalizing constants between the initial distribution $\pi_0$ and some target $\pi_T$, $T\ge1$ can be estimated. The incremental weights clearly depend on the choice of the backward kernels. See \cite{DelMoral:2006hc} and Section~\ref{sub:Optimal and suboptimal backward kernels} for details on calculating the incremental weights.

\subsection{Sequence of distributions}
\label{sub:Sequence of distributions}

There are many ways to specify the sequence of distributions. For many applications, such a sequence arises from the problem setting naturally.

In \cite{Chopin:2002hg} a data tempering scheme was considered in the context of Bayesian inference for static parameters. Suppose data $\data = (y_1, \dots, y_n)$ are available and it is of interest to inference the posterior distribution of some parameter vector $\theta$, $\pi(\theta|\data)$. Then one can construct the following sequence of distributions $\{\pi_t\}_{t=1}^n$,
\begin{equation}
  \pi_t(\theta) = \pi(\theta|y_1,\dots,y_t).
\end{equation}
That is, the data is introduced one by one into the posterior. However, this scheme can be sensitive to the order of data being introduced. A modification is to introduce a batch of data at each iteration, also introduced in \cite{Chopin:2002hg}. The number of data points to be incorporated in each iteration can still be difficult to determine. The more data points introduced at each step, the more degeneracy (measured by, e.g., \ess) will be induced. It is natural to consider introducing data such that a constant level of degeneracy is maintained. It is intuitive to see that with enough data (large $t$) already introduced, the addition of the same amount of data will have less influence on the posterior than when there have only been a few data points (small $t$). It was shown in \cite{Chopin:2002hg} that, to maintain a constant level of degeneracy, it can be expected that the number of data points at each step increases geometrically.

Another generic scheme is called the \emph{geometric path}. Given the target distribution $\pi$ and another distribution $\eta$, which usually has the same support but heavier tails than that of $\pi$, a sequence of distributions $\{\pi_t\}_{t=0}^T$ can be constructed,
\begin{equation}
  \pi_t(x) = \pi(x)^{\alpha(t/T)}\eta(x)^{1-\alpha(t/T)}
\end{equation}
where $\alpha: [0,1] \to [0,1]$ is a monotonically increasing mapping with $\alpha(0) = 0$ and $\alpha(1) = 1$. Some variants of this scheme adapted particularly for the purpose of Bayesian modeling can be seen in Section~\ref{sub:smc1: An all-in-one approach} and~\ref{sub:smc2: A direct-evidence-calculation approach}. The sequence of distributions moves smoothly from $\eta$, which is usually easy to sample from or to construct an efficient proposal distribution for, towards the target distribution $\pi$. However, this scheme has one important drawback. For a high dimensional target with many well separated modes, it can be difficult for $\eta$ (or its proposal distribution) to produce samples within each of all the modes and the sampler may never reach part of the support of the target distribution $\pi$. This problem can be partially solved by increase the number of particles.

Despite this limitation, the geometric scheme has a significant advantage as we will see later (Section~\ref{sub:Optimal and suboptimal backward kernels} and~\ref{sub:Adaptive specification of distributions}). In short, when combined with certain transition kernels and backward kernels, it allows easy computation of the weights using quantities already computed in the last iteration without actually simulating the samples of the current iteration. Therefore it provides a way to conduct adaptive sampling with low computational cost.

There are other sequences, which often have particular use for certain applications. For example, for global optimization of a function $f$, such that $\int f(x)\intd x <\infty$ (that is, it can be normalized into a density function), one can simulate from a sequence of distributions, $\{\pi_t\}_{t\ge0}$, defined by,
\begin{equation}
  \pi_t(x) \propto f(x)^{\alpha(t)}
\end{equation}
where $\alpha: [0,\infty) \to [0,\infty)$ is a monotonically increasing mapping with $\alpha(t) \to \infty$ as $t \to \infty$. The sequence of distributions will concentrate more and more around the modes of $f$ (see also \cite{Marinari:1992vx}).

\subsection{Sequence of transition kernels}
\label{sub:Sequence of transition kernels}

It is easy to see that, the optimal proposal kernel is $K_t(x_{t-1}, x_t) = \pi_t(x_t)$, in the sense of minimizing the Monte Carlo variance of the importance weights. However, this choice is not possible except for trivial cases. Some sensible alternatives have been proposed in the past.

One approach is to use independent proposals, $K_t(x_{t-1}, x_t) = \mu_t(x_t)$ for some distribution $\mu_t$ at each iteration. Usually, $\mu_t$ belongs to a family of distributions with parameters determined by certain statistics of the particle system of the last generation, for example, a multivariate Normal distribution with the mean vector and the covariance matrix estimated from current samples. For general use, this can be overly restrictive and the performance can be difficult to calibrate, especially in high dimensional problems. In this situation, it is difficult for the independent proposal to capture the characteristics of the target distribution without knowing it in advance. And thus it can lead to large variance of importance weights and poor performance of the estimator.

An important alternative, advocated in \cite{DelMoral:2006hc} is to use \mcmc kernels targeting $\pi_t$. This strategy is particularly justified if the sequence of distributions moves smoothly or the kernel is fast mixing. When the sequence of distributions moves slowly from one to another, that is $\pi_t$ is not very different from $\pi_{t-1}$, and thus samples from $\eta_{t-1}$ is a good approximation to $\pi_t$, the kernel is likely to successfully move particles towards high probability regions of $\pi_t$. What makes it more attractive is the fact that we can use the vast literature on the design of efficient \mcmc algorithms to build the proposal distributions. In addition, as we will see very soon, when combined with certain backward kernels, this approach enables us to calculate the importance weights without actually simulating samples. And therefore it leads to low computational cost adaptive algorithms that can improve the performance considerably.

\subsection{Optimal and suboptimal backward kernels}
\label{sub:Optimal and suboptimal backward kernels}

The sequence of backward kernels $\{L_t\}_{t=0}^{T-1}$ should be optimized with respect to the sequence of transition kernels $\{K_t\}_{t=1}^T$. Let $\eta_t(x_t)$ denote the marginal distribution of $X_t$. That is,
\begin{equation}
  \eta_t(x_t) = \eta_0(x_0)\prod_{k=1}^tK_k(x_{k-1},x_k)
\end{equation}
if no resampling has occurred and,
\begin{equation}
  \eta_t(x_t) = \pi_l(x_l)\prod_{k=l+1}^tK_k(x_{k-1},x_k)
\end{equation}
if the last resampling occurs at time $l$.

As shown in Proposition~1{} in \cite{DelMoral:2006hc}, the backward kernel $L_{t-1}(x_t, x_{t-1})$ that minimizes the variance of unnormalized importance weights is given by,
\begin{equation}
  L_{t-1}^{\opt}(x_t,x_{t-1}) =
  \frac{\eta_{t-1}(x_{t-1})K_t(x_{t-1},x_t)}{\eta_t(x_t)}
\end{equation}
and in this case the weights are,
\begin{equation}
  W_t^{(i)} \propto \frac{\pi_t(X_t^{(i)})}{\eta_t(X_t^{(i)})}.
\end{equation}
The marginal $\eta_t(x_t)$ is typically not available and thus the above optimal backward kernel cannot be used in practice.

One sensible alternative is to substitute $\pi_{t-1}$ for $\eta_{t-1}$, that is,
\begin{equation}
  L_{t-1}(x_t,x_{t-1}) =
  \frac{\pi_{t-1}(x_{t-1})K_t(x_{t-1},x_t)}
  {\int \pi_{t-1}(x_{t-1})K_t(x_{t-1},x_t)\intd x_{t-1}}.
  \label{eq:subopt back kernel}
\end{equation}
This approach is justified if the particle system has has been resampled at time $t-1$, in which case $\eta_{t-1}$ is indeed equal to $\pi_{t-1}$ or when resampling was at least performed occasionally such that the degeneracy between $\eta_{t-1}$ and $\pi_{t-1}$ is controlled. The incremental weights can be computed if the integration above can be computed. Usually this is done through the unnormalized distribution $\gamma_{t-1}$ instead of $\pi_{t-1}$. When $\gamma_{t-1}$ is known analytically, the unnormalized incremental weights are,
\begin{equation}
  w_t(X_{t-1}^{(i)},X_t^{(i)}) =
  \frac{\gamma_t(X_t^{(i)})}
  {\int\gamma_{t-1}(x_{t-1})K_t(x_{t-1},X_t^{(i)})\intd x_{t-1}}.
  \label{eq:inc weight subopt}
\end{equation}
The requirement of the knowledge of the above integration can limit the use of the kernel in some applications.

When using an \mcmc kernel $K_t$ that is invariant to $\pi_t$ as the transition kernel, and when $\pi_{t-1}\approx\pi_t$, by substitute $\pi_t$ for $\pi_{t-1}$, Equation~\eqref{eq:subopt back kernel} becomes,
\begin{align}
  L_{t-1}(x_t,x_{t-1})
  &= \frac{\pi_t(x_{t-1})K_t(x_{t-1},x_t)}
  {\int \pi_t(x_{t-1})K_t(x_{t-1},x_t)\intd x_{t-1}} \notag\\
  &= \frac{\pi_t(x_{t-1})K_t(x_{t-1},x_t)}{\pi_t(x_t)}
  \label{eq:subopt back kernel mcmc}
\end{align}
where the second equation is due to the fact that $K_t$ is invariant to $\pi_t$. It is easy to see that the unnormalized incremental weights are,
\begin{equation}
  w_t(X_{t-1}^{(i)},X_t^{(i)}) =
  \frac{\gamma_t(X_{t-1}^{(i)})}{\gamma_{t-1}(X_{t-1}^{(i)})}.
  \label{eq:inc weight subopt mcmc}
\end{equation}
Note that, the incremental weights no longer depend on the samples from iteration $t$, $\{X_t^{(i)}\}_{i=1}^N$. Therefore, it can be calculated before the sampling step, which moves the particles according to the kernel $K_t$. Since the incremental weights solely depends on the specification of $\gamma_t$, which usually can be computed point-wise, given the current samples, it is possible to specify $\gamma_t$ (and therefore $\pi_t$) according to the calculated weights using information from the current samples before carrying out the actual simulation of the current iteration.

However the expression~\eqref{eq:inc weight subopt mcmc} is not without drawbacks. Compared to the expression~\eqref{eq:inc weight subopt}, which is more intuitive since it considers the transition kernel $K_t$, which depends on the current samples, it benefits less from fast mixing kernels. If $\pi_t$ is not close to $\pi_{t-1}$, then the variance of the incremental weights is likely to be large even when the kernel $K_t$ mixes fast. Indeed, later we will show empirically that, it is preferable to use more distributions rather than using multiple passes of \mcmc moves in a single iteration, provided that they use the same computational resources.

\section{Application to Bayesian model comparison}
\label{sec:Application to Bayesian model comparison}

The application of \smc samplers to Bayesian model comparison is straightforward. However, it has been overlooked in recent years. In this section, we outline common strategies of using \smc samplers for Bayesian model comparison. In the next section, we introduce some innovative extensions and refinements to existing practices.

As reviewed in Section~\ref{sub:Model choice problems}, the problem of interest is characterizing the posterior distribution over $\{\calM_k\}_{k\in\calK}$, a set of possible models, with model $\calM_k$ having parameter vector $\theta_k\in\Theta_k$ which also usually need to be inferred. Given prior distributions $\pi(\calM_k)$ and $\pi(\theta_k|\calM_k)$ and the likelihood function $p(\data|\theta_k,\calM_k)$, we seek the posterior distribution $\pi(\calM_k|\data)\propto p(\data|\calM_k)\pi(\calM_k)$. There are three fundamentally different approaches to the computations,
\begin{enumerate}
  \item Calculate posterior model probability distribution $\pi(\calM_k|\data)$ directly.
  \item Calculate the evidence, the marginal likelihood $p(\data|\calM_k)$, of each model.
  \item Calculate pairwise evidence ratios, the Bayes factor $B_{k_1k_2}$ for two models $\calM_{k_1}$ and $\calM_{k_2}$ directly.
\end{enumerate}
Each approach admits a natural \smc strategy.

\subsection[SMC1: An all-in-on approach]{\smc[1]: An all-in-one approach}
\label{sub:smc1: An all-in-one approach}

One could consider obtaining samples from the same distribution employed in the \rjmcmc (see Section~\ref{sub:Reversible jump mcmc}) approach to model comparison, namely,
\begin{equation}
  \pi^{(1)}(\calM_k,\theta_k) \propto
  \pi(\calM_k)\pi(\theta_k|\calM_k)p(\data|\theta_k,\calM_k)
\end{equation}
which is defined on the disjoint union space
$\bigcup_{k\in\calK}(\{\calM_k\}\times\Theta_k)$.

One obvious \smc approach is to define a sequence of distributions $\{\pi_t^{(1)}\}_{t=0}^T$ such that $\pi^{(1)}_0$ is easy to sample from, $\pi_{T}^{(1)} = \pi^{(1)}$ and the intermediate distributions move smoothly between them. In the remainder of this section, we use the notation $(\calM_t,\theta_t)$ to denote a random sample on the space $\bigcup_{k\in\calK}(\{\calM_k\}\times\Theta_k)$ at time $t$. One simple approach, which might be expected to work well, is the use of an annealing scheme such that,
\begin{equation}
  \pi^{(1)}_t(\calM_t,\theta_t) \propto \pi(\calM_t)\pi(\theta_t|\calM_t)
  p(\data|\theta_t,\calM_t)^{\alpha(t/T)},
  \label{eq:geometry_1}
\end{equation}
for some monotonically increasing $\alpha:[0,1]\to[0,1]$ with $\alpha(0) = 0$ and $\alpha(1) = 1$. Other approaches are possible and might prove more efficient for some problems (such as the ``data tempering'' approach (see Section~\ref{sub:Sequence of distributions}), which \cite{Chopin:2002hg} proposed for parameter estimation which could easily be incorporated in our framework), but this strategy provides a convenient generic approach. These choices lead to Algorithm~\ref{alg:smc1}.

\begin{algorithm}
\begin{algorithmic}
  \tophrule
  \STATE \emph{Initialisation:} Set $t\leftarrow0$.
  \STATE\STATESKIP Sample $X_0^{(i)} = (M_0^{(i)},\theta_0^{(i)})\sim\nu$
  for some proposal distribution $\nu$ (usually the joint prior).
  \STATE\STATESKIP Weight $W_0^{(i)} \propto w_0(X_0^{(i)}) =
  {\pi(M_0^{(i)}) \pi(\theta^{(i)}_0|M_0^{(i)})}/
  {\nu(M_0^{(i)},\theta_0^{(i)})}$.
  \STATE\STATESKIP Apply resampling if necessary (e.g., if \ess less than some
  threshold (see Section~\ref{sub:Sequential importance sampling and
    resampling}).

  \STATE \emph{Iteration:} Set $t\leftarrow t + 1$.
  \STATE\STATESKIP Weight $W_t^{(i)} \propto W_{t-1}^{(i)}
  p(\data|\theta_{t-1}^{(i)},M_{t-1}^{(i)})^{\alpha(t/T) - \alpha([t-1]/T)}$.
  \STATE\STATESKIP Apply resampling if necessary.
  \STATE\STATESKIP Sample $X_t^{(i)} \sim K_t(\cdot|X_{t-1}^{(i)})$, a
  $\pi_t^{(1)}$-invariant kernel.

  \STATE \emph{Repeat} the \emph{Iteration} step \emph{until $t = T$}.
  \bottomhrule
\end{algorithmic}
\caption{\smc[1]: An All-in-One Approach to Model Comparison.}
\label{alg:smc1}
\end{algorithm}


This approach might outperform \rjmcmc when it is difficult to design fast-mixing Markov kernels. There are many examples of such an annealed \smc strategy outperforming \mcmc at a given computational cost -- see, for example, \cite{Fan:2008tf,Johansen:2008kp,Fearnhead:2010ua}. Such trans-dimensional \smc has been proposed in several contexts such as \cite{Peters:2005wh}, and an extension was proposed and analyzed by \cite{Jasra:2008bb}.

We include this approach for completeness and study it empirically later. However, the more direct approaches described in the following sections lead more naturally to easy-to-implement strategies with good performance.

\subsection[SMC2: A direct-evidence-calculation approach]{\smc[2]: A direct-evidence-calculation approach}
\label{sub:smc2: A direct-evidence-calculation approach}

An alternative approach would be to estimate explicitly the evidence associated with each model. We propose to do this by sampling from a sequence of distributions for each model, starting from the parameter prior and sweeping through a sequence of distributions to the posterior.

Numerous strategies are possible to construct such a sequence of distributions, but one option is to use for each model $\calM_k\in\calM$, the sequence $\{\pi_t^{(2,k)}\}_{t=0}^{T_k}$, defined by,
\begin{equation}
  \pi_t^{(2,k)}(\theta_t) \propto
  \pi(\theta_t|\calM_k)p(\data|\theta_t,\calM_k)^{\alpha_k(t/T_k)}.
  \label{eq:geometry_2}
\end{equation}
where the number of distributions, $T_k$, and the annealing schedule, $\alpha_k:[0,1]\to[0,1]$, may be different for each model. This leads to Algorithm~\ref{alg:smc2}.

\begin{algorithm}
\begin{algorithmic}
  \tophrule
  \STATE For each model $k \in \calK$ perform the following algorithm.

  \STATE \emph{Initialisation:} Set $t\leftarrow0$.
  \STATE\STATESKIP Sample $\theta_0^{(k,i)}\sim\nu_k$ for some proposal
  distribution $\nu_k$ (usually the parameter prior).
  \STATE\STATESKIP Weight $W_0^{(k,i)} \propto w_0(\theta_0^{(k,i)}) =
  {\pi(\theta_0^{(k,i)}|\calM_k)}/{\nu_k(\theta_0^{(k,i)})}$.
  \STATE\STATESKIP Apply resampling if necessary.

  \STATE \emph{Iteration:} Set $t\leftarrow t + 1$.
  \STATE\STATESKIP Weight $W_t^{(k,i)} \propto W_{t-1}^{(k,i)}
  p(\data|\theta_{t-1}^{(k,i)},\calM_k)^{\alpha_k(t/T_k)-\alpha_k([t-1]/T_k)}$.
  \STATE\STATESKIP Apply resampling if necessary.
  \STATE\STATESKIP Sample $\theta_t^{(k,i)} \sim
  K_t(\cdot|\theta_{t-1}^{(k,i)})$, a $\pi_t^{(k,2)}$-invariant kernel.

  \STATE \emph{Repeat} the \emph{Iteration} step \emph{until $t = T_k$}.
  \bottomhrule
\end{algorithmic}
\caption{\smc[2]: A Direct-Evidence-Calculation Approach.}\label{alg:smc2}
\end{algorithm}


The estimator of the posterior model probabilities depends upon the approach taken to estimate the normalizing constant. Direct estimation of the evidence can be performed using the output of this \smc algorithm and the standard estimator, termed \smc[2]-\ds (see also Equation~\eqref{eq:ratio normalized}), given by,
\begin{equation}
  \sum_{i=1}^N \frac{\pi(\theta_0^{(k,i)}|\calM_k)}{\nu_k(\theta_0^{(k,i)})}
  \times \prod_{t=2}^T \sum_{i=1}^N W_{t-1}^{(k,i)}
  p(\data|\theta_{t-1}^{(k,i)},\calM_k)^{\alpha_k(t/T_k) - \alpha_k([t-1]/T_k)}
  \label{eq:smc2-ds}
\end{equation}
where $W_{t-1}^{(k,i)}$ is the normalized importance \draftinline{weight of} the $i$\xth particle during iteration $t-1$ for model $\calM_k$. An alternative approach to computing the evidence is also worthy of consideration. As has been suggested, and shown to perform well empirically previously \cite{Johansen:2006wm}, it is possible to use all of the samples from every generation of an \smc sampler to approximate the path sampling estimator and hence to obtain an estimate of the ratio of normalizing constants. Section~\ref{sub:Path Sampling via smc2/smc3} provides details for the use of path sampling for both this and other \smc algorithms discussed later.

This approach is appealing for several reasons. One is that it is designed to estimate directly the quantity of interest, the evidence, producing samples from that distribution at the same time. Another advantage of this approach over \smc[1] and the \rjmcmc is that it provides as good a characterization of each model as is required: It is possible to obtain a good estimate of the parameters of every model, even those for which the posterior probability is small. Perhaps most significant is the fact that this approach does not require the design of proposal distributions or Markov kernels which move from one model to another: Each model is dealt with in isolation. Whilst this may not be desirable in every situation, there are circumstances in which efficient moves between models are almost impossible to devise.

This approach also has some disadvantages. In particular, it is necessary to run a separate simulation for each model -- rendering it impossible to deal with countable collections of models (although this is not much \draftinline{of} a substantial problem in many interesting cases). The ease of implementation may often offset this limitation.

\subsection[SMC3: A relative-evidence-calculation approach]{\smc[3]: A relative-evidence-calculation approach}
\label{sub:smc3: A relative-evidence-calculation approach}

A final approach can be thought of as \emph{sequential model comparison}. Rather than estimating the evidence associated with any particular model, we could estimate pairwise evidence ratios directly. The \smc sampler starts with \draftinline{an} initial distribution being the posterior of one model (which could comes from a separate \smc sampler starting from its prior) and moves towards the posterior of another related model. Then the sampler can continue towards another related model.

Given a finite collection of models $\{\calM_k\}_{k\in\calK}$, suppose the models are ordered in a sensible way (e.g., $\calM_{k-1}$ is nested within $\calM_k$ or $\theta_k$ is of higher dimension than $\theta_{k-1}$). For each $k\in\calK$, we consider a sequence of distributions $\{\pi_t^{(3,k)}\}_{t=0}^{T_k}$, such that,
\begin{align*}
  \pi_0^{(3,k)}(\calM,\theta) &=
  \pi(\theta|\data,\calM_k) \bbI_{\{\calM_k\}}(\calM) \\
  \pi_{T_k}^{(3,k)}(\calM,\theta) &=
  \pi(\theta|\data,\calM_{k+1})
  \bbI_{\{\calM_{k+1}\}}(\calM) = \pi_{0}^{(3,k+1)}(\calM,\theta).
\end{align*}
where $(\calM,\theta)$ denote a random variable on the disjoint union space $(\{\calM_k\}\times\Theta_k)\cup(\{\calM_{k+1}\}\times\Theta_{k+1})$. When it is possible to construct an \smc sampler that iterates over this sequence of distributions, the estimate of the ratio of normalizing constants is the Bayes factor estimate of model $\calM_{k+1}$ in favor of model $\calM_k$.

This approach is conceptually appealing, but requires the construction of a smooth path between the posterior distributions of interest. The geometric annealing strategy which has been advocated as a good generic strategy in the previous sections is only appropriate when the support of successive distributions is non-increasing. This is unlikely to be the case in interesting model comparison problems.

Here we consider a sequence of distributions on the disjoint union space $(\{\calM_k\}\times\Theta_k\})\cup(\{\calM_{k+1}\}\times\Theta_{k+1}\})$, with the sequence of distributions $\{\pi_t^{(3,k)}\}_{t=0}^{T_k}$ defined proportional to that of the full posterior (see Section~\ref{sub:Model choice problems}) restricted to the space of these two models,
\begin{equation}
  \pi_t^{(3,k)}(\calM_t,\theta_t) \propto
  \pi_t(\calM_t) \pi(\theta_t|\calM_t) p(\data|\theta_t,\calM_t)
\end{equation}
where $\calM_t\in\{\calM_k,\calM_{k+1}\}$ and the prior of models at time $t$, $\pi_t(\calM_t)$ is defined by
\begin{equation}
  \pi_t(\calM_{k+1}) = \alpha_k(t/T_k)
  \label{eq:smc3_prior}
\end{equation}
for some monotonically increasing $\alpha_k:[0,1]\to[0,1]$ such that $\alpha_k(0) = 0$ and $\alpha_k(1) = 1$. It is clear that the \mcmc moves between iterations need to be similar to those in the \rjmcmc or \smc[1] algorithms. The difference is that instead of efficient exploration of the whole model space, only moves between two models are required and the sequence of distributions employed helps to ensure exploration of both model spaces. The algorithm for this particular sequence of distributions is outlined in Algorithm~\ref{alg:smc3}. It can be extended to other possible sequence of distributions between models.

\begin{algorithm}
\begin{algorithmic}
  \tophrule
  \STATE \emph{Initialisation:} Set $k\leftarrow1$.
  \STATE\STATESKIP Use Algorithm~\ref{alg:smc2} to obtain weighted samples
  for $\pi_{T_1}^{(3,1)}$, the parameter posterior for model $\calM_1$

  \STATE \emph{Relative Evidence Calculation}
  \STATE\STATESKIP Set $k\leftarrow k + 1$, $t\leftarrow0$.
  \STATE\STATESKIP Denote current weighted samples as
  $\{W_0^{(k,i)},X_0^{(k,i)}\}_{i=1}^N$ where $X_0^{(k,i)} =
  (M_0^{(k,i)},\theta_0^{(k,i)})$
  \STATE\STATESKIP Apply resampling if necessary.

  \STATE\STATESKIP \emph{Iteration:} Set $t\leftarrow t + 1$.
  \STATE\STATESKIP\STATESKIP Weight $W_t^{(k,i)} \propto W_{t-1}^{(k,i)}
  {\pi_t(M_{t-1}^{(k,i)})}/{\pi_{t-1}(M_{t-1}^{(k,i)})}$.
  \STATE\STATESKIP\STATESKIP Apply resampling if necessary.
  \STATE\STATESKIP\STATESKIP Sample $(M_t^{(k,i)},\theta_t^{(ki)}) \sim
  K_t(\cdot|M_{t-1}^{(k,i)}\theta_{t-1}^{(k,i)})$, a $\pi_t^{(3,k)}$-invariant
  kernel.

  \STATE\STATESKIP \emph{Repeat the \emph{Iteration} step up to $t = T_k$}.

  \STATE \emph{Repeat} the \emph{Relative Evidence Calculation} step \emph{until
    sequentially all relative evidences are calculated}.
  \bottomhrule
\end{algorithmic}
\caption{\smc[3]: A Relative-Evidence-Calculation Approach to Model Comparison.}
\label{alg:smc3}
\end{algorithm}


An advantage of this approach is that it provides direct estimate of the Bayes factor which is of interest for model comparison purpose while not requiring exploration of as complicated a space as that employed within \rjmcmc or \smc[1]. The estimator of normalizing constants in \smc[3] follows in exactly the same manner as in the \smc[2] case. In \smc[3], the same estimator provides a direct estimate of the Bayes factor.

\subsection{Path sampling via \smc[2]/\smc[3]}
\label{sub:Path Sampling via smc2/smc3}

The estimation of the normalizing constants associated with our sequences of distributions can be achieved by a Monte Carlo approximation to the \emph{path sampling} formulation given by \cite{Gelman:1998ei}. This is similar to the technique for population \mcmc as described in Section~\ref{sub:MCMC Application to Bayesian model comparison}. In the context of \smc, this approach is also very closely related to the use of \draftinline{annealed importance sampling} (\ais) for the same purpose \cite{Neal:2001we} but as will be demonstrated below the incorporation of some other elements of more general \smc algorithms can improve performance at negligible cost. Recall that, given a parameter $\alpha$ which defines a family of distributions, $\{\pi_{\alpha} = \gamma_{\alpha} / Z_\alpha\}_{\alpha \in [0,1]}$ which move smoothly from $\pi_0 = \gamma_0 / Z_0$ to $\pi_1 = \gamma_1 / Z_1$ as $\alpha$ increases from zero to one, one can estimate the logarithm of the ratio of their normalizing constants via a simple integral relationship which holds under very mild regularity conditions,
\begin{equation*}
  \log\Round[bigg]{\frac{Z_1}{Z_0}} =
  \int_{0}^{1} \Exp_{\pi_\alpha} \Square[bigg]{
    \frac{\diff\log\gamma_{\alpha}(X)}{\diff\alpha}
  } \intd\alpha,
\end{equation*}
where the inner expectation is taken with respect to $\pi_{\alpha}$. Note that the sequence of distributions in the \smc[2] and \smc[3] algorithms above, can both be interpreted as belonging to such a family of distributions, with $\alpha = \alpha_k(t/T_k)$, where the mapping $\alpha_k:[0,1]\to[0,1]$ is again monotonically increasing with $\alpha_k(0) = 0$ and $\alpha_k(1) = 1$.

The \smc sampler provides us with a set of weighted samples obtained from a sequence of distributions suitable for approximating this integral. At each iteration $t$ we can obtain an estimate of the expectation within the integral for $\alpha = \alpha_k(t/T)$ via the usual importance sampling estimator, and this integral can then be approximated via numerical integration. Whenever the sequence of distributions employed by \smc[3] has appropriate differentiability it is also possible to employ path sampling to estimate, directly, the Bayes factor via this approach. In general, given an increasing sequence $\{\alpha_t\}_{t=0}^T$ where $\alpha_0 = 0$ and $\alpha_T = 1$, a family of distributions $\{\pi_{\alpha}\}_{\alpha\in[0,1]}$ as before, and an \smc sampler that iterates over the sequence of distribution $\{\pi_t = \pi_{\alpha_t} = \gamma_{\alpha_t}/Z_{\alpha_t}\}_{t=0}^T$, then with the weighted samples $\{W_t^{(i)},X_t^{(i)}\}_{i=1}^N$, and $t = 0,\dots,T$, a path sampling estimator of the ratio of normalizing constants $\Xi_T = \log(Z_1/Z_0)$ can be approximated (using an elementary Trapezoidal scheme) by,
\begin{equation}
  \hat\Xi_{T}^{N} = \sum_{t=1}^T
  \frac{1}{2}(\alpha_t - \alpha_{t - 1})(U_t^N + U_{t-1}^N),
  \label{eq:path_est}
\end{equation}
where
\begin{equation}
  U_t^N = \sum_{i=1}^N
  W_t^{(i)} \frac{\diff\log\gamma_{\alpha}(X_t^{(i)})}{\diff\alpha}
  \Bigm|_{\alpha = \alpha_t}.
\end{equation}

We term these estimators \smc[2]-\ps and \smc[3]-\ps in the followings. The combination of \smc and path sampling is somewhat natural and has been proposed before, e.g., \cite{Johansen:2006wm} although not there in a Bayesian context. Despite the good performance observed in the setting of rare event simulation, the estimation of normalizing constants by this approach seems to have received little attention in the literature. We suspect that this is because of widespread acceptance of the suggestion of \cite{DelMoral:2006hc}, that \smc doesn't outperform \ais when normalizing constants are the object of inference or that of \cite{Calderhead:2009bd} that all simulation-based estimators based around path sampling can be expected to behave similarly. We will demonstrate below that these observations, whilst true in certain contexts, do not hold in full generality.

\section{Extensions and refinements}
\label{sec:Extensions and refinements}

The algorithms introduced in the last section can be seen as straightforward application of the well established \smc algorithms to Bayesian model comparison. By construction, \smc algorithms can be more robust than many \mcmc and other algorithms. However, as with any Monte Carlo algorithms, without careful design, the performance can be far from satisfactory for realistic applications. In this section, we introduce some extensions and refinement that can further improve the presented framework. Of course, they cannot guarantee that the algorithms will perform well for all possible situations. However, they provide robust and reliable solutions for many realistic applications with minimal manual tuning. For more difficult problems, they also provide solid foundations on top of which algorithms with higher performance can be built.

We will use the \pet compartmental model example for illustrative purpose in this section. More comprehensive performance comparisons can be found in Section~\ref{sec:Performance comparison}. We will consider both the simulated and the real data (see Section~\ref{sec:Simulated and real pet data}). For the real data, to ease the presentation, instead of visualizing results for a quarter of a million data sets, we consider three typical voxels, shown in Figure~\ref{fig:typical real pet}. As we can see, they vary considerably in characteristics \draftinline{though} all can be described as ``typical'' \pet data. Most graphical presentations will be for these three data sets while summarizing statistics such as the reduction of variances will be based on simulations for all the real data sets. The purpose of the current work is to advocate robust and self-tuning algorithms. The variability in the data sets provides excellent test examples. In addition, in this section, the models are configured with the non-informative priors without ordering (see Section~\ref{sub:Choice of priors}) and thus the parameters are exchangeable, similar to that of a mixture model. This creates a multimodal posterior surface for models with two or more compartments.

\begin{figure}[t]
  \linespread{1.1}\selectfont
  \includegraphics[width=\linewidth]{fig_src/Typical_PET}
  \caption{Typical real \protect\pet data}
  \label{fig:typical real pet}
\end{figure}


\subsection{Improved univariate numerical integration}
\label{sub:Improved univariate numerical integration}

As seen in the last section, the path sampling estimator requires evaluation of the expectation,
\begin{equation*}
  \Exp_{\pi_\alpha}\Square[bigg]{
    \frac{\diff\log\gamma_{\alpha}(X)}{\diff\alpha}}
\end{equation*}
for $\alpha\in[0,1]$, which can be approximated by importance sampling using samples generated by an \smc sampler operating on the sequence of distributions $\{\pi_t = \pi_{\alpha_t} = \gamma_{\alpha_t}/Z_t\}_{t=0}^T$ directly for $\alpha\in\{\alpha_t\}_{t=0}^T$. For arbitrary $\alpha\in[0,1]$, finding $t$ such that $\alpha\in(\alpha_{t-1},\alpha_t)$, the expectation can be easily approximated using existing \smc samples -- the quantities required in the importance weights to obtain such an estimate have already been calculated during the running of the \smc algorithm and such computations have little computational cost.

As noted by \cite{Friel:2012} we can use more sophisticated numerical integration strategies to reduce the path sampling estimator bias. For example, higher order Newton-Cotes rules rather than the Trapezoidal rule can be implemented straightforwardly. In the case of \smc it is especially straightforward to estimate the required expectations at arbitrary $\alpha$ and thus we can use higher order integration schemes. We can also use numerical integrations which make use of a finer mesh $\{\alpha_t'\}_{t=0}^{T'}$ than $\{\alpha_t\}_{t=0}^T$. Since higher order numerical integrations based on approximations of derivatives obtained from Monte Carlo methods may potentially be unstable in some situations, the second approach can be more appealing in some applications. A demonstration of the bias reduction effect is provided in Section~\ref{sub:pet compartmental model}.

\subsection{Adaptive specification of distributions}
\label{sub:Adaptive specification of distributions}

In settings in which the importance weights at time $t$ depend only upon the samples at time $t-1$, such as that considered here, it is relatively straightforward to consider sample-dependent, adaptive specification of the sequence of distributions (typically by choosing the value of a tempering parameter, such as $\alpha_t = \alpha_k(t/T_k)$ in Algorithm~\ref{alg:smc2}) based upon the current samples. In \cite{Jasra:2010eh} such a method of \draftinline{adaptively} placing the distributions in \smc algorithms based on controlling the rate at which the effective sample size (\ess; \cite{Kong:1994ul}) falls was proposed. With very little computation cost, this provides an automatic method of specifying a tempering schedule in such a way that the \ess decays in a regular fashion. In \cite[Algorithm 2]{Schafer:2011bx} a similar technique is used but by moving the particle system only when it resamples. They are in a setting which would be equivalent to resampling at every time step (with longer time steps, followed by multiple applications of the \mcmc kernel) in our formulation. We advocate resampling only adaptively when \ess is smaller than \draftinline{a} certain preset threshold, and here we propose a more general adaptive scheme for the selection of the sequence of distributions which has significantly better properties when adaptive resampling is employed.

The \ess was designed to assess the loss of efficiency arising from the use \draftinline{of} simple weighted samples (rather than random samples from the distribution of interest) in the computation of expectations. It is obtained by considering a sample approximation of a low order Taylor expansion of the variance of the importance sampling estimator of an arbitrary test function to that of the simple Monte Carlo estimator; the test function itself vanishes from the expression as a consequence of this low order expansion.

In our context, the \ess calculated using the current weight of each particle is simply,
\begin{equation}
  \ess_t = \left[ {\sum_{i=1}^N\left( \frac{W_{t-1}^{(i)}
          w_t^{(i)}}{\sum_{j=1}^NW_{t-1}^{(j)}w_t^{(j)}}\right)^2}
  \right]^{-1} = \frac{\Round[Big]{\sum_{i=1}^NW_{t-1}^{(i)}w_t^{(i)}}^2}
  {\sum_{j=1}^N\Round[Big]{W_{t-1}^{(j)}}^2\Round[Big]{w_t^{(j)}}^2}
\end{equation}
where $\{W_{t-1}^{(i)}\}_{i=1}^N$ denote the \emph{normalized weights} at the end of iteration $t - 1$, and $\{w_t^{(i)}\}_{i=1}^N$ denote the \emph{unnormalized} incremental weights during iteration $t$. It is clearly appropriate to use this quantity (which corresponds to the coefficient of variation of the current normalized importance weights) to assess weight degeneracy and to make decisions about appropriate resampling times \cite{DelMoral:2012jq} but it is rather less apparent that it is the correct quantity to consider when adaptively specifying a sequence of distributions in an \smc sampler.

The \ess of the current sample weights tells us about the accumulated mismatch between proposal and target distributions (on an extended space including the full trajectory of the sample paths) since the last resampling time. Fixing either the relative or absolute reduction in \ess between successive distributions does \emph{not} lead to a common discrepancy between successive \emph{target} distributions unless resampling is conducted after every iteration as will be demonstrated below.

When specifying a sequence of distributions it is natural to aim for a similar discrepancy between each pair of successive distributions. In the context of our setting, the natural question to ask is consequently, how large can we make $\alpha_t - \alpha_{t-1}$ whilst ensuring that $\pi_{t}$ remains sufficiently similar to $\pi_{t-1}$. One way to measure the discrepancy would be to consider how good an importance sampling proposal $\pi_{t-1}$ would be for the estimation of expectations under $\pi_t$ and a natural way to measure this is via the sample approximation of a Taylor expansion of the relative variance of such an estimator exactly as in the \ess.

The exact \ess of an importance sample of size $N$ with proposal $\pi_{t-1}$ and target $\pi_t$ is defined as \cite{Kong:1994ul},
\begin{equation}
  \text{Exact }\ess_t =
  \frac{N}{1 + \var_{\pi_{t-1}}\Square[Big]{\frac{\pi_t(X)}{\pi_{t-1}(X)}}}
  \label{eq:essexact}
\end{equation}
which is widely approximated by the empirical equivalent, replacing the denominator with the empirical mean squared normalized importance weights.

In the context of adaptive specification of an \smc tempering schedule, we are interested in the discrepancy between adjacent distributions, $\pi_{t-1}$ and $\pi_t$, and so the \ess defined in Equation~\eqref{eq:essexact} is a natural quantity to consider.

However, the \ess as used in the recent \smc literature is invariably computed using the empirical mean of squared normalized importance weights of the current population. If these importance weights have been accumulated over several iterations, then they coincide with an \ess based on the excursion since the last resampling epoch. A new quantity, termed \emph{\cess} later, proposed in this work instead uses a weighted sample from $\pi_{t-1}$ to approximate exactly the quantity described by Equation~\eqref{eq:essexact}. The different name is used to distinguish this quantity from that usually termed the \ess in the \smc literature.

The approximation leading to the \cess, given samples $\{W_{t-1}^{(i)},X_{t-1}^{(i)}\}_{i=1}^N$ and normalized incremental weights $w_{t}^{(i)} = \pi_t(X_{t-1}^{(i)})/\pi_{t-1}(X_{t-1}^{(i)})$ is simply,
\begin{align*}
  \text{Exact }\ess_t = \frac{N}{1 + \var_{\pi_{t-1}}
    \Square[Big]{\frac{\pi_t(X)}{\pi_{t-1}(X)}}}
  \approx& \frac{N}{\sum_{i=1}^N W_{t-1}^{(i)} 
    \Round[Big]{\frac{\pi_t(X_{t-1}^{(i)})}{\pi_{t-1}(X_{t-1}^{(i)})}}^2}\\
  \approx& \frac{N}{\sum_{i=1}^N W_{t-1}^{(i)}
    \Round[Big]{\frac{w_t^{(i)}}{\sum_{j=1}^N W_{t-1}^{(j)} w_t^{(j)}}}^2}.
\end{align*}
\begin{draftpar}
The first approximation is obtained by replacing the expectation under $\pi_{t-1}$ with its weighted sample average. That is, given the last generation of the particles system $\{W_{t-1}^{(i)}, X_{t-1}^{(i)}\}_{i=1}^N$, which approximates $\pi_{t-1}$, the variance is expressed and approximated as,
\begin{align*}
  \var_{\pi_{t-1}}\Square[Big]{\frac{\pi_t(X)}{\pi_{t-1}(X)}}
  &= \Exp_{\pi_{t-1}}\Square[Big]{\Round[Big]{\frac{\pi_t(X)}{\pi_{t-1}(X)}}^2} -
  \Round[Big]{\Exp_{\pi_{t-1}}\Square[Big]{\frac{\pi_t(X)}{\pi_{t-1}(X)}}}^2 \\
  &\approx \sum_{i=1}^N W_{t-1}^{(i)} \Round[Big]{\frac{\pi_t(X_{t-1}^{(i)})}{\pi_{t-1}(X_{t-1}^{(i)})}}^2
  - \Round[Big]{\int \pi_{t-1}(x)\frac{\pi_t(x)}{\pi_{t-1}(x)}\intd x}^2 \\
  &= \sum_{i=1}^N W_{t-1}^{(i)} \Round[Big]{\frac{\pi_t(X_{t-1}^{(i)})}{\pi_{t-1}(X_{t-1}^{(i)})}}^2 - 1
\end{align*}
The second approximation is simply obtained by rewriting the normalized inremental weights, $\pi_t(X_{t-1})/\pi_{t-1}(X_{t-1})$ as $(\gamma_t(X_{t-1})/\gamma_{t-1}(X_{t-1}))(Z_{t-1}/Z_t)$ where $Z_t$ and $Z_{t-1}$ are the normalizing constants of $\pi_t$ and $\pi_{t-1}$, respectively. Then the ratio of the normalizing constant is replaced by the approximation~\ref{eq:ratio normalized}
\end{draftpar}

Such a procedure leads us to a quantity which we have termed the \emph{conditional} \ess (\cess). By scaling the above approximation,
\begin{equation}
  \cess_t = \Square[bigg]{\sum_{i=1}^N N W_{t-1}^{(i)} \Round[bigg]{
        \frac{w_t^{(i)}}{\sum_{j=1}^N N W_{t-1}^{(j)}w_t^{(j)}}}^2}^{-1}
  = \frac{\Round[Big]{\sum_{i=1}^NW_{t-1}^{(i)}w_t^{(i)}}^2}
  {\sum_{j=1}^N \frac{1}{N} W_{t-1}^{(j)} \Round[Big]{w_t^{(j)}}^2}
\end{equation}
which is equal to the \ess only when resampling is conducted during every iteration. The factor of $1/N$ in the denominator arises from the fact that $\{W_{t-1}^{(i)}\}_{i=1}^N$ is normalized to sum to unity rather than to have expectation unity. The bracketed term coincides with a sample approximation (using the actual samples which are properly weighted to target $\pi_{t-1}$) of the expected sum of the unnormalized weights squared divided by the square of a sample approximation of the expected sum of unnormalized weights when considering sampling from $\pi_{t-1}$ and targeting $\pi_t$ by simple importance sampling.

\begin{figure}[t]
  \UseAltLinespread
  \includegraphics[width=\linewidth]{fig_src/Adaptive_Dist}
  \caption[Variations of the distribution specification parameter for the
  \protect\pet compartmental model using adaptive \protect\smc algorithms]
  {A typical plot of $\alpha_t - \alpha_{t-1}$ against $\alpha_t$ for the
    two-compartments \pet model with the simulated data set using the \smc[2]
    algorithm. The specifications of the adaptive parameter (\ess or \cess)
    are adjusted such that all four samplers use roughly the same number of
    distributions.}
  \label{fig:adaptive_alpha}
\end{figure}


More specifically, in practice, when using \cess to adaptively place distributions, a value $\cess^\star \in (0, 1)$ is chosen, and at each iteration $t$, $\alpha_t$ is chosen such that $\cess_t = \cess^\star$ (with a preset numeric error tolerance). This can be done using an algorithm such as binary search in our setting of \smc[1]--\smc[3], since it is clear that $\cess_t$ is monotonically decreasing when $\alpha_t$ increases. Figure~\ref{fig:adaptive_alpha} shows the variation of $\alpha_t$ when fixed reductions in \ess and \cess are used to specify the sequence of distributions, both when resampling is conducted during every iteration (or equivalently, when the value of $\ess/N$ falls below a threshold of $1.0$, where $N$ is the number of particles) and when resampling is conducted only when the value of $\ess/N$ falls below a threshold of $0.5$. It is found that for the simulated \pet data sets, the \cess-based scheme leads to a reduction in estimator variance around 20\% relative to a manually tuned ($\alpha_k(t/T) = (t/T)^5$) schedule while the \ess-based strategy provides little improvement over the linear case ($\alpha_k(t/T) = t/T$) unless resampling is conducted during every iteration. The effect for the real data sets, which vary considerably from each \draftinline{other} as seen in Figure~\ref{fig:typical real pet}, is more prominent. The variance reduction can \draftinline{be} more than 50\% when using the \cess-based strategy. More performance comparisons for various settings of the samplers can be found in Section~\ref{sub:Nonlinear ordinary differential equations} and~\ref{sub:pet compartmental model}.

\subsubsection[Relations of cess*, number of distributions, and estimator variance]{Relations of $\cess^\star$, number of distributions, and estimator variance}
\label{ssub:Relations of cess*, number of distributions, and estimator variance}

It is intuitively \draftinline{seen} that, when using a \cess-based adaptive scheme, where at each iteration $t$, $\cess_t$ is fixed with a value $\cess^\star$, the more close $\cess^\star/N$ is to $1$, the larger \draftinline{the} number of distributions \draftinline{that} will be placed and better estimates can be obtained. Unfortunately it is not trivial to establish a quantitative \draftinline{relationship} among these three quantities even asymptotically. However, it is straightforward to conduct an empirical study of these relations.

We consider the simulated \pet data set using the \smc[2] sampler with 1,000 particles. Figure~\ref{fig:cess iter mean} plots the average number of distributions (from 100 simulations for each value of $\cess^\star$) against the value of $(1 - \cess^\star/N)$ on log scales. It can be seen that the number of distributions is proportional to $(1 - \cess^\star/N)^{-1}$. This relation holds even for sampler configurations where a very small number of distributions are used. Similarly, Figure~\ref{fig:cess path var} shows the relation between the variance of path sampling estimator and the value of $\cess^\star$. Similar relations can be observed for the standard estimator (Equation~\eqref{eq:smc2-ds}), which is not shown here.

\begin{figure}[t]
  \UseAltLinespread
  \includegraphics[width=\linewidth]{fig_src/CESS_Iter_Mean}
  \caption[Relationship between average number of distributions and
  \protect\cess]
  {Relationship between average number of distributions and $\cess^\star$ for the two-compartments \pet model with the simulated data set using the \smc[2] algorithm (on logarithm scale). The averages are calculated from 100 simulations for each sampler configuration.}
  \label{fig:cess iter mean}
\end{figure}

\begin{figure}[t]
  \UseAltLinespread
  \includegraphics[width=\linewidth]{fig_src/CESS_Path_Var}
  \caption[Relationship between the variance of the path sampling estimator and
  \protect\cess]
  {Relationship between the variance of the path sampling estimator and $\cess^\star$ for the two-compartments \pet model with the simulated data set using the \smc[2] algorithm (on logarithm scale). The variances are calculated from 100 simulations for each sampler configuration.}
  \label{fig:cess path var}
\end{figure}


Though the coefficient of the proportionality varies for different applications or data sets, the relation shown in Figures~\ref{fig:cess iter mean} and~\ref{fig:cess path var} provide a useful \draftinline{guideline} to select the value of $\cess^\star$. A small sample experiment can be conducted before using a value of $\cess^\star$ more close to $1$ to obtain satisfactory estimates or to utilize given computational resources.

\subsection{Adaptive specification of proposals}
\label{sub:Adaptive specification of proposals}

The \smc sampler is remarkably robust to the mixing speed of the \mcmc kernels employed as can be seen in the empirical study later. However, as with any sampling algorithms, faster mixing does not harm performance and in some cases will considerably improve it. In the particular case of Metropolis random walk kernels, the mixing speed relies on adequate proposal scales.

We use a simple approach based on \cite{Jasra:2010eh}. They applied an idea used within adaptive \mcmc methods \cite{Andrieu:2006tw} to \smc samplers by using variance of parameters estimated from its particle system approximation as the proposal scales for the next iteration, suitably scaled with reference to the dimensions of the parameters to be proposed. Although, in practice we found that such an automatic approach does not always lead to optimal acceptance rates, it generally produces satisfactory results and is simple to implement. In difficult problems alternative approaches to adaptation could be employed; one approach demonstrated in \cite{Jasra:2010eh} is to simply employ a pair of acceptance rate thresholds and to alter the proposal scale from the simply estimated value whenever the acceptance rate falls outside those thresholds.

More sophisticated proposal strategies could undoubtedly improve performance further and warrant further investigation. One possible approach is using the Metropolis adjusted Langevin algorithm (\mala; see \cite{Roberts:1996vd}). In summary, \mala derives a Metropolis-Hastings proposal kernel for a target $\pi$ which satisfies suitable differentiability and positivity conditions, from the Langevin diffusion,
\begin{equation*}
  \diff L_t = \frac{1}{2}\nabla\log\pi(L_t)\diff t + \diff B_t
\end{equation*}
where $B_t$ is the standard Brownian motion. Given a state $X^{t-1}$, a new state is proposed by discrete approximation to the above diffusion. That is, for a fixed $h > 0$,
\begin{equation}
  X^t\sim\calN\Round[bigg]{X^{t-1}+\frac{1}{2}\nabla\log\pi(X^{t-1}), hI_d}
\end{equation}
where $I_d$ is the identity matrix and $d$ is the dimension of the state space. The new proposed state is accepted or rejected through the usual Metropolis-Hastings algorithm. Compared to a ``vanilla'' random walk, which often has very robust theoretical properties, \mala is attractive when it is possible and its convergence conditions \cite{Roberts:1996vd} can be met, because only one discrete approximation parameter $h$ needs to be tuned for optimal performance. In addition, results from \cite{Roberts:2001ta} suggested that \mala can be more efficient than a random walk when using optimal scalings. We could also use the particle approximation at time $t - 1$ to estimate the covariance matrix of $\pi_t$ and thus tune the scale $h$ on-line. As these algorithms are known to be somewhat sensitive to scaling, and we seek approaches robust enough to employ with little user intervention, we have not investigated this strategy further in this work.

\begin{figure}[t]
  \UseAltLinespread
  \includegraphics[width=\linewidth]{fig_src/Adaptive_Proposal}
  \caption[Acceptance rates of adaptive \protect\smc algorithms]
  {Average random walk acceptance rates for the two-compartments \pet model with data sets in Figure~\ref{fig:typical real pet} using adaptive proposal scales.}
  \label{fig:pet adaptive proposal}
\end{figure}

\begin{figure}[t]
  \UseAltLinespread
  \includegraphics[width=\linewidth]{fig_src/Fixed_Proposal}
  \caption[Acceptance rates of non-adaptive \protect\smc algorithms]
  {Average random walk acceptance rates for the two-compartments \pet model
    with data sets in Figure~\ref{fig:typical real pet} using fixed proposal
    scales}
  \label{fig:pet fixed proposal}
\end{figure}


\draftinline{An} adaptive specification of proposals is most useful when manual tuning is difficult or even impossible. We consider the three real \pet data sets shown in Figure~\ref{fig:typical real pet}. When using adaptive proposal scales, the average acceptance rates of the four random walk blocks (for parameters $\phi_{1:r}$, $\theta_{1:r}$, $\tau$ and $\nu$, respectively), \draftinline{are} shown in Figure~\ref{fig:pet adaptive proposal}.The results are not close to the optimal value $0.234$, which is commonly used in practice, but they are more than acceptable. In contrast, in Figure~\ref{fig:pet fixed proposal} we show the average acceptance rates when using a scheme of proposal scales which is obtained from another typical real \pet data set. The scheme is tuned such that for that particular data set and for all parameters, the acceptance rates fall within the range $[0.2, 0.4]$ for $\alpha_t \in [0, 1]$. It can be seen that, not only do the results vary considerably due to the variety of the data sets, which is expected, \draftinline{but also} for one of the data sets, the parameter $\nu$ fails to move at all. Considering that there are about a quarter of a million such data sets to be estimated in a single \pet scan, adaptively specifying the proposal scales is the only feasible approach. Note that, this problem is not unique to the \pet compartmental model at all. In many realistic applications, there are a large number of data sets which vary considerably from each other.

\subsection{An automatic and adaptive algorithm}
\label{sub:An automatic and adaptive algorithm}

With the above refinements, we are ready to implement the \smc[2] algorithm with minimal tuning and application specific effort while providing robust and accurate estimates of the model evidence $p(\data|\calM_k)$. First the geometric annealing scheme that connects the prior $\pi(\theta_k|\calM_k)$ and the posterior $\pi(\theta_k|\data,\calM_k)$, provides a smooth path for a wide range of problems.

Second, the actual annealing schedule under this scheme can be determined through the adaptive schedule as described above. The advantage of the adaptive schedule will be shown empirically later.

Third, we can adaptively specify the Metropolis random walk (or \mala) proposal scales through the estimation of their scaling parameters as the sampler iterates. In contrast to the \mcmc setting, where such adaptive algorithms will usually require a burn-in period, which will not be used for further estimation, in \smc, the variance and covariance estimates come at almost no cost, as all the samples will later be used for marginal likelihood estimation. Additionally, adaptation within \smc does not require separate theoretical justification in the sense that in principle the Strong Law of Large Numbers (\slln) holds directly -- something which can significantly complicate the development of effective, theoretically justified schemes in the \mcmc setting. Nonetheless, some asymptotic \draftinline{results} can be found in \cite{Beskos:2013vx}, including a version of the Central Limit Theorem for adaptive resampling among other results. Alternatively, we can also specify the proposal scales in a deterministic, but sensible way. Since \smc algorithms are relatively robust to the change of scales, such deterministic scales will not require the same degree of tuning as is required to obtain good performance in \mcmc algorithms.

Though we described the algorithm in the setting of \smc[2], it can also be applied to other \smc strategies. \smc[1] is less straightforward as the between model moves still require efforts to design and implement. In \smc[3], the specification of the sequences between posterior distributions are less generic compared to the geometric annealing scheme in \smc[2]. However, the adaptive schedule and automatic tuning of \mcmc proposal scales, both can be applied in these two algorithms in principal. We outline the strategy in Algorithm~\ref{alg:adaptive}.

\begin{algorithm}
\begin{algorithmic}
  \tophrule
  \STATE \emph{Accuracy control}
  \STATE\STATESKIP Set constant $\cess^\star\in(0,1)$, using a small pilot
  simulation if necessary.

  \STATE \emph{Initialization:} Set $t\leftarrow0$.
  \STATE\STATESKIP Perform the \emph{Initialization} step
    as in Algorithm~\ref{alg:smc1} or~\ref{alg:smc2}

  \STATE \emph{Iteration:} Set $t\leftarrow t + 1$

  \STATE\STATESKIP \emph{Step size selection}
  \STATE\STATESKIP\STATESKIP Use a binary search %(or other search algorithms)
  to find $\alpha^\star$ such that $\cess_{\alpha^\star} = \cess^\star$
  \STATE\STATESKIP\STATESKIP Set $\alpha_t \leftarrow\alpha^\star$ if
    $\alpha^\star \le 1$, otherwise set $\alpha_t\leftarrow1$

  \STATE\STATESKIP \emph{Proposal scale calibration}
  \STATE\STATESKIP\STATESKIP
  Computing the importance sampling estimates of first two moments of
  parameters.
  \STATE\STATESKIP\STATESKIP
  Set the proposal scale of the Markov proposal $K_t$ with the estimated
  parameter variances.

  \STATE\STATESKIP Perform the \emph{Iteration} step as in
  Algorithm~\ref{alg:smc1} or~\ref{alg:smc2} with the found $\alpha_t$
  and proposal scales.

  \STATE \emph{Repeat} the \emph{Iteration} step %up to $t = T$ (or $T_k$ in the
    %case of Algorithm~\ref{alg:smc2})}
    \emph{until $\alpha_t = 1$} then set $T=t$.
  \bottomhrule
\end{algorithmic}
\caption{An Automatic, Generic Algorithm for Bayesian Model Comparison}
\label{alg:adaptive}
\end{algorithm}


As laid out above, the algorithms require minimal tuning. \draftinline{Theirs} robustness, accuracy and efficiency will be shown in Section~\ref{sec:Performance comparison} through comprehensive empirical studies. Here we show some interesting yet intuitively expected results. We consider the simulated \pet data sets, using an \smc[2] sampler with 1,000 particles. It is already shown that using either the adaptive specification of distribution placement or the \mcmc proposal scales can give better results. The combination of the two can lead to even better results. The use of the \cess-based schedule scheme will not only place more distributions where the target distributions changes more rapidly, but also it will place more distributions where the \mcmc algorithm mixes \draftinline{more slowly}. To illustrate the idea, we consider four configurations of the sampler. The proposal scales are specified either adaptively or using a fixed scheme which is manually tuned. The placement of the distributions is either \cess-based or using a fixed schedule $\alpha_t = \alpha_k(t/T_k) = (t/T_k)^5$.

\begin{table}[t]
  \UseAltLinespread
  \caption{The standard Bayes factor estimates for a
    simulated \protect\pet data set using the \protect\smc[2] algorithm. }
  \label{tab:pet four sampler same dist}
  \begin{tabularx}{\linewidth}{lXX}
    \toprule
    & \multicolumn{2}{c}{Proposal scales} \\
    \cmidrule(lr){2-3}
     & Fixed & Adaptive \\
    \midrule
    Fixed annealing ($\alpha_k(t/_k) = (t/T_k)^5$) & $1.6\pm0.27$ & $1.6\pm0.22$ \\
    \cess-based adaptive annealing & $1.6\pm0.19$ & $1.6\pm0.15$ \\
    \bottomrule
    \multicolumn{3}{r}{%
      \begin{minipage}{\linewidth-2em}\vskip1ex\sffamily
        All four samplers are configured such that about 200 distributions are
        used. The estimates $\log B_{2,1}\pm\sd$) are obtained from 100
        simulations for each sampler.
      \end{minipage}}
  \end{tabularx}
\end{table}


The results are shown in Table~\ref{tab:pet four sampler same dist}. More comprehensive results can be found in Section~\ref{sub:pet compartmental model}. However, from this particular example, we can see that the combination of the two adaptive schemes provides superior performance at very little \draftinline{computational} cost. Table~\ref{tab:pet four sampler same iter} shows actually equivalent results. Instead of fixing the number of distributions, we configured the samplers such that they all give roughly the same precision of the estimates. It is rather obvious to see that the use of adaptive methods actually give a considerable reduction of computational cost for a given desired precision of the estimates.

\begin{table}[t]
  \caption[Cost of Bayes factor estimates]
  {Number of distributions used for the simulated \pet data set using
    \smc[2] algorithm. All four samplers are configured such that the Bayes
    factor estimates ($\log B_{1,2}$) have a standard deviation of
    approximately $0.27$.}
  \label{tab:pet four sampler same iter}
  \begin{tabularx}{\linewidth}{lXX}
    \toprule
    & \multicolumn{2}{c}{Proposal scales} \\
    \cmidrule(lr){2-3}
    Schedule & Fixed & Adaptive \\
    \midrule
    Quadratic & $200$ & $182$ \\
    \cess     & $175$ & $157$ \\
    \bottomrule
  \end{tabularx}
\end{table}


Although further enhancements and refinements are clearly possible, we focus in the remainder of this chapter on this simple, generic algorithm which can be easily implemented in any application and has proved sufficiently powerful to provide good estimation in the examples we have encountered thus far.

\section{Theoretical considerations}
\label{sec:Theoretical considerations}

The convergence results for the standard estimator can be found in \cite{DelMoral:2006hc} and references therein. In this work, given our advocation of \smc[2]-\ps, we extend the results for the path sampling estimator from \smc samplers. Here we present Proposition~\ref{prop:path_clt}, which is specific to path sampling estimator using the simplest Trapezoidal approach to numerical integration. It follows as a simple corollary to a more general result given in Appendix~\appref{sec:Proof of proposition 5.1} which could be used to characterize more general numerical integration schemes.

\begin{proposition}\label{prop:path_clt}
  Under the same regularity conditions as are required for the central limit theorem given in \cite{DelMoral:2006hc} to hold, given an \smc sampler that iterates over a sequence of distributions $\{\pi_t = \gamma_{\alpha_t}/Z_{\alpha_t}\}_{t=0}^T$ and applies multinomial resampling at each iteration, the path sampling estimator, $\hat\Xi_{T}^{N}$, as defined in Equation~\eqref{eq:path_est} obeys a central limit theorem in the following sense: Let $\xi_t(\cdot) = \frac{\diff\log \gamma_{\alpha}(\cdot)}{\diff\alpha}\Bigm|_{\alpha = \alpha_t}$, $\beta_{0} = \alpha_0 / 2$, $\beta_{T} = \alpha_T / 2$ and for $t = 1,\ldots,T-1$, $\beta_t = (\alpha_{t + 1} - \alpha_{t-1})/2$, then, provided $\xi_t$ is bounded,
  \begin{equation}
    \lim_{N\to\infty}\sqrt{N}(\hat\Xi_{T}^{N} - \Xi_T)
    \xrightarrow{D}\calN(0, V_T(\xi_{0:T}))
  \end{equation}
  where $\xrightarrow{D}$ denotes convergence in distribution and $V_t$, $0\le t \le T$ is defined by the following recursion,
  \begin{align}
    V_0(\xi_0) =&  \beta_0^2
    \int \pi_0(x_0) (\xi_0(x_0)
    - \pi_0(\xi_0))^2 dx_0 \\
    V_t(\xi_{0:t}) =& V_{t-1}\Round[bigg]{\xi_{0:t-2}, \xi_{t-1}
    + \frac{\beta_t}{\beta_{t-1}}
    \frac{\pi_{t}(\cdot)}{\pi_{t-1}(\cdot)}
    \int K_t(\cdot,x_t) (\xi_t(x_t)-\pi_t(\xi_t)) \intd x_t
    } \\\notag
    &+ \beta_t^2 \int\frac{\pi_t(x_{t-1})^2}{\pi_{t-1}(x_{t-1})}
    K_t(x_{t-1},x_t)(\xi_t(x_t) - \pi_t(\xi_t))^2) \intd x_{t-1} \intd x_t.
  \end{align}
\end{proposition}

Application of similar arguments to those used in \cite{DelMoral:2006hc} to the historical process associated with the \smc sampler would lead to essentially the same result, but we find this approach more transparent. We note that much recent analysis of \smc algorithms has focused on relaxing the relatively strong assumptions used in the results upon which this result is based -- looking at more general resampling schemes \cite{DelMoral:2012jq} and relaxing compactness assumptions \cite{Whiteley:2013vx} for example. However, we feel that this simple result is sufficient to show the relationship between the path sampling and simple estimators and that in this instance the \draftinline{relative} simplicity of the resulting expression justifies these stronger assumptions.
 
\section{Performance comparison}
\label{sec:Performance comparison}

In this section, we will use three examples to illustrate the algorithms. The Gaussian mixture model is discussed first, with implementations for all three \smc algorithms with comparison to \rjmcmc and population \mcmc. It will be shown that all five algorithms agree on the results while the performance in terms of Monte Carlo variance varies considerably. It will also be demonstrated how the adaptive refinements of the algorithms behaves in practice. We will reach the conclusion that considering ease of implementation, performance and generality, the \smc[2] algorithm is most promising among all three strategies.

Then two more realistic examples, a nonlinear \ode model and the \pet compartmental model are used to study the performance and robustness of algorithm \smc[2] compared to \ais and population \mcmc. Various configurations of the algorithms are considered including both sequential and parallelized implementations.

The \cpp implementations, which make use of the \vsmc library of \cite{vsmcjss}, of all examples can be found at \url{https://github.com/zhouyan/vSMCExample} and the library is also introduced in Chapter~\ref{cha:vSMC: A C++ Library for Parallel SMC}.

\subsection{Gaussian mixture model}
\label{sub:Gaussian mixture model}

Since \cite{Richardson:1997ea}, the Gaussian mixture model (\gmm) has provided a canonical example of a model-order-determination problem. We use the model formulation of \cite{DelMoral:2006hc} to illustrate the efficiency and robustness of the methods proposed in this chapter compared to other approaches. The model is as follows; data $\data = (y_1,\dots,y_n)$ are independently and identically distributed as
\begin{equation*}
  y_i|\theta_r \sim \sum_{j=1}^r \omega_j\calN(\mu_j,\lambda_j^{-1})
\end{equation*}
where $\calN(\mu_j,\lambda_j^{-1})$ denotes the Normal distribution with mean $\mu_j$ and precision $\lambda_j$; $\theta_r = (\mu_{1:r},\lambda_{1:r},\omega_{1:r})$ and $r$ is the number of components in each model. The parameter space is thus $\Real^r \times (\Real^{+})^r\times \Delta_r$ where $\Delta_r = \{\omega_{1:r}:0\le\omega_j\le1; \sum_{j=1}^r\omega_j=1\}$ is the standard $r$-simplex. The priors which are the same for each component are taken to be $\mu_j\sim\calN(\xi,\kappa^{-1})$, $\lambda_j\sim\calGa(\nu,\chi)$ and $\omega_{1:r}\sim\calD(\rho)$ where $\calD(\rho)$ is the symmetric Dirichlet distribution with parameter $\rho$ and $\calGa(\nu,\chi)$ is the Gamma distribution with shape $\nu$ and scale $\chi$. The prior parameters are set in the same manner as in \cite{Richardson:1997ea}. Specifically, let $y_{\text{\textsc{min}}}$ and $y_{\text{\textsc{max}}}$ be the minimum and maximum of data $\data$, the prior parameters are set such that
\begin{equation*}
  \xi = (y_{\text{\textsc{max}}} + y_{\text{\textsc{min}}}) / 2, \quad
  \kappa = (y_{\text{\textsc{max}}} - y_{\text{\textsc{min}}})^{-2}, \quad
  \nu = 2, \quad \chi = 50\kappa, \quad \rho = 1
\end{equation*}
The data is simulated from a four components model with $\mu_{1:4} = (-3, 0,3, 6)$, and $\lambda_j =2$, $\omega_j = 0.25$, $j = 1,\dots,4$.

We consider several algorithms. First the \rjmcmc algorithm as in \cite{Richardson:1997ea}, and second an implementation of the \smc[1] algorithm. Next \ais, population \mcmc and \smc[2] are used for within-model simulations. The last is an implementation of the \smc[3] algorithm. In all the algorithms, the local move which does not change the dimension of the model is constructed as a composition of Metropolis-Hastings random walk kernels:
\begin{enumerate}
  \item Update $\mu_{1:r}$ using a multivariate Normal random walk proposal.
  \item Update $\lambda_{1:r}$ using a multivariate Normal random walk on logarithmic scale, i.e., on $\log\lambda_{j}$, $j = 1, \dots, r$.
  \item Update $\omega_{1:r}$ using a multivariate Normal random walk on logit scale, i.e., on $\omega_{j}/\omega_r$, $j = 1,\dots,r-1$.
\end{enumerate}
The \rjmcmc, \smc[1] and \smc[3] algorithms use two additional pairs of reversible jump moves. The first is a combine and split move; the second is a birth and death move. Both are constructed in the same manner as in \cite{Richardson:1997ea}. Also in these implementations, an adjacency condition was imposed on the means $\mu_{1:r}$, such that $\mu_1 < \mu_2 < \dots < \mu_r$. No such restriction was used for other algorithms.

In the \smc[1], \smc[2], \ais and population \mcmc implementations, the distributions are chosen with a geometric schedule, i.e., as in Equation~\eqref{eq:geometry_1} for \smc[1] and Equation~\eqref{eq:geometry_2} for the other three. This annealing scheme has been used in \cite{DelMoral:2006hc,Jasra:2007in} and many other works. The geometric scheme can also be seen in \cite{Calderhead:2009bd} for population \mcmc tempering. A schedule $\alpha(t/T) = (t/T)^p$, with $p = 2$ was used. The rationale behind this particular schedule can be seen in \cite{Calderhead:2009bd} and other values of $p$ were also tried while $p\approx2$ performs best in this particular example. The adaptive schedule was also implemented for \smc[2] and \ais algorithms.

The proposal scales for each block of the random walks are specified dynamically according to values of $\alpha(t/T)$ for the \smc[2] and \ais algorithms and also manually tuned for other algorithms such that the acceptance rates fall in $[0.2, 0.5]$. Later for the \smc[2] and \ais algorithms, we also consider adaptive schedule of the distribution specification parameter $\alpha(t/T)$ and the proposal scales of the random walks.

For \smc[2], \smc[3] and \ais we consider both the direct estimator and the path sampling estimator. For population \mcmc we consider the path sampling estimator.

\subsubsection{Results}
\label{sec:gmm_res}

The \smc[1] implementation uses 10,000 particles and 500 distributions. The \rjmcmc implementation uses five million iterations in addition to one million iterations of burn-in period for adaptation. The resulting estimates of model probabilities are shown in Table~\ref{tab:gmm-prob}.

The \smc[2], \smc[3] and \ais implementations use 1,000 particles and 500 iterations. Population \mcmc implementation uses 50 chains and 10,000 iterations in addition to 10,000 iterations used for adaptation (the burn-in period) -- these implementations have approximately equal computational costs. For all algorithms where resampling is needed, the stratified resampling algorithm is applied (see Section~\ref{sub:Sequential importance sampling and resampling}).

From the results obtained under the \smc[1] and \rjmcmc algorithms it is clear that, in this particular example, simulations for models with fewer than ten components are adequate to characterize the model space. Therefore, under this configuration, the cost is roughly the same in terms of computational resources as that of the \smc[1] and \rjmcmc algorithms. From the results of \rjmcmc and \smc[1], we consider the four and five components models (i.e., the true model and the most competitive one amongst the others). The estimates are shown in Table~\ref{tab:gmm-pair} which, like all of the other tables in this section, summarizes the Monte Carlo variability of 100 replicate runs of each algorithm.

\begin{table}
    \caption[Gaussian mixture model posterior model probability estimates]
    {Gaussian mixture model posterior model probability estimates obtained via
      \smc[1] and \rjmcmc}
    \label{tab:gmm-prob}
    \begin{tabu}{X[2l]X[2l]X[1c]X[1c]X[1c]X[1c]X[1l]X[1c]X[1c]}
      \toprule
      & & \multicolumn{7}{c}{Number of components} \\
      \cmidrule(lr){3-9}
      Quantity & Algorithm & $\le2$ & $3$ & $4$ & $5$ & $6$ & $7$ & $\ge8$ \\
      \midrule
      $\Prob(\calM_k|\data)$ & \smc[1]
      & $0$ & $0.0022$ & $0.89$ & $0.10$ & $0.0064$ & $0.0014$ & $0$ \\
                         & \rjmcmc
      & $0$ & $0.0013$ & $0.89$ & $0.10$ & $0.0062$ & $0.0025$ & $0$ \\
      $\log B_{4,k}$     & \smc[1]
      & $\infty$ & $6.00$ & $0$ & $2.15$ & $4.93$ & $6.45$ & $\infty$ \\
                         & \rjmcmc
      & $\infty$ & $6.53$ & $0$ & $2.15$ & $4.97$ & $5.87$ & $\infty$ \\
      \bottomrule
    \end{tabu}
\end{table}

\begin{table}
  \caption{Gaussian mixture model Bayes factor estimates obtained via \smc[2],
    \smc[3], \ais and \pmcmc}
  \label{tab:gmm-pair}
  \begin{tabu}{X[1.2l]X[1c]X[1c]X[1c]X[1c]X[1c]X[1c]X[1c]}
    \toprule
    & \multicolumn{7}{c}{Algorithm} \\
    \cmidrule(lr){2-8}
    Quantity
    & \smc[2]-\ds & \smc[2]-\ps & \smc[3]-\ds & \smc[3]-\ps
    & \ais-\ds & \ais-\ps & \pmcmc \\
    \midrule
    $\log B_{4,5}$
    & $2.15$ & $2.15$ & $2.16$ & $2.21$ & $2.16$ & $2.17$ & $2.63$ \\
    \sd
    & $0.25$ & $\Best0.22$ & $0.61$ & $0.62$ & $1.12$ & $1.10$ & $0.41$ \\
    \bottomrule
  \end{tabu}
\end{table}


From Tables~\ref{tab:gmm-prob} and~\ref{tab:gmm-pair}, it can be seen that the standard estimators (\rjmcmc, \smc[1], \smc[2]-\ds, \smc[3]-\ds and \ais-\ds) agree with each other. Among the path sampling estimators, \smc[2]-\ps and \ais-\ps have little bias. \smc[3]-\ps shows a little more bias. The population \mcmc algorithm has a considerably larger bias as the number of distributions is relatively small (as noted previously, a larger number will negatively affect the mixing speed). In terms of Monte Carlo variance, in Table~\ref{tab:gmm-pair}, \smc[2] clearly has an advantage compared to its no-resampling variant, \ais. The differences of Monte Carlo standard deviation between \smc[2], \smc[3] and population \mcmc, although they do not affect model selection in this particular example, are considerable.

\paragraph{Effects of resampling}

It is clear from these results that resampling (when required) can substantially improve the estimation of normalizing constants within an \smc framework. This does not contradict the statement in \cite{DelMoral:2006hc} which suggests that resampling may not much help when the normalizing constant is the object of interest, the theoretical argument which supports this relies upon the assumption that the Markov kernel used to mutate the particles mixes extremely rapidly and the result is obtained under the assumption that resampling is performed after every iteration. When the Markov kernel is not so rapidly mixing, the additional stability provided by the resampling operation can out-weight the attendant increase in Monte Carlo variance and that is what we observed here (and in the case of the other examples considered below; results not shown.)

\begin{figure}
  \linespread{1.1}\selectfont
  \includegraphics[width=\linewidth]{fig_src/GMM_Resample}
  \caption[Variance of standard standard estimator and path sampling using
  adaptive resampling]
  {Monte Carlo variance of standard estimator and path sampling estimator
    using different threshold of $\ess/N$ for Gaussian mixture model using the
    \smc[2] algorithm and the stratified resampling. The variances are plotted
    on log scale.}
  \label{fig:gmm resample}
\end{figure}


On the other hand, in this work we advocate resampling adaptively instead of resampling every iteration. The proposed adaptive schedule also has significant advantage in this situation. It is of interest to see if adaptive resampling has performance improvement when compared to resampling every iteration. We consider using the \smc[2] algorithm with 1,000 particles, 100 distributions scheduled as $\alpha(t/T) = (t/T)^2$, and various threshold of $\ess/N$ under which resampling is performed. The Monte Carlo variance of both the standard estimator and the path sampling estimator is shown in Figure~\ref{fig:gmm resample}. It can be shown that neither performing resampling every iteration or never performing resampling (i.e., the \ais algorithm), gives optimal results. Though it may be difficult to determine an optimal value, the commonly used value $0.5$ seems to be suitable for this particular example, in the sense that the variance is only slightly higher than that of sampler performing resampling at every iteration while the computational cost is reduced. The reduction in computational cost might be negligible for some applications. However, as noted before, when parallel computing is considered, frequent resampling can be a bottleneck of performance in reality.

\paragraph{Effects of adaptive schedules}

To assess the evolution of distributions with an adaptive schedule, we consider the relation between $\alpha_t - \alpha_{t-1}$ and $\alpha_t$. As stated before, one of the motivations of using \cess for adaptive placement of distribution is to ensure that $\alpha_t$ evolves the same path regardless \draftinline{of} the resampling strategies. Earlier in Figure~\ref{fig:adaptive_alpha} (page~\ref{fig:adaptive_alpha}) we showed the evolution of $\alpha_t$ when fixing \ess or \cess and resampling every iteration or only when $\ess/N < N$, where $N$ is the number of particles. As shown in the plot, when fixing \cess, the evolution of the distributions is not affected by the resampling strategy. In contrast, fixing \ess yields a sequence of distributions which depends strongly upon the resampling strategy.

In terms of the actual performance when using the \cess adaptive strategy in the \smc[2] and \ais algorithms, a reduction of standard deviation of 20\% was observed comparing to $\alpha(t/T) = (t/T)^2$, the one shown in Table~\ref{tab:gmm-pair}. When applied to the \smc[3] algorithm, 50\% reduction was observed. If the \ess adaptive strategy is used instead, similar standard deviation reduction is observed when resampling is performed every iteration but no significant effect was observed when resampling was only performed when $\ess/N < 0.5$ (i.e., using \ess rather than \cess entirely eliminated the benefit.)

\paragraph{Effects of adaptive proposal scales}

When using the \smc[2] algorithm, if the adaptive strategy of \cite{Andrieu:2006tw} is applied, where the important sampling estimates of the variance of parameters are included in the adaptation, the acceptance rates fall within $[0.2, 0.5]$ dynamically without manual tuning as for the results in Table~\ref{tab:gmm-pair}. It should be noted that in this particular example, it is the variance of $\log\lambda_{1:r}$ being estimated as the corresponding random walk block operates on the log scale. The same principle applies to the weight parameters $\omega_{1:r}$, which have random walks on the logit scale. Approximately 20\% reduction in standard deviation was observed.

\subsection{Nonlinear ordinary differential equations}
\label{sub:Nonlinear ordinary differential equations}

In this section, this will now be further explored in a more complex model, a nonlinear ordinary differential equations (\ode) system. This model, which was studied in \cite{Calderhead:2009bd}, is known as the Goodwin model. The \ode system, for an $r$-component model, is,
\begin{align*}
  \frac{\diff X_1(t)}{\diff t} &= \frac{a_1}{1 + a_2 X_r(t)^{\rho}}
  - \alpha X_1(t)  & \\
  \frac{\diff X_i(t)}{\diff t} &= k_{i-1}X_{i-1}(t) - \alpha X_i(t)
  & i = 2,\dots,r \\
  X_i(0) &= 0 & i = 1,\dots,r
\end{align*}
The parameters $\{\alpha,a_1,a_2,k_{1:r-1}\}$ have common prior distribution $\calGa(0.1, 0.1)$. Under this setting, $X_{1:r}(t)$ can exhibit either unstable oscillation or a constant steady state. The data are simulated for $r=3$ and~$5$ at equally spaced time points from $0$ to $60$, with time step $0.5$. The last 80 data points of $(X_1(t), X_2(t))$ are used for inference. Normally-distributed noise with standard deviation $\sigma=0.2$ is added to the simulated data. Following \cite{Calderhead:2009bd}, the variance of the additive measurement error is assumed to be known. Therefore, the posterior distribution has $r+2$ parameters for an $r$-component model.

As shown in \cite{Calderhead:2009bd}, when $\rho > 8$, due to the possible instability of the \ode system, the posterior can have a considerable number of local modes. In this example, we set $\rho = 10$. Also, as the solution to the \ode system is somewhat unstable, slightly different data can result in very different posterior distributions.

The example from the previous section suggests that \smc[2] performs well relative to the other \smc possibilities. Given the wide range of settings in which it can be easily deployed, we will now concentrate further on this method. It also suggests that in the simple case of Gaussian mixtures, a complete adaptive strategy for both distribution specification and proposal scales works well.

\subsubsection{Results}

We compare results from the \smc[2] and population \mcmc algorithms. For the \smc implementation, 1,000 particles and 500 iterations were used, with the distributions specified specified by Equation~\eqref{eq:geometry_2}, with $\alpha_k(t/T_k) = (t/T)^5$, or via the completely adaptive specification. For population \mcmc algorithm, 50,000 iterations are performed for the adaptation of the proposal scales (the burn-in period), and another 10,000 iterations are used for inference. The same tempering as was used for \smc is used here. Note that, in a sequential implementation of population \mcmc, with each iteration updating one local chain and attempting a global exchange, the computational cost of after burn-in iterations is roughly the same as the entire \smc algorithm. In addition, changing $T$ within the range of the number of cores available does not substantially change the computational cost of a generic parallel implementation of population \mcmc algorithm. We compare results from $T = 10,30,$ and~$100$. For the \smc algorithms, stratified resampling is applied.

The results for data generated from the simple model ($r = 3$) and complex model ($r = 5$), again summarizing variability amongst 100 runs of each algorithm, are shown in Table~\ref{tab:node-s} and~\ref{tab:node-c}, respectively.
\begin{draftpar}
The model selection results, displayed as the frequencies of each model being selected by the Bayes factor using estimators from each algorithm, are shown in Table~\ref{tab:node-s-mo} and~\ref{tab:node-c-mo}, respectively. It can be seem that all \smc algorithms and the population \mcmc algorithm with sufficient large numbers of distributions can give accurate results of model selection. With a smaller number of distributions, the population \mcmc algorithm may occasionally choose a wrong model. However, the accuracy of the estimators differ considerably among these algorithms.
\end{draftpar}

\begingroup
\afterpage{\begin{table}
  \caption{Nonlinear \ode model marginal likelihood and Bayes factor estimates
    with data generated from simple (three components) model.}
  \label{tab:node-s}
  \begin{tabu}{X[1l]X[2c]X[2c]X[2c]X[4r]X[4r]X[6r]}
    \toprule
    &&&& \multicolumn{2}{c}{Marginal likelihood ($\log p(\data|\calM_k)\pm\sd$)} & Bayes factor ($\log B_{3,5}\pm\sd$) \\
    \cmidrule(lr){5-6}
    $T$   & Proposal & Annealing & Algorithm   & $m = 3$                & $m = 5$                & \\ \midrule
    $10 $ & Manual         & Prior (5) & \pmcmc      & $-109.7\pm3.2$         & $-120.3\pm2.5$         & $10.6\pm3.8$ \\
    $30 $ &                &           &             & $\SubBest-105.0\pm1.2$ & $\SubBest-116.1\pm2.2$ & $\SubBest11.2\pm2.5$ \\
    $100$ &                &           &             & $-134.7\pm7.9$         & $-144.1\pm6.2$         & $9.4\pm11.2$ \\ \midrule
    $500$ & Manual         & Prior (5) & \smc[2]-\ds & $-104.6\pm2.0$         & $-112.7\pm1.8$         & $8.1\pm2.8$ \\
          &                &           & \smc[2]-\ps & $-104.5\pm1.8$         & $-112.7\pm1.5$         & $8.2\pm2.5$ \\
    $500$ & Manual         & Adaptive  & \smc[2]-\ds & $-104.5\pm1.1$         & $-112.7\pm1.1$         & $8.1\pm1.6$ \\
          &                &           & \smc[2]-\ps & $-104.6\pm1.0$         & $-112.8\pm1.0$         & $8.2\pm1.5$ \\
    $500$ & Adaptive       & Adaptive  & \smc[2]-\ds & $-104.5\pm0.5$         & $-112.7\pm0.4$         & $8.1\pm0.8$ \\
          &                &           & \smc[2]-\ps & $\Best-104.6\pm0.4$    & $\Best-112.8\pm0.3$    & $\Best8.1\pm0.6$ \\
    \bottomrule
    \multicolumn{7}{r}{%
      \begin{minipage}{\linewidth-2em}\vskip1ex\sffamily
        $T$: The number of distributions. That is, the number parallel \mcmc
        chains for \pmcmc algorithm and the number of iterations for the \smc
        algorithm. For the \pmcmc algorithm, at each iteration, a single \mcmc
        chain is chosen randomly to be updated. There is a 50,000 iterations
        burin period and another 10,000 iterations are used for inference. For
        the \smc algorithm, 1,000 particles are used.

        Proposal: The proposal scales of the \mcmc kernels. The ``Manual''
        proposal is manually tuned such that the accept rates fall between
        $[0.2, 0.6]$. The ``Adaptive'' proposal is dynamically set using
        moments estimates from the last iteration of the particle system.

        Annealing: The annealing scheme of the distributions, $\alpha(t/T_k)$
        in Equation~\ref{eq:geometry_2}. The ``Prior (5)'' scheme is
        concentrated around the prior distribuiton, with $\alpha(t/T_k) =
        (t/T_k)^5$. The ``Adaptive'' scheme set $\alpha$ dynamically such that
        for each iteration, \cess is a constant. The constant value of \cess
        is chosen such that on average, 500 iterations was produced.

        \Best: The estimate with the smallest variance among all algorithms
        settings.

        \SubBest: The estimate with the smallest variance for a single
        algorithm (\smc[2] or \pmcmc) among different settings.
      \end{minipage}}
  \end{tabu}
\end{table}
\begin{table}
  \def\B{\color{MBlue}\it}
  \def\R{\color{MRed}\bf}
  \begingroup\small
    \begin{tabularx}{\linewidth}{lCClCCC}
      \toprule
      &&&& \multicolumn{2}{c}{Marginal likelihood} & \\
      &&&& \multicolumn{2}{c}{($\log p(\paramk|\data)\pm\text{\sd}$)} & \\
      \cmidrule(lr){5-6}
      $T$ & Proposal Scales & Annealing Scheme & Algorithm & $m = 3$ & $m = 5$ & Bayes factor $\log B_{5,3}$ \\ \midrule
      $10 $ & Manual    & Prior (5) & \pmcmc      & $-1651.0\pm27.9$   & $-85.1\pm36.6$   & $1565.9\pm42.1$ \\
      $30 $ &           &           &             & $\B-1639.7\pm7.4$  & $\B-78.9\pm11.2$ & $\B1560.8\pm12.8$ \\
      $100$ &           &           &             & $-1624.6\pm15.7$   & $-75.7\pm24.8$   & $1548.9\pm25.6$ \\ \midrule
      $500$ & Manual    & Prior (5) & \smc2-\ds & $-1640.7\pm10.8$   & $-78.5\pm9.8$    & $1562.2\pm10.1$ \\
            &           &           & \smc2-\ps & $-1640.8\pm 8.4$   & $-79.2\pm7.9$    & $1561.6\pm 8.5$ \\
      $500$ & Manual    & Adaptive  & \smc2-\ds & $-1639.7\pm 6.9$   & $-78.6\pm4.8$    & $1561.1\pm7.1$ \\
            &           &           & \smc2-\ps & $-1640.1\pm 5.4$   & $-78.8\pm3.7$    & $1561.3\pm6.8$ \\
      $500$ & Adaptive  & Adaptive  & \smc2-\ds & $-1639.8\pm 2.2$   & $-79.4\pm1.7$    & $1560.4\pm3.1$ \\
            &           &           & \smc2-\ps & $\R-1640.2\pm 1.9$ & $\R-78.5\pm1.5$  & $\R1561.7\pm2.3$ \\
      \bottomrule
    \end{tabularx}
  \endgroup
  \caption{Results for non-linear \ode models with data generated from complex
    model. Number {\B italic}: Minimum variance for the same algorithm. {\R Bold}: Minimum variance for all samplers.}
  \label{tab:node-c-all}
\end{table}
\clearpage}
\endgroup

As shown in both cases, the number of distributions can affect the performance of population \mcmc algorithms considerably. When using 10 distributions, large bias from numerical integration for path sampling estimator was observed, as expected. With 30 distributions, the performance is comparable to the \smc[2] sampler, though some bias is still observable. With 100 distributions, there is a much larger variance because, with more chains, the information travels more slowly from rapidly mixing chains to slowly mixing ones and consequently the mixing of the overall system is inhibited.

The \smc algorithm provides results comparable to the best of three population \mcmc implementations in essentially all settings, including one in which both the annealing schedule and proposal scaling were fully automatic. In fact, the completely adaptive strategy was the most successful.

It can be seen that increasing the number of distributions not only reduces the bias of numerical integration for path sampling estimator, but also reduces the variance considerably. On the other hand increasing the number of particles can only reduce the variance of the estimates, in accordance with the Central Limit Theorem (see \cite{DelMoral:2006hc} for the standard estimator and extensions for path sampling estimator, Proposition~\ref{prop:path_clt}). The bias arises mostly from the numerical integration scheme and cannot be reduced by using more particles. Though there exists the trade-off between the number of particles and the number distributions, increasing either of them can always benefit the accuracy of the estimators. We will study this trade-off more carefully with the next more realistic example.

\subsection[\protect\pet compartmental model]{\protect\pet compartmental model}
\label{sub:pet compartmental model}

It is now interesting to compare the proposed algorithm with other state-of-art algorithms using the more realistic \pet compartmental model example.

As mentioned before, real neuroscience data sets involve a very large number (about a quarter of a million per brain) of time series, which are typically somewhat heterogeneous (also see Figure~\ref{fig:typical real pet}). Robustness is therefore especially important. An application-specific \mcmc algorithm was developed for this problem in \cite{Zhou2013} and its results are shown in the Section~\ref{ssub:Generalized harmonic mean estimator}. A significant amount of tuning of the algorithms was required to obtain good results. The results shown in Figure~\ref{fig:petplot} and discussed later are very close to those of \cite{Zhou2013} but, as is shown later, they were obtained with almost no manual tuning effort and at similar computational cost.

For the \smc and population \mcmc algorithms, the requirement of robustness means that the algorithm must be able to calibrate itself automatically to different data (and thus different posterior surfaces). A sequence of distributions which performs well for one time series may not perform even adequately for another series. Specification of proposal scales that produces fast-mixing kernels for one data series may lead to slow mixing for another (as we already see in Section~\ref{sub:Adaptive specification of proposals}.) In the following experiment, we will use the simulated data sets, and choose schedules that perform both well and poorly for this particular time series. The objective is to see if the algorithm can recover from a relatively poorly specified schedule and obtain reasonably accurate results.

\begin{figure}[t]
  \UseAltLinespread
  \includegraphics[width=\linewidth]{fig_src/PETPlot-smc2-ps-bw}
  \caption
  [Volume of distribution estimates of real \protect\pet
   compartmental model data]
  {Volume of distribution estimates of real \protect\pet
   compartmental model data.}
  \label{fig:petplot}
\end{figure}


\subsubsection{Results}

In this example we focus on the comparison between \smc[2] and population \mcmc. We also consider parallelized implementations of algorithms. In this case, due to its relatively small number of chains, population \mcmc can be parallelized completely (and often cannot fully utilize the hardware capability if a na\"\i ve approach to parallelization is taken; while we appreciate that more sophisticated parallelization strategies are possible, these depend intrinsically upon the model under investigation and the hardware employed and given our focus on automatic and general algorithms, we don't consider such strategies here.) Population \mcmc algorithm under this setting is implemented such that each chain is updated at each iteration. Further, for the \smc algorithms, we consider two cases. In the first we can parallelize the algorithm completely (in the sense that each core has a single particle associated with it.) In this setting we use a relatively small number of particles and a larger number of time steps. In the second, we need a few passes to process a large number of particles at each time step, and accordingly we use fewer time steps to maintain the same total computation time. These two settings allow us to investigate the trade-off between the number of particles and time steps. In both implementations, we consider three schedules, $\alpha(t/T) = t/T$ (linear), $\alpha(t/T) = (t/T)^5$ (prior), and $\alpha(t/T) = 1 - (1 - t/T)^5$ (posterior). The linear schedule can be seen as an off-the-shelf choice while the prior schedule is expected to perform generally well for many applications. The posterior schedule is expected to perform poorly compared to the others. It places more distributions close to the posterior than the prior, where with the introducing of the likelihood function, the intermediate distributions are likely to change more dramatically with respect to the change of $\alpha$. This is included there to test if the algorithm is capable of producing sensible results when the sequence of distributions is specified poorly, which is quite a possible scenario in realistic applications. In addition, the adaptive schedule based upon \cess is also implemented for the \smc[2] algorithm. For the \smc algorithms, stratified resampling is applied.

Result from 100 replicate runs of the two algorithms under various regimes can be found in Table~\ref{tab:pet-py} and~\ref{tab:pet-bf} for the marginal likelihood and Bayes factor estimates, respectively. The \smc algorithms consistently outperforms population \mcmc algorithms in the parallel settings. The Monte Carlo standard deviation of \smc algorithms is typically of the order of one fifth of the corresponding estimates from population \mcmc in most scenarios. In some settings with the smaller number of samples, the two algorithms can be comparable. Also at the lowest computational costs, the samplers with more time steps and fewer particles outperform those with the converse configuration by a fairly large margin in terms of estimator variance. It shows that with limited resources, ensuring the similarity of consecutive distributions, and thus good mixing, can be more beneficial than a larger number of particles. However, when the computational budget is increased, the difference becomes negligible.

In the particular case of the posterior schedule, it is not \draftinline{surprising} that all path sampling estimates suffer considerably large biases. The standard estimator using the \smc[2] algorithm is able to give relatively accurate results (though with a larger variance compared to other schedules). In summary, the \smc algorithm is much more robust than population \mcmc algorithm in this example.

\afterpage{\begin{table}
  \def\B{\color{MBlue}\it}
  \def\R{\color{MRed}\bf}
  \begingroup\small
    \begin{tabularx}{\linewidth}{lllCCCC}
      \toprule
      \multicolumn{3}{l}{Proposal scales}  & \multicolumn{3}{c}{Manual} & Adaptive \\
      \cmidrule(lr){1-3}\cmidrule(lr){4-6}\cmidrule(lr){7-7}
      \multicolumn{3}{l}{Annealing scheme} & Prior (5) & Posterior (5) & \multicolumn{2}{c}{Adaptive} \\
      \cmidrule(lr){1-3}\cmidrule(lr){4-4}\cmidrule(lr){5-5}\cmidrule(lr){6-7}
      %
      $T$    & $N$   & Algorithm   & \multicolumn{4}{c}{Marginal likelihood estimates ($\log p(\theta_k|\data)\pm\text{\sd}$)} \\ \midrule
      $500$  & $30 $ & \pmcmc      & $ -39.1\pm  0.56$ & $-926.8\pm376.99$ && \\
      $500$  & $192$ & \smc2-\ds & $\B-39.2\pm0.25$ & $\B-39.7\pm1.06$ & $\B-39.2\pm0.18$ & $\R-39.1\pm0.12$ \\
             &       & \smc2-\ps & $\B-39.2\pm0.25$ & $-91.3\pm21.69$  & $\B-39.2\pm0.18$ & $-39.1\pm0.13$   \\
      $100 $ & $960$ & \smc2-\ds & $-39.3\pm0.36$   & $-40.6\pm1.41$   & $-39.2\pm0.31$   & $-39.2\pm0.19$   \\
             &       & \smc2-\ps & $-39.3\pm0.35$   & $302.1\pm46.29$  & $-39.3\pm0.31$   & $-39.2\pm0.18$   \\ \midrule
      $1000$ & $30 $ & \pmcmc      & $ -39.3\pm  0.46$ & $-884.1\pm307.88$ && \\
      $1000$ & $192$ & \smc2-\ds & $\B-39.2\pm0.19$ & $\B-39.4\pm0.68$ & $\B-39.2\pm0.17$ & $\R-39.1\pm0.10$ \\
             &       & \smc2-\ps & $\B-39.2\pm0.19$ & $-66.0\pm13.26$  & $\B-39.2\pm0.17$ & $\R-39.1\pm0.10$  \\
      $200 $ & $960$ & \smc2-\ds & $-39.2\pm0.22$   & $-39.8\pm1.21$   & $-39.2\pm0.18$   & $-39.1\pm0.11$   \\
             &       & \smc2-\ps & $-39.2\pm0.22$   & $175.5\pm26.84$  & $-39.2\pm0.18$   & $-39.2\pm0.11$   \\ \midrule
      $2000$ & $30 $ & \pmcmc      & $ -39.3\pm  0.28$ & $-928.7\pm204.93$ && \\
      $2000$ & $192$ & \smc2-\ds & $-39.2\pm0.14$   & $\B-39.3\pm0.41$ & $-39.1\pm0.12$   & $-39.1\pm0.07$  \\
             &       & \smc2-\ps & $-39.2\pm0.14$   & $-51.2\pm4.30$   & $-39.2\pm0.12$   & $-39.1\pm0.07$  \\
      $400 $ & $960$ & \smc2-\ds & $\B-39.2\pm0.13$ & $-39.4\pm0.73$   & $\B-39.2\pm0.11$ & $-39.2\pm0.07$   \\
             &       & \smc2-\ps & $\B-39.2\pm0.13$ & $106.0\pm14.36$  & $\B-39.2\pm0.11$ & $\R-39.2\pm0.06$ \\ \midrule
      $5000$ & $30$  & \pmcmc      & $ -39.3\pm  0.21$ & $-917.6\pm129.54$ && \\
      $5000$ & $192$ & \smc2-\ds & $-39.2\pm0.09$   & $\B-39.2\pm0.20$ & $-39.2\pm0.08$   & $-39.1\pm0.04$   \\
             &       & \smc2-\ps & $-39.2\pm0.09$   & $-43.8\pm2.13$   & $-39.2\pm0.08$   & $-39.1\pm0.04$   \\
      $1000$ & $960$ & \smc2-\ds & $\B-39.2\pm0.08$ & $-39.2\pm0.31$   & $\B-39.2\pm0.07$ & $\R-39.2\pm0.03$ \\
             &       & \smc2-\ps & $\B-39.2\pm0.08$ & $-65.7\pm5.54$   & $\B-39.2\pm0.07$ & $\R-39.2\pm0.03$ \\
      \bottomrule
    \end{tabularx}
  \endgroup
  \caption[\protect\pet compartmental model marginal likelihood estimates]
  {Marginal likelihood estimates of two components \pet model. $T$:
    Number of distributions in \smc and number of iterations used for
    inference in \pmcmc. $N$: Number of particles in \smc and number chains in
    \pmcmc. The \pmcmc and \smc with $N = 192$ are completely $N$-way
    parallelized.  \smc with $N = 960$ are $N/5$-way parallelized. {\B
      Italic}: Minimum variance for the same computational cost and the same proposal
    scales and annealing schemes.  {\R Bold}: Minimum variance for the same computaitonal
    cost and all proposal scales and annealing schemes.}
  \label{tab:pet-py}
\end{table}

% & & $ -39.1\pm  0.56$ & $-926.8\pm376.99$ &&& \\
% & $\B-39.2\pm0.25$ & $\B-39.7\pm1.06$ & $\B-39.2\pm0.18$ & $\R-39.1\pm0.12$ & $\B-39.2\pm0.13$ \\
% & $\B-39.2\pm0.25$ & $-91.3\pm21.69$  & $\B-39.2\pm0.18$ & $-39.1\pm0.13$   & $\B-39.2\pm0.13$ \\
% & $-39.3\pm0.36$   & $-40.6\pm1.41$   & $-39.2\pm0.31$   & $-39.2\pm0.19$   & $-39.2\pm0.18$ \\
% & $-39.3\pm0.35$   & $302.1\pm46.29$  & $-39.3\pm0.31$   & $-39.2\pm0.18$   & $-39.2\pm0.19$ \\ \midrule
% & & $ -39.3\pm  0.46$ & $-884.1\pm307.88$ &&& \\
% & $\B-39.2\pm0.19$ & $\B-39.4\pm0.68$ & $\B-39.2\pm0.17$ & $\R-39.1\pm0.10$ & $-39.2\pm0.11$ \\
% & $\B-39.2\pm0.19$ & $-66.0\pm13.26$  & $\B-39.2\pm0.17$ & $\R-39.1\pm0.10$ & $\R-39.2\pm0.10$ \\
% & $-39.2\pm0.22$   & $-39.8\pm1.21$   & $-39.2\pm0.18$   & $-39.1\pm0.11$   & $-39.2\pm0.11$ \\
% & $-39.2\pm0.22$   & $175.5\pm26.84$  & $-39.2\pm0.18$   & $-39.2\pm0.11$   & $-39.2\pm0.12$ \\ \midrule
% & & $ -39.3\pm  0.28$ & $-928.7\pm204.93$ &&& \\
% & $-39.2\pm0.14$   & $\B-39.3\pm0.41$ & $-39.1\pm0.12$   & $-39.1\pm0.07$   & $\B-39.2\pm0.07$ \\
% & $-39.2\pm0.14$   & $-51.2\pm4.30$   & $-39.2\pm0.12$   & $-39.1\pm0.07$   & $\B-39.2\pm0.07$ \\
% & $\B-39.2\pm0.13$ & $-39.4\pm0.73$   & $\B-39.2\pm0.11$ & $-39.2\pm0.07$   & $-39.2\pm0.08$ \\
% & $\B-39.2\pm0.13$ & $106.0\pm14.36$  & $\B-39.2\pm0.11$ & $\R-39.2\pm0.06$ & $-39.2\pm0.08$ \\ \midrule
% & & $ -39.3\pm  0.21$ & $-917.6\pm129.54$ &&& \\
% & $-39.2\pm0.09$   & $\B-39.2\pm0.20$ & $-39.2\pm0.08$   & $-39.1\pm0.04$   & $-39.1\pm0.05$ \\
% & $-39.2\pm0.09$   & $-43.8\pm2.13$   & $-39.2\pm0.08$   & $-39.1\pm0.04$   & $-39.1\pm0.04$ \\
% & $\B-39.2\pm0.08$ & $-39.2\pm0.31$   & $\B-39.2\pm0.07$ & $\R-39.2\pm0.03$ & $-39.2\pm0.06$ \\
% & $\B-39.2\pm0.08$ & $-65.7\pm5.54$   & $\B-39.2\pm0.07$ & $\R-39.2\pm0.03$ & $\B-39.2\pm0.04$ \\
\clearpage}
\afterpage{\newgeometry{hmargin={1in,1in},vmargin={1in,2in},bindingoffset=\mclassbinding}
\begin{table}
  \UseAltLinespread
  \caption[\protect\pet compartmental model the Bayes factor estimates]
  {The Bayes factor estimates of two-compartments \pet model.} 
  \label{tab:pet-bf}
  \begin{tabu}{X[0.4l]X[0.4l]X[0.7c]X[0.8r]X[1r]X[0.8r]X[0.8r]}
      \toprule
      \multicolumn{3}{l}{Proposal scales}  & \multicolumn{3}{c}{Manual} & Adaptive \\
      \cmidrule(lr){1-3}\cmidrule(lr){4-6}\cmidrule(lr){7-7}
      \multicolumn{3}{l}{Annealing scheme} & Prior (5) & Posterior (5) & \multicolumn{2}{c}{Adaptive} \\
      \cmidrule(lr){1-3}\cmidrule(lr){4-4}\cmidrule(lr){5-5}\cmidrule(lr){6-7}
      %
      $T$    & $N$   & Algorithm   & \multicolumn{4}{c}{Bayes factor estimates ($\log B_{2,1}\pm\sd$)} \\ \midrule
      $500$  & $30 $ & \pmcmc      & $1.7\pm0.62$ & $-70.9\pm525.79$ && \\
      $500$  & $192$ & \smc2-\ds & $\SubBest{1.6\pm0.27}$ & $\SubBest{1.3\pm1.13}$  & $\SubBest{1.6\pm0.20}$ & $\Best{1.6\pm0.15}$  \\
             &       & \smc2-\ps & $\SubBest{1.6\pm0.27}$ & $-3.9\pm30.02$  & $\SubBest{1.6\pm0.20}$ & $\Best{1.6\pm0.15}$  \\
      $100 $ & $960$ & \smc2-\ds & $1.6\pm0.37$   & $0.5\pm1.55$    & $1.6\pm0.34$   & $1.6\pm0.21$    \\
             &       & \smc2-\ps & $1.6\pm0.37$   & $-13.1\pm66.30$ & $1.6\pm0.33$   & $1.6\pm0.21$    \\ \midrule
      $1000$ & $30 $ & \pmcmc      & $1.6\pm0.49$ & $-67.3\pm400.21$ && \\
      $1000$ & $192$ & \smc2-\ds & $\SubBest{1.6\pm0.21}$ & $\SubBest{1.5\pm0.79}$  & $1.6\pm0.20$   & $1.6\pm0.13$   \\
             &       & \smc2-\ps & $\SubBest{1.6\pm0.21}$ & $-0.6\pm15.47$  & $1.6\pm0.20$   & $1.6\pm0.13$   \\
      $200 $ & $960$ & \smc2-\ds & $1.6\pm0.25$   & $1.1\pm1.25$    & $1.6\pm0.19$   & $1.6\pm0.12$   \\
             &       & \smc2-\ps & $1.6\pm0.24$   & $-11.7\pm34.68$ & $\SubBest{1.6\pm0.18}$ & $\Best{1.6\pm0.11}$ \\ \midrule
      $2000$ & $30 $ & \pmcmc      & $1.6\pm0.31$ & $-95.5\pm264.74$ && \\
      $2000$ & $192$ & \smc2-\ds & $\SubBest{1.6\pm0.14}$ & $\SubBest{1.6\pm0.44}$  & $1.6\pm0.13$   & $1.6\pm0.09$    \\
             &       & \smc2-\ps & $\SubBest{1.6\pm0.14}$ & $1.6\pm6.06$    & $1.6\pm0.13$   & $1.7\pm0.09$    \\
      $400 $ & $960$ & \smc2-\ds & $1.6\pm0.16$   & $1.5\pm0.74$    & $\SubBest{1.6\pm0.12}$ & $\Best{1.6\pm0.08}$  \\
             &       & \smc2-\ps & $1.6\pm0.16$   & $-4.2\pm17.15$  & $\SubBest{1.6\pm0.12}$ & $\Best{1.6\pm0.08}$  \\ \midrule
      $5000$ & $30$  & \pmcmc      & $1.6\pm0.24$ & $-60.3\pm198.10$ && \\
      $5000$ & $192$ & \smc2-\ds & $1.6\pm0.10$   & $\SubBest{1.6\pm0.23}$  & $1.6\pm0.09$   & $1.6\pm0.05$    \\
             &       & \smc2-\ps & $1.6\pm0.10$   & $1.3\pm2.98$    & $1.6\pm0.09$   & $1.6\pm0.05$    \\
      $1000$ & $960$ & \smc2-\ds & $\SubBest{1.6\pm0.09}$ & $1.6\pm0.33$    & $\SubBest{1.6\pm0.08}$ & $\Best{1.6\pm0.04}$  \\
             &       & \smc2-\ps & $\SubBest{1.6\pm0.09}$ & $-0.2\pm6.63$   & $\SubBest{1.6\pm0.08}$ & $\Best{1.6\pm0.04}$  \\
      \bottomrule
    \multicolumn{7}{r}{%
      \begin{minipage}{\linewidth-2em}\vskip1ex\sffamily
        $T$: The number of distributions.

        $N$: The number of particles.

        Proposal: The proposal scales of the \mcmc kernels.

        Annealing: The annealing scheme of the distributions,
        $\alpha_k(t/T_k)$ in Equation~\ref{eq:geometry_2}.

        \Best: The estimate with the smallest variance among all algorithms
        settings.

        \SubBest: The estimate with the smallest variance for a single
        algorithm (\smc[2] or \pmcmc) among different settings.
      \end{minipage}}
    \end{tabu}
\end{table}
\restoregeometry

% & $1.7\pm0.62$ & $-70.9\pm525.79$ &&& \\
% & $\B1.6\pm0.27$ & $\B1.3\pm1.13$  & $\B1.6\pm0.20$ & $\R1.6\pm0.15$ & $\B1.6\pm0.16$ \\
% & $\B1.6\pm0.27$ & $-3.9\pm30.02$  & $\B1.6\pm0.20$ & $\R1.6\pm0.15$ & $\B1.6\pm0.16$ \\
% & $1.6\pm0.37$   & $0.5\pm1.55$    & $1.6\pm0.34$   & $1.6\pm0.21$   & $1.6\pm0.17$ \\
% & $1.6\pm0.37$   & $-13.1\pm66.30$ & $1.6\pm0.33$   & $1.6\pm0.21$   & $1.6\pm0.17$ \\ \midrule
% & $1.6\pm0.49$ & $-67.3\pm400.21$ &&& \\
% & $\B1.6\pm0.21$ & $\B1.5\pm0.79$  & $1.6\pm0.20$   & $1.6\pm0.13$   & $1.6\pm0.14$ \\
% & $\B1.6\pm0.21$ & $-0.6\pm15.47$  & $1.6\pm0.20$   & $1.6\pm0.13$   & $1.6\pm0.13$ \\
% & $1.6\pm0.25$   & $1.1\pm1.25$    & $1.6\pm0.19$   & $1.6\pm0.12$   & $1.6\pm0.13$ \\
% & $1.6\pm0.24$   & $-11.7\pm34.68$ & $\B1.6\pm0.18$ & $\R1.6\pm0.11$ & $\B1.6\pm0.12$ \\ \midrule
% & $1.6\pm0.31$ & $-95.5\pm264.74$ &&& \\
% & $\B1.6\pm0.14$ & $\B1.6\pm0.44$  & $1.6\pm0.13$   & $1.6\pm0.09$   & $1.6\pm0.11$ \\
% & $\B1.6\pm0.14$ & $1.6\pm6.06$    & $1.6\pm0.13$   & $1.7\pm0.09$   & $1.6\pm0.10$ \\
% & $1.6\pm0.16$   & $1.5\pm0.74$    & $\B1.6\pm0.12$ & $\R1.6\pm0.08$ & $\B1.6\pm0.09$ \\
% & $1.6\pm0.16$   & $-4.2\pm17.15$  & $\B1.6\pm0.12$ & $\R1.6\pm0.08$ & $\B1.6\pm0.09$ \\ \midrule
% & $1.6\pm0.24$ & $-60.3\pm198.10$ &&& \\
% & $1.6\pm0.10$   & $\B1.6\pm0.23$  & $1.6\pm0.09$   & $1.6\pm0.05$   & $1.6\pm0.06$ \\
% & $1.6\pm0.10$   & $1.3\pm2.98$    & $1.6\pm0.09$   & $1.6\pm0.05$   & $1.6\pm0.06$ \\
% & $\B1.6\pm0.09$ & $1.6\pm0.33$    & $\B1.6\pm0.08$ & $\R1.6\pm0.04$ & $1.6\pm0.06$ \\
% & $\B1.6\pm0.09$ & $-0.2\pm6.63$   & $\B1.6\pm0.08$ & $\R1.6\pm0.04$ & $\B1.6\pm0.05$ \\
\clearpage}

\paragraph{Effects of adaptive schedule}

A set of samplers with adaptive schedules are also used. Due to the nature of the schedule, it cannot be controlled to have exactly the same number of time steps as non-adaptive procedures. However, the \cess was controlled such that the average number of time steps are comparable with the fixed schedules and in most cases slightly less than the fixed numbers.

It is found that, with little computational overhead, adaptive schedules do provide the best results (or very nearly so) and do so without user intervention. The reduction of Monte Carlo standard deviation varies among different configurations. For moderate or larger number of distributions, a reduction about 50\% was observed. In addition, it should be noted that, in this example, the bias of the path sampling estimates are much more sensitive to the schedules than the previous Gaussian mixture model example. A vanilla linear schedule does not provide a low bias estimator at all even when the number of distributions is increased to a considerably larger number. The prior schedule though provides a nearly unbiased estimator, there is no clear theoretical evidence showing that this should work for other situations. Even it has more general usage, as suggested in \cite{Calderhead:2009bd}, the power still has to be chosen (in the previous \gmm example, $p = 2$ was the best choice while in this \pet example $p = 5$ is more suitable). In contrast, The adaptive schedule, without any manual calibration, can provide a nearly unbiased estimator, even when path-sampling is employed, in addition to potential variance reduction.

\paragraph{Bias reduction for path sampling estimator}

As seen in Table~\ref{tab:pet-py} and~\ref{tab:pet-bf}, a bad choice of schedule $\alpha(t/T)$ can results in considerable bias for the basic path sampling estimator, here for \smc[2]-\ps but the problem is independent of the mechanism by which the samples are obtained. Increasing the number of iterations can reduce this bias but at the cost of additional computation time. As outlined in Section~\ref{sub:Improved univariate numerical integration}, in the case of the \smc algorithms discussed here, it is possible to reduce the bias without increasing computational cost significantly. To demonstrate the bias reduction effect, we constructed \smc sampler for the above \pet example with only 1,000 particles and about 20 iterations specified using the \cess based adaptive strategy. The path sampling estimator was approximated using Equation~\eqref{eq:path_est} as well as other higher order numerical integration or by integrating over a grid that contains $\{\alpha_t\}$ at which the samples was generated. The results are shown in Table~\ref{tab:pet-bias}

\begin{table}
  \begingroup\small
  \begin{tabularx}{\linewidth}{lXXXX}
    \toprule
    & \multicolumn{4}{c}{Number of grid points (compared to sampled iterations)} \\
    \cmidrule(lr){2-5}
    Integration rule & $\times1$ & $\times2$ & $\times4$ & $\times8$ \\
    \midrule
    Trapezoid
    & $-52.2\pm5.01$ & $-45.5\pm1.93$ & $-42.1\pm1.21$ & $-40.5\pm1.06$ \\
    Simpson
    & $-43.2\pm1.39$ & $-41.0\pm1.10$ & $-40.0\pm1.04$ & $-39.4\pm1.04$ \\
    Simpson $3/8$
    & $-42.1\pm1.21$ & $-40.5\pm1.06$ & $-39.7\pm1.04$ & $-39.3\pm1.04$ \\
    Boole
    & $-40.9\pm1.09$ & $-39.9\pm1.04$ & $-39.4\pm1.04$ & $-39.2\pm1.05$ \\
    \bottomrule
  \end{tabularx}\endgroup
  \caption{Path sampling estimator of marginal likelihood of two components
    \pet model. The estimator was approximated using samples from \smc[2]
    algorithm with 1,000 particles and 20 iterations, with different numerical
    integration strategies. Large sample result (see Table~\ref{tab:pet-py})
    provide an estimate of $-39.2$.}
  \label{tab:pet-bias}
\end{table}


\paragraph{Trade-off between the number of particles and distributions}

\begin{figure}[t]
  \UseAltLinespread
  \includegraphics[width=\linewidth]{fig_src/Particle_Iter_Var}
  \caption[Variance of path sampling estimator and total number of samples
  using \protect\smc algorithm]
  {Variance of path sampling estimator and total number of samples using the \smc[2] algorithm. Multiple samplers are used to evluate the trade-off between the number of particles $N$ and the number of distributions $T$. They are configured such that the total number of samples $NT$ is a constant. In this figure, the variance of the path sampling estimator from 100 simulations of each sampler is plotted against the total number of samples $NT$ on the logarithm scale. Different configurations of $T$ are indicated with different sizes of the dots, larger dots representing larger $T$ (and thus smaller $N$). It can be seen that for a particular value of $NT$, changing $T$ may change the variance considerably and larger $T$ is preferred for most values of $NT$.}
  \label{fig:particle iter num}
\end{figure}


It can be seen through the nonlinear \ode and the \pet examples that, there is a trade-off between the number of particles and distributions. Increasing either of them can improve the accuracy of estimates. We consider a range of number of particles (from 100 to more than 10,000) and a range of number of distributions using the prior schedule ($\alpha(t/T) = (t/T)^5$; from as small as 10 to more than 1,000.) When applied to the simulated data sets, the variance of the path sampling estimates is plotted against the total number of samples (the product of these two quantities) in Figure~\ref{fig:particle iter num}. It can be seen that, for the same total number of samples, samplers with larger number of distributions outperform those with larger number of particles by a considerable large margin. However, as the number of samples increase, the difference becomes smaller and smaller. This suggests that it will be better to first allocate a fixed number of particles, which is at least large enough to approximate the initial distribution well, according to considerations such as computation hardwares. And then use the number of distributions as a performance parameter to tune the sampler for desired accuracy. When using the adaptive algorithms proposed in this work, it is equivalent to tune the value of $\cess^\star$. As shown in Section~\ref{sub:Adaptive specification of distributions}, the relation between $\cess^\star$ and the variance of estimators provides a predictable way to configure the samplers, at least for the particular examples considered in this chapter.

\paragraph{Fast mixing \mcmc kernels and number of distributions}

As mentioned in Section~\ref{sub:Optimal and suboptimal backward kernels}, the suboptimal backward kernel and its associated incremental weights used in the above examples could perform poorly if the adjacent distributions are not close, even when the transition kernel mixes well. However, both fast mixing kernels and more intermediate distributions (and thus a smoother sequence), can improve the performance of the sampler. To improve the mixing speed of the kernel, one can apply multiple passes of \mcmc moves at each iteration. Here we compare two samplers \draftinline{using} the simulated data sets, both with 1,000 particles. One sampler \draftinline{uses} 10 \mcmc moves at each iteration. The other only use one \mcmc move but ten times the number of distributions. The annealing scheme, for simplicity, is chosen to be $\alpha(t/T) = (t/T)^5$. The results of the path sampling variance is shown in Figure~\ref{fig:fast mcmc iter}. It can be seen that, given the same total number of samples simulated, using more distributions almost always outperforms using more \mcmc moves. However, with \draftinline{a sufficiently} large number of samples, the difference is minimal.

\begin{figure}[t]
  \UseAltLinespread
  \includegraphics[width=\linewidth]{fig_src/MCMC_Iter_Var}
  \caption[Variance of path sampling estimator and total number of samples
  using \protect\smc algorithm]
  {Variance of the path sampling estimates and total number of samples using the \smc[2] algorithm (on logarithm scale). The variances are calculated from 100 simulations for each sampler configuration. All samplers use 1,000 particles but with different number of distributions $T$ and passes of \mcmc moves at each iteration $K$. And $TK$ is the number of total samples generated. Samplers with the same value of $TK$ have roughly the same computational cost.}
  \label{fig:fast mcmc iter}
\end{figure}


\paragraph{Real data results}

Finally, the methodology of \smc[2]-\ps was applied to measured positron emission tomography data using the same compartmental setup as in the simulations. The data that lead to the $V_D$ estimation as shown in Figure~\ref{fig:petplot} comes from a study into opioid receptor density in Epilepsy, with the data being described in detail in \cite{Jiang:2009kf} (also see section~\ref{sec:Simulated and real pet data}). It is expected that there will be considerable spatial smoothness to the estimates of the volume of distribution, as this is in line with the biology of the system being somewhat regional. Some regions will have much higher receptor density while others will be much lower, yielding higher and lower values of the volume of distribution, respectively. While we did not impose any spatial smoothness but rather estimated the parameters independently for each time series at each spatial location, as can be seen, smooth spatial estimates of the volume of distribution consistent with neurological understanding were found using the approach. This method is computationally feasible for the entire brain on a voxel-by-voxel basis, due to the ease of parallelization of the \smc algorithm. In the analysis performed here 1,000 particles were used, along with an adaptive schedule using a constant $\cess^\star/N = 0.999$, resulting in about 180 to 200 intermediate distributions. The model selection results are very close to those obtained by a previous study of the same data \cite{Zhou2013}, although the present approach requires much less implementation effort and has roughly the same computational cost.

\subsection{Summary}

These three illustrative applications have essentially shown three aspects of using \smc as a generic tool for Bayesian model selection. Firstly, as seen in the Gaussian mixture model example, all the different variants of \smc proposed, including both direct and path sampling versions, produce results which are competitive with other model selection methods such as \rjmcmc and population \mcmc. In addition, in this somewhat simple example, \smc[2] performs well, and leads to low variance estimates with no appreciable bias. The effect of adaptation was studied more carefully in the nonlinear \ode example, and it was shown that using both adaptive selection of distributions as well as adaptive proposal variances leads to very competitive algorithms, even against those with significant manual tuning. This suggests that an automatic process of model selection using \smc[2] is possible. In the final example, considering the easy parallelization of algorithms such as \smc[2] suggests that great gains in variance estimation can be made using settings such as \gpu computing for application where computational resources are of particular importance (such as in image analysis as in the PET example). It is also clear that the negligible cost of the bias reduction techniques described means that one should always consider using these to reduce the bias inherent in path sampling estimation.

\section{Discussions}
\label{sec:Bayesian SMC discussion}

It has been shown that \smc is an effective Monte Carlo method for Bayesian inference for the purpose of model comparison. Three approaches have been outlined and investigated in several illustrative applications including the challenging scenarios of nonlinear \ode models and \pet compartmental systems. The proposed strategy is always competitive and often substantially outperforms the state of the art in this area.

It has been demonstrated that it is possible to use the \smc algorithms to estimate the model probabilities directly (\smc[1]), or through individual model evidence (\smc[2]), or pair-wise relative evidence (\smc[3]). In addition, both \smc[2] and \smc[3] algorithms can be coupled with the path sampling estimator.

Among the three approaches, \smc[1] is applicable to very general settings. It can provide a robust alternative to \rjmcmc when inference on a countable collection of models is required (and could be readily combined with the approach of \cite{Jasra:2008bb} at the expense of a little additional implementation effort). However, like all Monte Carlo methods involving between model moves, it can be difficult to design efficient algorithms in practice. The \smc[3] algorithm is conceptually appealing. However, the existence of a suitable sequence of distributions between two posterior distributions may not be obvious.

The \smc[2] algorithm, which only involves within-model simulation, is most straightforward to implement in many interesting problems. It has been shown to be exceedingly robust in many settings. As it depends largely upon a collection of within-model \mcmc moves, any existing \mcmc algorithms can be reused in the \smc[2] framework. However, much less tuning is required because the algorithm is fundamentally less sensitive to the mixing of the Markov kernel and it is possible to implement effective adaptive strategies at little computational cost. With adaptive placement of the intermediate distributions and specification of the \mcmc kernel proposals, \draftinline{they} provide a robust and essentially automatic model comparison method.

Compared to population \mcmc, \smc[2] has greater flexibility in the specification of distributions. Unlike population \mcmc, where the number and placement of distributions can affect the mixing speed and hence performance considerably, increasing the number of distributions will always benefit an \smc sampler given the same number of particles. When coupled with a path sampling estimator, this leads to less bias and variance. Compared to its no-resampling variant (i.e., \aic), it has been shown that \smc samplers with resampling can reduce the variance of normalizing constant estimates considerably.

Even after three decades of intensive development, no Monte Carlo method can solve the Bayesian model comparison problem completely automatically without any manual tuning. However, \smc algorithms and the adaptive strategies demonstrated in this chapter show that even for realistic, interesting problems, these samplers can provide good results with very minimal tuning and few design difficulties. For many applications, they could already be used as near automatic, robust solutions. For more challenging problems, the robustness of the algorithms can serve as solid foundation for specific algorithm designs.
