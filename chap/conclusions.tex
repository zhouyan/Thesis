\chapter{Conclusions}
\label{cha:Conclusions}

This thesis is concerned with the use \smc for the purpose of Bayesian model
comparison. A generic framework was developed. Practical implementation tools
for generic \smc algorithms are also presented.

It is found that, compared to \mcmc approach to Bayesian model comparison, the
\smc approach is often more robust. Both the standard and the path sampling
estimators are more stable when compared to the generalized harmonic mean
estimator used in the \mcmc setting. Though there are also estimators used for
some specific \mcmc algorithms such as the Gibbs sampling, that is more stable
than the generalized harmonic mean estimator, they often require knowledge of
the models that are often absent in reality. In contrast, the \smc algorithm
and its estimators are more generic and therefore applicable in more areas of
interest.

There are considerable performance gain of the adaptive \smc algorithms, in
particular the \cess-based adaptive specification of distributions. Unlike
adaptive \mcmc algorithms, such strategies \dn{have} little computational cost. In
addition, it is also generic in the sense that it does not depend on a
specific form of the intermediate distributions. It is recommended that such
strategies should be employed for complex models, where the characteristics of
the posterior distribution is hardly known and it is difficult to manually
specify a smooth path from the prior towards the posterior.

In summary, the \smc framework for Bayesian model comparison presented in this
thesis has the potential to solve many realistic problems which are previously
difficult with the \mcmc algorithms or requires significantly less efforts to
optimize the algorithm.

\section{Contributions}
\label{sec:Contributions}

In Chapter~\ref{cha:Sequential Monte Carlo for Bayesian Computation}, the
algorithms presented can accurately approximate the Bayes factor with little
or no human tuning for many applications of interest. Some theoretical results
of the use of path sampling estimator within this framework was developed as
extension to the results of the standard estimator. A novel adaptive algorithm
for specification of the placement of distributions in the \smc[2] algorithm
is introduced. It provides better (and more sensible) results than using the
\ess as a criterion of how to introduce a new distribution. This method can be
extended to other algorithms such as \smc[3] straightforwardly. The
performance of the presented algorithms is studied in detail through various
empirical experiments.

In Chapter~\ref{cha:vSMC: A C++ Library for Parallel SMC}, a \cpp library is
introduced. It provides a tool for the implementation of generic \smc
algorithms. Compared to some established softwares, it has either a higher
level of flexibility in the sense of enabling the implementation of general
algorithms instead of particular models; or higher performance through the use
of parallel computing.

\section{Future directions}
\label{sec:Further directions}

Though many convincing results have been shown in Chapter~\ref{cha:Sequential
  Monte Carlo for Bayesian Computation}, theoretical development is also
needed. In particular, the better performance of \cess-based adaptive scheme
has only been shown empirically. It might be of interest to establish if it is
better in some sense when compared to some commonly used deterministic
scheme and under what conditions.

Some algorithms have potential applications in scenarios different than those
shown in this thesis. For example, the \smc[3] algorithm may be used for
Bayesian model expansion. It may be of interest to see if the approach
presented in this thesis has any significant advantage when compared to
alternatives in terms of accuracy and computational efficiency.

In this thesis, most examples use a geometric annealing scheme to specify the
intermediate distributions. However, the adaptive strategies proposed are not
limited to this setting. There are other forms of the sequence of
distributions, such as the data tempering mentioned before. The adaptive
strategies proposed. It is of interest to see if strategies studied in this
thesis can benefit more general situations.

The work presented in Chapter~\ref{cha:vSMC: A C++ Library for Parallel SMC}
can be useful for many researchers. However, it still requires considerable
expertise in \cpp. It is of interest to provide a easier to use interface on
top of the library. Some popular parallel programming models are not included
in this library, such as \cuda \cite{cuda}. Further work \dn{is} needed to include
them in the presented framework so the library can be more useful to those
familiar with them.
