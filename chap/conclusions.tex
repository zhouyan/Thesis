\chapter{Conclusions}
\label{cha:Conclusions}

This thesis is concerned with the use \smc for the purpose of Bayesian model
comparison. A generic framework was developed. Practical implementation tools
for generic \smc algorithms are also presented.

\section{Contributions}
\label{sec:Contributions}

In Chapter~\ref{cha:Sequential Monte Carlo for Bayesian Computation}, the
algorithms presented can accurately approximate the Bayes factor with little
or no human tuning for many applications of interest. Some theoretical results
of the use of path sampling estimator within this framework was developed as
extension to the results of the standard estimator. A novel adaptive algorithm
for specification of the placement of distributions in the \smc[2] algorithm
is introduced. It provides better (and more sensible) results than using the
\ess as a criterion of how to introduce a new distribution. This method can be
extended to other algorithms such as \smc[3] straightforwardly. The
performance of the presented algorithms is studied in detail through various
empirical experiments.

Chapter~\ref{cha:vSMC: A C++ Library for Parallel SMC}, a \cpp library was
developed. It provides a tool for the implementation of generic \smc
algorithms. Compared to some established softwares, it has either a higher
level of flexibility in the sense of enabling the implementation of general
algorithms instead of particular models; or higher performance through the use
of parallel computing.

\section{Further directions}
\label{sec:Further directions}

\draftnote{Note sure if ``model expansion'', where a model is expanded to
  include more parameters and the expansion is stopped when it is determined
  through model comparison that further complexity is not worthy, is a
  standard term understood by many.}

Though many convincing results have been shown in Chapter~\ref{cha:Sequential
  Monte Carlo for Bayesian Computation}, theoretical development is also
needed. In particular, the better performance of \cess based adaptive scheme
has only been shown empirically. It might be of interest to establish if it is
better in some sense when compared to some commonly used deterministic
scheme and under what conditions. Some algorithms have potential applications
in scenarios different than those shown in this thesis. For example, the
\smc[3] algorithm may be used for Bayesian model expansion. It may be of
interest to see if the approach presented in this thesis has any significant
advantage when compared to alternatives in terms of accuracy and computational
efficiency.

The work presented in Chapter~\ref{cha:vSMC: A C++ Library for Parallel SMC}
can be useful for many researchers. However, it still requires considerable
expertise in \cpp. It is of interest to provide a easier to use interface on
top of the library. Some popular parallel programming models are not included
in this library, such as \cuda \cite{cuda}. Further work are need to include
them in the presented framework so the library can be more useful to those are
familiar with them.
