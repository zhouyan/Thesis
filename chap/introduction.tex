\chapter{Introduction}
\label{cha:Introduction}

\section{Context}
\label{sec:Context}

Bayesian model comparison has been studied and practiced for a long time.
There are considerable computational difficulty when using this approach, as
many high dimensional integrations are involved. The development of Monte
Carlo algorithms has enabled the practice of Bayesian model comparison for a
wide range of realistic applications. However, algorithms such as Markov chain
Monte Carlo (\mcmc) cannot efficiently simulate high dimensional multimodal
distributions in many situations.

Population based algorithms has been developed in recent decades. However,
there are little literature on its application to Bayesian model comparison.
This thesis presents a framework based on sequential Monte Carlo (\smc)
algorithms within which Bayesian model comparison can be carried out in an
(semi-) automatic fashion while better accuracy can be obtained compared to
some other recent developments. This is made possible through the use of
various adaptive strategies.

This thesis also present work on the practical implementations of \smc
algorithms. Compared to \mcmc, practical tools for \smc are relatively fewer.
In addition, there are considerable interest in the utilization of parallel
computing for the implementation of \smc algorithms. The work presented in
this thesis provides a toolbox with which researchers can implement generic
\smc algorithm on parallel computers with relative ease.

\section{Outline}
\label{sec:Outline}

This thesis is concerned with the methodology of using \smc algorithms for the
purpose of Bayesian model comparison and their practical implementations. A
realistic model, positron emission tomography (\pet) compartmental model, will
be used as a running example throughout this thesis to demonstrate various
methodologies. Works on the application of Bayesian model comparison for \pet
was published in \cite{Zhou2013}. This model is introduced later in this
introduction chapter. It is followed by the following chapter.
\begin{description}
  \item[Chapter 2] provides a review of some commonly used model selection
    methods. In particular some information-theoretic selection criterion and
    the Bayesian approach. By comparison, it will be shown that Bayesian model
    comparison is still of interest for some realistic applications where its
    use was previously limited by the computational cost.
  \item[Chapter 3] reviews some Monte Carlo algorithms in the context of
    Bayesian model comparison. It will be shown that there are considerable
    difficulty for many problems of interest.
  \item[Chapter 4] presents a framework based on \smc that can be used for the
    purpose of Bayesian model comparison. In particular, various adaptive
    strategies will be discussed. This chapter is an extension to
    \cite{Zhou:2012uz} and \cite{Zhou:2013vx}.
  \item[Chapter 5] presents a \cpp library for the practical implementation of
    \smc algorithms. Parallel computing is of particular interest of this
    work. Part of this chapter is based on \cite{vsmcjss}.
\end{description}
