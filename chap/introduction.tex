\chapter{Introduction}
\label{cha:Introduction}

\section{Context}
\label{sec:Context}

\draftnote{Add a gentler intro paragraph at the beginning. I might still
  improve it before all is settled.}

Model comparison and selection are problems found throughout the discipline of
statistics. It can appear in different forms, such as the choice of regressors
in regression analysis, or the determination of the number of components in
mixture models. Often, there can be more than one model that can be
potentially used to describe the data and to make predications or for other
purposes. However, some models might be better than others in the sense that
the estimation and predication based on them have smaller errors or variances,
etc. Some models are more simpler than others while providing the comparable
accuracy. In many application areas, model selection is important for the
purpose of identifying the underlying reasons of certain phenomena observed
through the data. It is also important when finding the best estimation or
prediction procedures for some applications.

Bayesian model comparison has been studied and practiced for a long time.
There are considerable computational difficulty when using this approach, as
many high dimensional integrations are involved. The development of Monte
Carlo algorithms has enabled the practice of Bayesian model comparison for a
wide range of realistic applications. However, algorithms such as Markov chain
Monte Carlo (\mcmc) cannot efficiently simulate high dimensional multimodal
distributions in many situations.

Population based algorithms has been developed in recent decades. However,
there are little literature on its application to Bayesian model comparison.
This thesis presents a framework based on sequential Monte Carlo (\smc)
algorithms within which Bayesian model comparison can be carried out in an
(semi-) automatic fashion while better accuracy can be obtained compared to
some other recent developments. This is made possible through the use of
various adaptive strategies.

This thesis also present work on the practical implementations of \smc
algorithms. Compared to \mcmc, practical tools for \smc are relatively fewer.
In addition, there are considerable interest in the utilization of parallel
computing for the implementation of \smc algorithms. The work presented in
this thesis provides a toolbox with which researchers can implement generic
\smc algorithm on parallel computers with relative ease.

\section{Notations}
\label{sec:Notations}

Most notations used in this thesis are introduced and defined in context. A
few conventions are followed through out the thesis.

Capital Latin letters, such as $X$, is used to denote random variables and
corresponding lower case letters, such as $x$ are used to denote their
realizations. In the context of Markov chain, we use notations such as $X^t$
to denote the random variable to indicate its dependency on time $t$. For
various Monte Carlo estimators, we use notations such as $X^{(i)}$ to denote
the random variables of samples, including the case of  \mcmc algorithms. The
difference between $X^t$ and $X^{(i)}$ is to explicitly express that in some
algorithms, not all samples from a Markov chain are used for estimation
purpose. For \smc algorithms, we use $X_t^{(i)}$ to denote the particle value
of of particle $i$ at time $t$. For a sequence of variables, such as
$X_1,\dots,X_n$, we use the notation $X_{1:n}$ to denote the sequence.

The letter $\data$ is used throughout this thesis to denote the data. The
letter $\Exp$ is used to denote the expectation of random variables. And
wherever appropriate, $\Exp_{\pi}$ is used to denote the expectation with
respect to a distribution $\pi$. The letters $\Pr$ is used to denote
probabilities of random events.

Other notations are usually only used in particular sections or their meaning
are clear in the context where they are used.

\section{Outline}
\label{sec:Outline}

This thesis is concerned with the methodology of using \smc algorithms for the
purpose of Bayesian model comparison and their practical implementations. A
realistic model, positron emission tomography (\pet) compartmental model, will
be used as a running example throughout this thesis to demonstrate various
methodologies. Works on the application of Bayesian model comparison for \pet
was published in \cite{Zhou2013}. This model is introduced later in this
introduction chapter. It is followed by the following chapter.
\begin{description}
  \item[Chapter 2] provides a review of some commonly used model selection
    methods. In particular some information-theoretic selection criterion and
    the Bayesian approach. By comparison, it will be shown that Bayesian model
    comparison is of interest for some realistic applications where its
    use was previously limited by the computational cost.
  \item[Chapter 3] reviews some Monte Carlo algorithms in the context of
    Bayesian model comparison. It will be shown that there are considerable
    difficulty for many problems of interest.
  \item[Chapter 4] presents a framework based on \smc that can be used for the
    purpose of Bayesian model comparison. In particular, various adaptive
    strategies will be discussed. This chapter is an extension to
    \cite{Zhou:2012uz} and \cite{Zhou:2013vx}.
  \item[Chapter 5] presents a \cpp library for the practical implementation of
    \smc algorithms. Parallel computing is of particular interest of this
    work. Part of this chapter is based on \cite{vsmcjss}.
\end{description}
