\ifx\inthesis\undefined % In Thesis
\documentclass[11pt, hyper, bib, fontset=Cambria]{marticle}
\let\firstyear\iftrue  \let\endfirstyear\fi
\let\secondyear\iftrue \let\endsecondyear\fi
\let\thirdyear\iffalse \let\endthirdyear\fi

\def\DeclareAbbr#1#2{\def#1{\textsc{#2}\xspace}}
\DeclareAbbr{\ess}{ess}
\DeclareAbbr{\aic}{aic}
\DeclareAbbr{\bic}{bic}
\DeclareAbbr{\dic}{dic}
\DeclareAbbr{\fic}{fic}
\DeclareAbbr{\fpe}{fpe}
\DeclareAbbr{\gic}{gic}
\DeclareAbbr{\mcmc}{mcmc}
\DeclareAbbr{\mdl}{mdl}
\DeclareAbbr{\mle}{mle}
\DeclareAbbr{\nic}{nic}
\DeclareAbbr{\nls}{nls}
\DeclareAbbr{\pet}{pet}
\DeclareAbbr{\rjmcmc}{rjmcmc}
\DeclareAbbr{\smc}{smc}
\DeclareAbbr{\tic}{tic}
\DeclareAbbr{\unix}{unix}

\def\bugs{\textsc{\sffamily bugs}\xspace}
\def\jags{\textsc{\sffamily jags}\xspace}

\def\aicc{\hbox{\aic{}\ensuremath{_c}}\xspace}

\def\kl{Kullback-Leibler divergence\xspace}
\def\mha{Metropolis-Hastings algorithm\xspace}

\def\dkl{D_{\mathrm{KL}}}

\def\bth{{\mathbfit{\theta}}}
\def\bTh{{\mathbfit{\Theta}}}

\def\hbth{\widehat{\bth}}
\def\tbth{\widetilde{\bth}}

\def\xm{(x^1,\dots,x^m)}
\def\Xm{(X^1,\dots,X^m)}
\def\bthm{(\bth^1,\dots,\bth^m)}
\def\bThm{(\bTh^1,\dots,\bTh^m)}
\def\vec#1#2{#1_1,\dots,#1_{#2}}
\def\svec#1#2{#1^1,\dots,#1^{#2}}
\def\mcvec#1#2{#1^{(1)},\dots,#1^{(#2)}}

\def\calGa{\mathcal{Ga}}
\def\calK{\mathcal{K}}
\def\calLN{\mathcal{LN}}
\def\calL{\mathcal{La}}
\def\calM{\mathcal{M}}
\def\calN{\mathcal{N}}

\def\bbN{\mathbb{N}}
\def\bbR{\mathbb{R}}

\def\bfX{\mathbfit{X}}
\def\bfx{\mathbfit{x}}
\def\bfY{\mathbfit{Y}}
\def\bfy{\mathbfit{y}}
\def\bfu{\mathbfit{u}}

\addbibresource{../thesis.bib}
\author{Yan Zhou}
\date{\today}

\title{Monte Carlo Methods}
\begin{document}
\maketitle
\else % In Thesis
\chapter{Monte Carlo Methods}
\label{cha:Monte Carlo Methods}
\fi % In Thesis

In its simplest form, the classical Monte Carlo method approximate the
expectation of a function $\varphi$ with respect to a distribution $\pi$,
\begin{equation}
  \Exp_{\pi}[\varphi(X)] = \int\varphi(x)\pi(x)\intd x
\end{equation}
provided the above expectation exists, by drawing \iid samples from $\pi$, say
$\{\Xiu\}_{i=1}^N$, and approximating the expectation by the empirical average,
\begin{equation}
  \hat\varphi^N = \frac{1}{N}\sum_{i=1}^N\varphi(\Xiu).
  \label{eq:vanilla mc}
\end{equation}
The estimator $\hat\varphi^N$ converges almost surely to
$\Exp_{\pi}[\varphi(X)]$ when $N\to\infty$.  Clearly this method can only be
applied when drawing samples directly from the target distribution $\pi$ is
possible, which is of not the case in the context Bayesian model comparison,
where the target distribution are usually complex posterior distribution only
known up to a normalizing constant. In addition the use of samples from $\pi$
is in general not necessary optimal in the sense that the variance of the
estimator is not minimized \parencite[see][sec.~3.3]{Robert:2004tn}. Many
improved Monte Carlo estimation methods and algorithms for simulation from
complex distributions have been developed in the past few decades. In the
remaining of this chapter, some of the most important ones are reviewed. Their
use in the context of Bayesian model comparison is also analyzed.

\section{Importance sampling}
\label{sec:Importance sampling}

The \emph{importance sampling} method is based on the identity, termed
\emph{importance fundamental identity},
\begin{equation}
  \Exp_{\pi}[\varphi(X)]
  = \int\varphi(x)\pi(x)\intd x
  = \int\varphi(x)\frac{\pi(x)}{\eta(x)}\eta(x)\intd x
  = \Exp_{\eta}\Square[Big]{\varphi(X)\frac{\pi(X)}{\eta(X)}},
\end{equation}
where $\eta$ is a distribution with respect to which $\pi$ is absolutely
continuous. Thus given \iid samples $\{\Xiu\}_{i=1}^N$ from distribution
$\eta$, the expectation $\Exp_{\pi}[\varphi(X)]$ can be approximated by the
following importance sampling estimator,
\begin{equation}
  \hat\varphi_{\text{\textsc{is}}}^N
  = \frac{1}{N}\sum_{i=1}^N\varphi(\Xiu)\frac{\pi(\Xiu)}{\eta(\Xiu)}.
\end{equation}
The above estimator also converges almost surely to $\Exp_{\pi}[\varphi(X)]$
when $N\to\infty$. However its variance is not necessarily finite. In general
the variance is finite if and only if,
\parencite[see][sec.~3.3]{Robert:2004tn},
\begin{equation}
  \int(\varphi(x))^2\frac{(\pi(x))^2}{\eta(x)} < \infty.
\end{equation}
This suggests that the density $\eta$ shall have tails heavier than those of
$\pi$. To access the above inequality, evaluating of a more complex
integration than the original problem is required. Some rather restrictive
sufficient conditions for finite variance estimator was mentioned in
\textcite{Geweke:1989tm}.

In practice both $\pi$ and $\eta$ are often only known up to some normalizing
constants, which can be approximated with the same samples. This leads to the
estimator,
\begin{equation}
  \hat\varphi_{w\text{\textsc{is}}}^N
  = \frac{\sum_{i=1}^N\wiu\varphi(\Xiu)}{\sum_{i=1}^N\wiu}
\end{equation}
where $\wiu = \pi(\Xiu)/\eta(\Xiu)$, and is termed the \emph{weights}.  This
estimator also converges almost surely to $\Exp_{\pi}[\varphi(X)]$ when
$N\to\infty$. Though this estimator has a small bias, the variance and mean
square errors are improved in some situations when compared to the vanilla
Monte Carlo methods as in equation~\eqref{eq:vanilla mc}
\parencite{Casella:1998tj}. As suggested in
\textcite[][sec.~3.3]{Robert:2004tn}, to minimize the variance of the
estimator, the distribution $\eta$ should be chosen such that
$|\varphi(x)|\pi(x)/\eta(x)$ is almost constant with a finite variance. That
is, it is preferable for the distribution $\eta$ to be proportional to
$|\varphi|\pi$ and have heavier tails.

The importance sampling technique is rarely used directly for estimating the
marginal likelihood which is used to compute the Bayes factor for the purpose
of Bayesian model comparison. Some techniques, such as the widely used
harmonic mean estimator and its generalizations, are somehow related. However,
they are mostly used with dependent samples from Markov chain Monte Carlo
algorithms and are therefore introduced later in the next section. The
importance sampling method however is fundamental to a more advanced family of
Monte Carlo methods, sequential Monte Carlo, whose application in Bayesian
model comparison is the central topic of this thesis.

\section{Markov chain Monte Carlo}
\label{sec:Markov chain Monte Carlo}

Both the vanilla Monte Carlo and importance sampling methods require
simulations directly from a distribution, which is often not feasible in
realistic applications. Especially for the purpose of Bayesian modeling, the
target is often a complex posterior distribution. Estimation techniques based
on dependent samples were developed. The most important type, Markov chain
Monte Carlo (\mcmc), uses dependent samples generated by a Markov chain with
the target as a limiting distribution for estimation of desired expectations.

% \subsection{Discrete time Markov chains}
% \label{sub:Discrete time Markov chains}

\note{A compact yet necessary review of discrete time Markov chains is
  lacking}

The basic idea is that, given a Markov chain $(X^{(1)},\dots,\Xiu,\dots)$,
whose limiting distribution is $\pi$, the quantity $\frac{1}{N}\sum{i=1}^N
\varphi(\Xiu)$ converges almost surely to $\Exp_{\pi}[\varphi(X)]$ as
$N\to\infty$ for a $\pi$-integrable function $\varphi$. And therefore samples
generated by this Markov transition can be used for estimation of various
quantities in a similar fashion as with vanilla Monte Carlo and importance
sampling methods.

The construction of such a $\pi$-invariant Markov kernel leads to development
of various widely used \mcmc algorithms. In this section, some of the most
important ones are reviewed.

\subsection{Metropolis-Hastings algorithms}
\label{sub:Metropolis-Hastings algorithms}

In its most general form, a Metropolis-Hastings algorithm produces a Markov
chain with limiting distribution $\pi$ with a conditional density $q(\cdot|x)$
called the \emph{instrumental} or \emph{proposal} distribution through the
following transition. At time $t$, given sample $x^t$,
\begin{enumerate}
  \item Draw $Y^t \sim q(y|x^t)$.
  \item Take
    \begin{equation*}
      X^{t+1} =
      \begin{cases}
        Y^t, &\text{with probability } \rho(x^t,Y^t),\\
        x^t  &\text{with probability } 1 - \rho(x^t,Y^t).
      \end{cases}
    \end{equation*}
    where
    \begin{equation}
      \rho(x,y) =
      \min\Curly[Big]{\frac{\pi(y)}{\pi(x)}\frac{q(x|y)}{q(y|x)},1}.
    \end{equation}
\end{enumerate}
The probability $\rho(x,y)$ is called the \emph{Metropolis-Hastings acceptance
  probability}. The conditions for the Markov chain produced by this algorithm
having the limiting distribution $\pi$ are quite minimal
\parencite[see][chap.~7]{Robert:2004tn}. The design of efficient proposal $q$
has been a difficult problem and has attracted substantial attention in the
past. The efficiency of the proposals critically influence the performance of
the resulting estimators, etc. A proposal independent of the current state
$X^t$ leads to the independent Metropolis-Hastings algorithm. A more widely
used family of proposals are the random walks.

A random walk Metropolis-Hastings algorithm use a proposal that is symmetric
($g(t) = g(-t)$). The random walk does not satisfies conditions for a
uniformly ergodicity. However it is geometric ergodic under certain
conditions. \textcite{Mengersen:1996th} proposed a condition based on
log-concavity of $\pi$ in the tails. However we found that in practice it is
often too difficult to verify those conditions for complex models. However a
more important limitation of random walks is similar to that of independent
proposals is the difficulty of designing an efficient proposal.  Intuitively
the algorithm can easily produce chains that either move too fast or too slow.
In particular of multimodal densities whose modes are separated by extreme
small probability areas, these areas clearly limited the move of the random
walks. If the chains move too fast, than it is very likely that most proposed
values fall in small probability areas and the probability of jumping from one
mode to another is arbitrarily small. This often leads to extreme small
acceptance rates. On the other hand, if the chain moves too slow, it will take
long time for the chain to explore the whole parameter space.

Many adaptive strategies have been developed for this family of algorithms and
others \parencite[see][for a recent review]{Andrieu:2008kh}.

\subsection{Gibbs sampling}
\label{sub:Gibbs sampling}

In a general setting, a multi-stage Gibbs sampler assumes that the random
variable $X$ can be written as $X = (X_1,\dots,X_p)$, where $X_i$'s are either
unidimensional or multidimensional. Moreover, suppose that we can simulate
from the corresponding conditional densities $\pi_1,\dots,\pi_p$, defined as,
\begin{equation}
  X_i|x_1,\dots,x_{i-1},x_{i+1},\dots,x_p
  \sim \pi_i(x_i|x_1,\dots,x_{i-1},x_{i+1},\dots,x_p)
\end{equation}
for $i = 1,\dots,p$. The associated \emph{Gibbs sampler} is given by the
following algorithm that transit $X^t$ to $X^{t+1}$. Given $X^t$ at time $t$,
by generating $X^{t+1}$ in $p$ steps. At each step $i$, $X_i^{t+1}$ is
generated from
\begin{equation}
  X_i^{t+1} \sim
  \pi_i(x_i|x_1^{t+1},\dots,x_{i-1}^{t+1},x_{i+1}^t,\dots,x_p^t).
\end{equation}
The densities $\pi_1,\dots,\pi_p$ are called the \emph{full conditionals}.

Gibbs samples have several features that make it one of the most popular \mcmc
algorithms used in practice, but also limited its use in some cases
\parencite[see][chap.~10]{Robert:2004tn}.
\begin{enumerate}
  \item A Gibbs sampler has an acceptance rate of $1$, which is quite obvious
    from the setting of this algorithm. Therefore all samples are used in
    estimations, etc.
  \item The construction of a Gibbs sampler requires prior knowledge of the
    analytical properties of the target. This largely limited its use in the
    cases of complex models.
  \item The Gibbs sampler does not apply to problems where the number of
    parameters varies.
\end{enumerate}

\subsection{Computation of Bayes factor with \protect\mcmc samples}
\label{sub:Computation of Bayes factor with mcmc samples}

Both of the two algorithms described so far can provide within model
simulations. These dependent samples can be used for the purpose of Bayesian
model comparison for finite set of models through estimating the marginal
likelihood and thus the Bayes factor. Recall that, the marginal likelihood is
written as,
\begin{equation*}
  p(\data|k) = \int f(\data|\theta_k,k)\pi(\theta_k|k)\intd\theta_k,
\end{equation*}
For simplicity, in this section we drop the dependency on the model indicator
$k$ and simply write $p(\data) = \int f(\data|\theta)\pi(\theta)\intd\theta$.  With
samples generated by a \mcmc algorithm targeting the posterior distribution
$\pi(\theta|\data) \propto f(\data|\theta) \pi(\theta)$ available, say $\{\supi\theta\}_{i=1}^N$,
an estimator of $p(\data)$ can be obtained by the harmonic mean
\parencite{Newton:1994wm},
\begin{equation}
  \widehat{p(\data)}_{\text{\textsc{hm}}}^N =
  \Round[Big]{\frac{1}{N}\sum_{i=1}^N\frac{1}{f(\data|\supi\theta)}}^{-1}
\end{equation}
Unfortunately this estimator can suffer instability problem when samples with
small likelihoods are generated. In fact this estimator does not always has a
finite variance an therefore in general does not satisfy a Central Limit
Theorem. However, sampling from the posterior density provides other
advantages. For example, the posterior means of parameters can be
simultaneously estimated, which are the Bayesian estimators for quadratic loss
functions. An improvement seen in \textcite{Kass:1995vb} is,
\begin{equation}
  \widehat{p(\data)}_{\text{\textsc{ghm}}}^N = \Round[Big]{
    \frac{1}{N}\sum_{i=1}^N
    \frac{\gamma(\supi\theta)}{f(\data|\supi\theta)\pi(\supi\theta)}}^{-1},
\end{equation}
where $\gamma$ is a proper density function. This is based on the identity,
\begin{equation}
  \frac{1}{p(\data)}
  = \int \frac{\gamma(\theta)}{p(\data,\theta)}
  \frac{p(\data,\theta)}{p(\data)} \intd \theta
  = \int \frac{\gamma(\theta)}{p(\data,\theta)} \pi(\theta|\data) \intd \theta
\end{equation}
It can be seen that the density $\gamma$ plays a role similar to the
importance sampling distribution. For similar reasons, high efficiency is most
likely to be achieved if $\gamma$ is roughly proportional to $f(\data|\theta)$
\parencite{Kass:1995vb}. And the above equation suggests that the estimator
has a finite variance if the tails of $\gamma$ is thin enough compared to the
unnormalized posterior distribution $\pi(\theta|\data)$.
\textcite{Gelfand:1994ux} suggested the use of a multivariate normal
distribution with moments approximated from the samples as a natural choice of
$\gamma$. In addition, the variance of the estimator can also be estimated for
$1/\widehat{p(\data)}$ from the posterior samples through,
\begin{equation}
  \var\Square[Bigg]{\frac{1}{\widehat{p(\data)}}} =
  \frac{1}{N^2}\sum_{i=1}^N \Round[Bigg]{
    \frac{\gamma(\supi\theta)}{f(\data|\supi\theta)\pi(\supi\theta)}
    - \frac{1}{\widehat{p(\data)}}}^2.
\end{equation}
The above equation provides a way of monitoring the convergence of the
estimator. Though more stable than the harmonic mean estimator
$\widehat{p(\data)}_{\text{\textsc{hm}}}$, this generalized estimator still
requires considerable care in the implementation, especially the choice of the
density $\gamma$ to ensure good performance and indeed a finite variance
estimator.

In the particular case of Gibbs sampler, \textcite{Chib:1995em} provides an
alternative approach based on that the identity,
\begin{equation}
  p(\data) = \frac{f(\data|\theta)\pi(\theta)}{\pi(\theta|\data)}
\end{equation}
holds for any value of $\theta$. Therefore an estimator can be obtained by
substituting $\theta$ with a specific value, say $\theta^*$, which is usually
chosen from the high probability region of the posterior distribution and
approximating the denominator using outputs from the Gibbs sampler. In
addition to the usual requirement of a Gibbs sampler, that all the full
conditionals can be sampled from, this method also requires that all these
densities are known including their normalizing constants, and thus can be
computed point-wise. The advantage is that this estimator does not suffer from
the instability problem as the harmonic mean estimator and its
generalizations. A generalization to other Metropolis-Hastings algorithms was
provided by \textcite{Chib:2001gq}, where only the proposal distributions are
required to be known including their normalizing constants.

\subsection{Reversible jump \protect\mcmc}
\label{sub:Reversible jump mcmc}

The reversible jump \mcmc (\rjmcmc) algorithm, introduced by
\textcite{Green:1995dg}, is a technique widely used for simulations where the
dimension of the parameter space is not fixed. In the context of Bayesian
model selection, it can used for inference of the full posterior density
$\pi(\theta_k,k|\data)$ density, which is defined on the space $\Theta =
\bigcup_{k\in\calK}\{k\}\times\Theta_k$ (see section~\ref{sec:Bayesian model
  choice}.) Situations where this can be reduced to the estimation of the
Bayes factor (section~\ref{sub:Bayes factor}), techniques reviewed above can
be used. However, when $\calK$ is (infinite) countable, or for other reasons,
direct inference on the full posterior distribution is desired, \rjmcmc is the
most widely used technique.

Not unlike the \mcmc algorithms reviewed before, \rjmcmc also relies on the
existence of the detailed balance condition, only the parameter space is more
complicated and hence the design the Markov transition kernel. The \rjmcmc
algorithm adapt the Metropolis-Hastings algorithm to construct such kernels.
Instead a single type of moves defined a proposal density, a countable set of
moves are considered, say $m\in\calM$. Each type of moves is capable of moving
the current state of the Markov chain between, say $\Theta_k$ and
$\theta_{k'}$ (where in the case of $k = k'$, the move is similar to a those
in a \mcmc algorithm on a fixed dimension space). At state
$\theta_k\in\Theta_k$, a move type $m$ together with a new state
$\theta_{k'}\in\Theta_{k'}$ are proposed according to the
$q(theta_k,theta_{k'}r_m(\theta_k)$, where $r_m(\theta_k)$ is the probability
of choosing type $m$ move when at state $\theta_k$; and
$q(\theta_k,\theta_{k'})$ is the proposal kernel for the new state when a move
of type $m$ is made. The move is accepted with probability,
\begin{equation}
  \alpha(\theta_k,\theta_{k'}) =
  \min\Curly[Big]{1,
    \frac{\pi(M_{k'})\pi(\theta_{k'}|M_{k'})p(\data|\theta_{k'},M_{k'})}
    {\pi(M_k)\pi(\theta_k|M_k)p(\data|\theta_k,M_k)}
    \frac{q_m(\theta_{k'},\theta_k)rm_(\theta_{k'})}
    {q_m(\theta_k,\theta_{k'})rm_(\theta_k)}
  }.
\end{equation}
In practice, the proposed new state $\theta_{k'}$ is often implemented by
drawing a vector of continuous random variables, say $u$, independent of
$\theta_k$ and a deterministic mapping of vector $(\theta_k,u)$ to
$\theta_{k'}$. The inverse of the move from $\theta_{k'}$ back to $\theta_k$
then uses the inverse of this transformation. Through a simple change of
variable, the conditional density $q_m(\theta_k,\theta_{k'})$ can be expressed
in terms of the density of vector $u$ and its density $q(u)$. The acceptance
probability becomes
\begin{equation}
  \alpha(\theta_k,\theta_{k'}) =
  \min\Curly[Big]{1,
    \frac{\pi(M_{k'})\pi(\theta_{k'}|M_{k'})p(\data|\theta_{k'},M_{k'})}
    {\pi(M_k)\pi(\theta_k|M_k)p(\data|\theta_k,M_k)}
    \frac{r_m(\theta_{k'}}{r_m(\theta_k)}
    \frac{1}{q(u)}
    \Vert[Big]{

    }
  }.
\end{equation}

The method for constructing the reversible kernel, say $K$ which moves states
within the variable space $\Theta$, outlined below is based on
\textcite[][chap.~11]{Robert:2004tn}. To simplify the notations, we use
variables $x$, $y$, etc., instead of $(\theta_k, k)$. Also let $\pi(\diff x)$
denote the target distribution (it is the posterior density in Bayesian model
selection context). And as before, $\Theta_k$ denote the model parameter space
for $k$. The whole parameter space is denoted by $\Theta$ as in
equation~\eqref{eq:variable dimension parameter space}. The kernel $K$ is
decomposed according to the model in which it proposes a move to another model
(possible the same model). For model $k$, let $q_k$ denotes a transition
measure on $\Theta_k$ and $\rho_k$ denote the corresponding acceptance
probability. Then,
\begin{equation}
  K(x,B) = \sum_k\int_B\rho_k(x,y)q_k(x,dy)+\omega(x)\mathbb{1}_B(x),
\end{equation}
where
\begin{equation}
  \omega(x) = 1 - \sum_k(\rho_k q_k)(x,\Theta_k),
\end{equation}
that is the probability of no move. The fundamental assumption in
\textcite{Green:1995dg} is that joint measure $\pi(\diff x)q_k(x,\diff y)$
\emph{must be absolutely continuous with respect to a symmetric measure}
$\xi_k(\diff x, \diff y)$ on $\Theta\times\Theta$. That is, there exists a density
$g_k(x,y)$ of $q_k(x,\diff y)\pi(\diff x)$ against this dominating measure.
Then the acceptance probability $\rho_k$ can be written in the
Metropolis-Hastings form,
\begin{equation}
  \rho_k(x,y) = \min\Curly[Big]{1,\frac{g_k(y,x)}{g_k(x,y)}}.
\end{equation}
In such a setting the reversibility is ensured by the symmetry of the measure
$\xi_k$.

The first difficulty of the \rjmcmc approach is the determination of the
measure $\xi_k$, which is restricted to be symmetric. \textcite{Green:1995dg}
proposed the use of auxiliary variables. For example in the notations of
settings of section~\ref{sec:Variable dimension models}, if the jumps are
decomposed into moves between pairs of models, say $\calM_1$ and $\calM_2$
(that is $k_1$ and $k_2$), then artificial spaces are supplemented to
the model parameter spaces $\Theta_{k_1}$ and $\Theta_{k_2}$ to create a bijective
mapping between them. Without losing generality, assuming $\dim(\Theta_{k_1}) >
\dim(\Theta_{k_2})$ and if the move from $\Theta_{k_1}$ to $\Theta_{k_2}$ can be
represented by a deterministic transformation $\theta_{k_2}= T(\theta_{k_1})$,
\textcite{Green:1995dg} imposes a dimension matching condition which is that
the move from $\calM_2$ to $\calM_1$ is concentrated on the curve,
\begin{equation}
  \{\theta_{k_1}:\theta_{k_2}= T(\theta_{k_1})\}.
\end{equation}
In practice, $\theta_{k_i}$ is completed by a simulation $\bfu_i$ from density
$g_i$ into $(\theta_{k_i},\bfu_i)$. The mapping from $(\theta_{k_1},\bfu_i)$ to
$(\theta_{k_2},\bfu_2)$ is a bijection $T$. The probability of acceptance for
the move from $\calM_1$ to $\calM_2$ is therefore,
\begin{equation}
  \min\Curly[Big]{1,
    \frac{\mu(\theta_{k_2},k_2)}{\mu(\theta_{k_1},k_1)}
    \frac{p_{21}}{p_{12}} \frac{g_2(\bfu_2)}{g_1(\bfu_1)}
    \Abs[Big]{
      \frac{\partial T(\theta_{k_1},\bfu_1)}{\partial (\theta_{k_1},\bfu_1)}}},
\end{equation}
where $p_{ij}$ is the probability of proposing a move to $\calM_j$ when the
current state is $\calM_i$, and $g_i$ is the density of $\bfu_i$. This
proposal satisfies the detailed balance condition if the move from $\calM_2$
to $\calM_1$ also satisfies the bijection $T$. \textcite{Green:1995dg} pointed
out that the target density $\mu$ does not need to be normalized, but the
posterior densities for difference models have to be known up to the
\emph{same} normalizing constant.

Though \textcite{Green:2009tr} argued that the additional challenges of
implementation of \rjmcmc is just a ``myth'', we still outlines some
difficulties of these techniques. The ability of simulation the posterior
density $\mu(\theta_k,k)$ is obviously very appealing to the Bayesian
model selection problem. And the \rjmcmc techniques is under active
development today.

The main difficulties lie in the choice of cross-model proposals and the
bijection $T$. Though the mapping $T$ theoretically is quite flexible, its
creation and optimization can be quite difficult in practice. This is
specifically true when the parameter space is complicated. In some extreme
cases, creating a valid kernel is already difficult, leave alone the
optimization. Inefficient proposals results in Markov chains that are slow to
explore the whole parameter space. This is the same as for the \mcmc
algorithms. However the natural ideas of neighborhood and others, which proved
to be very practical for within model simulations, may no longer be intuitive
in the variable dimension model settings.

Some discussion of the optimization of the cross-model moves can be found in
\textcite{Green:2009tr}. Also similar development to the adaptive scheme for
\mha can be found extended to for \rjmcmc \parencite[for
example][]{Hastie:2005vi}. However little other work are known for this kind
of improvement of \rjmcmc. \textcite{Green:2011tk} discussed a method called
\emph{delayed rejection}. In this method, a rejection of a proposal does not
immediately lead to the acceptance of current state, instead a second proposal
is attempted. Their numerical results showed efficiency improve but with
increased computation cost.

Despite all these efforts, the challenges of implementations of \rjmcmc is
still the main issue that limits its use in practice. We are going to look for
alternative ways of doing Bayesian computation and it is proposed in the next
chapter.

\subsection{Population \protect\mcmc}
\label{sub:Population mcmc}

\section{Sequential Monte Carlo}
\label{sec:Sequential Monte Carlo}

\subsection{Sequential importance sampling and resampling}
\label{sub:Sequential importance sampling and resampling}

\subsection{\protect\smc sampler}
\label{sub:smc sampler}

\ifx\inthesis\undefined
\printbibliography
\end{document}\else\relax\fi
