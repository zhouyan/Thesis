\chapter{Positron emission tomography compartmental model}
\label{cha:Positron emission tomography compartmental model}

\section{Compartmental model}
\label{sec:Compartmental model}

Compartmental models are a class of models that describe systems in which some
real or abstract quantity flows between different (physical or conceptual)
compartments, each with its own characteristics. It is often of interest to
infer both parameters that describe the dynamics of the system and the number
of compartments that are required in order to adequately describe measured
data within this framework.

A compartmental system comprises a finite number of macroscopic subunits
called compartments, each of which is assumed to contain homogeneous and
well-mixed material. The compartments interact by material flowing from one
compartment to another. There may be flows into one or more compartments from
outside the system (inflows) and there may be flows from one or more
compartments out of the system (outflows) \cite{Jacquez:1996gc}. In this
paper, linear compartmental models are considered, in particular those that
are identifiable in \pet studies \cite{Schmidt99}. In these models the rate
of tracer flow from a compartment is proportional to the quantity of tracer in
that compartment. In such models the flow may be parameterized by a pair of
transfer coefficients, which are termed rate constants and may take the value
zero, for each pair of compartments.

This class of models yields a set of ordinary differential equations that
describe the flow of tracer. Consider an $m$-compartment model. Let
$\mathbfit{f}(t)$ be the vector whose $i$\xth element corresponds to the
concentration in the $i$\xth compartment at time $t$. Let $\mathbfit{b}(t)$
describe all flow into the system from outside. The $i$\xth element of
$\mathbfit{b}(t)$ is the rate of inflow into the $i$\xth compartment from the
environment. The dynamics of such a model may be written as:
\begin{align*}
  \dot{\mathbfit{f}}(t) &= A\mathbfit{f}(t) + \mathbfit{b}(t), \\
  \mathbfit{f}(0) &= \mathbfit\xi,
\end{align*}
where $\mathbfit\xi$ is the vector of initial concentrations and
$\dot{\mathbfit{f}}$
denotes the time derivative of $\mathbfit{f}$. The matrix $A$ is formed from
the rate constants (see \cite{Gunn:2001cx}).  The solution to this equation
is,
\begin{equation*}
  \mathbfit{f}(t) = e^{A t}\mathbfit\xi + \int_0^t e^{A(t-s)}\mathbfit{b}(s)\intd s,
\end{equation*}
where the matrix exponential $e^{A t} = \sum_{k = 0}^{\infty}
\frac{(A t)^k}{k!}$.

\section{Application to positron emission tomography}
\label{sec:Application to positron emission tomography}

Positron emission tomography (\pet) is an analytical imaging technology that
uses compounds labelled with positron emitting radionuclides as molecular
tracers to image and measure biochemical process \emph{in vivo}. It is one of
the few methods available to neuroscientists to study biochemical processes
within living brain, as methodology such as magnetic resonance imaging is
primarily only able to study effects via blood flow changes, while \pet can
study changes in the biochemical systems themselves. This is of considerable
interest within research into diseases where biochemical changes are known to
be responsible for symptomatic changes, such as in schizophrenia and other
psychiatric diseases \cite{FrankleL2002}.  In a clinical setting, \pet is now
one of the most commonly used diagnostic procedures for cancer (both within
and outside the brain), as fluoro-deoxyglucose ([$^{18}$F]-FDG, an radiotracer
analogue of glucose) can be imaged. Cancer cells tend to be very metabolically
active, thus requiring more glucose than surrounding cells, resulting in a
greater uptake of [$^{18}$F]-FDG, leading to an indication of cancer location
on an [$^{18}$F]-FDG scan \cite{Gambhir2002}.

In a typical molecular assay, a positron-labelled tracer is injected
intravenously and the \pet camera scans a record of positron emission as
the tracer decays \cite{Phelps2000}.  With all events detected by the \pet
camera, the time course of the tissue concentrations are reconstructed as
three-dimension images \cite{Kinahan1989}. The digital image so captured
shows the signal integrated over small volume elements (voxels). Each voxel
has a volume of the order of a few cubic millimetres. This data provides
the tissue time-activity function, which is the total concentration of
tracer in all tissue compartments. In the \emph{plasma input compartmental
  model}, in addition to the \pet data, a separate measurement of the
concentration of tracer in the plasma is available. This measurement is
generally assumed to be noise free (it can be measured with much greater
accuracy than the signal of interest). This model is used in the current
study. See \cite{Gunn:2001cx} for details of \pet compartmental models in
general.

There are many reasons that linear \ode models, of which the plasma input
model is one, are the most commonly used in \pet analysis. Perhaps most
importantly, such systems have also been shown to characterize \pet
experimental data well \cite{Lammertsma96}. The amount of data available
to fit the model for each voxel is relatively small (20-40 time points),
and even with three compartment linear \ode models, the estimation of six
parameters is non-trivial; it's clear that attempting to estimate the
parameters of more general non-linear \ode systems robustly will be close
to impossible in this setting. Furthermore, on a voxel level, which is the
type of spatial analysis that is of interest here, the signal-to-noise
ratio of the data is not high, making any parameter estimation difficult.
Finally, as the models are estimated for every voxel in the brain
(typically around 100,000 voxels per scan), computational considerations
need to be taken into account. Thus, linear \ode models are both
experimentally useful, and computationally efficient and it is difficult to
justify the additional complexity that would arise from considering more
general models. In the remainder of this paper we confine ourselves to
linear \ode models for this reason, although nonlinear \ode models have
received considerable attention in other areas in recent years -- see
\cite{Lawson2011} and references therein.

A plasma input model with $m$ tissue compartments can be written as a set of
ordinary differential equations,
\begin{align*}
  \dot{\mathbfit{C}}_{\mathbfit{T}}(t)
  & = A \mathbfit{C}_{\mathbfit{T}}(t) + \mathbfit{b} C_P(t)\\
  C_T(t) & = \mathbfit{1}^T\mathbfit{C}_{\mathbfit{T}}(t) \\
  \mathbfit{C}_{\mathbfit{T}}(0) & = \mathbfit{0},
\end{align*}
where $\mathbfit{C}_{\mathbfit{T}}(t)$ is an $m$-vector of time-activity
functions of each tissue compartment, $C_P(t)$ is the plasma time-activity
function, i.e., the input function. $A$ is the $m \times m$ state transition
matrix, $\mathbfit{b} = (K_1, 0, \dots, 0)^T$ is an $m$-vector, where $K_1$ is
the rate constant of input from the plasma into tissue. The $m$-vectors
$\mathbfit{1}$ and $\mathbfit{0}$ correspond to the $m$-vectors of ones and
zeroes, respectively. The matrix $A$ takes the form of a diagonally dominant
matrix with non-positive diagonal elements and non-negative off-diagonal
elements. Furthermore, $A$ is negative semidefinite \cite{Gunn:2001cx}.  The
solution to this set of \ode{}s is:
\begin{align}
  C_T(t) & = C_P(t) \otimes H_{TP}(t) = \int_0^t C_P(t-s) H_{TP}(s)\intd s
  \label{eq:CT} \\
  H_{TP}(t) & = \sum_{i=1}^m \phi_i e^{-\theta_i t},\notag
\end{align}
where $\otimes$ is the convolution operator and the $\phi_i$ and $\theta_i$
parameters are functions of the rate constants (in the sense that there is a
one-to-one mapping between the set of rate constants and the set of
$\phi_i$ and $\theta_i$ parameters). The input function $C_P(t)$
is assumed to be nearly continuously measured. The tissue time-activity
function $C_T(t)$ is measured discretely, leading to measured values of the
integral of the signal over each of $n$ consecutive, non-overlapping time
intervals ending at time points $t_1, \dots, t_n$.  The macro parameter of
interest is the \emph{volume of distribution},
\begin{equation*}
  V_D := \int_0^{\infty} H_{TP}(t) \intd t = \sum_{i=1}^m
  \frac{\phi_i}{\theta_i}.
\end{equation*}
This corresponds to the steady state ratio of tissue concentration to
plasma concentration in a constant plasma concentration regime. That is, if
an injection of tracers into the plasma were made such that the plasma
concentration remained constant over the time, then the ratio of
concentration in the tissues to the concentration in the plasma after an
infinite time had passed would be exactly $V_D$.

It is assumed that the input is the same at all voxels of the reconstructed
image. This is not a particularly unrealistic assumption: the input is an
empirical function derived from online measurements of the concentrations
of the radiotracer within the blood, calculated on a per second basis. The
blood carries the tracer to the brain and the timescale on which the
radiotracer is measured is very fast in comparison with the time
acquisition of \pet time frames providing a temporal averaging. However,
the model for each voxel is not assumed to be the same, and different
number of compartments can be associated with each one. In the model
fitting, a ``mass univariate'' approach is taken with each voxel being
analysed separately. This approach is common in the literature and makes
the problem of dealing with a very large number of voxels feasible.
However, it imposes very stringent computational requirements: more than
200,000 voxels must be analyzed (i.e., the time series analysis must be
repeated separately for each of these voxels), meaning that robustness is
essential as complex model-specific characterizations and model/algorithm
tuning cannot be performed on a voxel by voxel basis.

The goal work is to obtain Bayesian estimates of the macro parameter $V_D$ and
also to estimate the posterior probabilities of models with different number
of tissue compartments. This macro parameter is highly important when
considering such quantities as receptor density and occupancy. In addition,
the number of compartments in the model typically can be identified with free
tracer, specifically bound tracer (tracer bound to the system under
investigation) and non-specifically bound tracer (tracer bound to different
competing systems), indicating the role of certain chemicals within particular
brain systems.

\section{Simulated and real \protect\pet data}
\label{sec:Simulated and real pet data}

Two data sets are used in this thesis. The first is simulated from a
three compartments model with the matrix of rate constants,
\begin{equation}
  A = \begin{bmatrix}
    - k_2 - k_3 - k_5 & k_4  & k_6 \\
    k_3               & -k_4 & 0   \\
    k_5               & 0    & -k_6
  \end{bmatrix},
\end{equation}
where $k_2 = 3 \times 10^{-3}$, $k_3 = 5.5 \times 10^{-3}$, $k_4 = 1.5 \times
10^{-3}$, $k_5 = 10^{-3}$ and $k_6 = 3 \times 10^{-3}$. The rate constant of
input $K_1 = 6\times 10^{-3}$. All parameters have the unit s$^{-1}$ except
$K_1$ which has unit ml s$^{-1}$ cm$^{-3}$ \cite{RLNomen}. The simulated data
has 32 time frames with lengths corresponding to the integration periods used
in real experiments ($27.5$, $32.5$, $2 \times 10$, $20$, $6 \times 30$, $75$,
$11 \times 120$, $210$, $5 \times 300$, $450$, and $2 \times 600$, all in
seconds).  Noise is added to the synthetic data such that the noise is
normally distributed with mean zero, and variance proportional to the time
activities divided by the length of time frames. The noise is scaled such that
the highest variance in the sequence is equal to a ``noise level'' variable
(with the others scaled in proportion).  This noise level ranges from $0.01$
to $5.12$, from lower than typical region of interest (\roi) analysis (in
which the data is averaged over a biologically meaningful region in order to
improve signal to noise ratio) to higher than the noise associated with
voxel-level analysis \citep{Peng:2008fx}.  For each noise level, 2,000 time
series were simulated.  Normally distributed errors were assumed (correctly,
in this simulation study). For each of these time series analysis was carried
out for each of the three possible models via likelihood-based and Bayesian
methods as detailed previously.

The other data is a real \pet scan, which is also studied in
\cite{Peng:2008fx}.  It has about a quarter of a million data sets. See
\cite{Zhou2013} for more details on this data set.

\section{Error models}
\label{sec:Error models}

In the scenarios considered in this thesis, linear one-, two-, and
three-compartment models are considered possible; the method could deal with
other compartmental models straightforwardly, but we focus on these as they
are the most interesting in the application of interest. Let $t_1, \dots, t_n$
be the end points of the time frames at which the tissue concentrations are
measured, let $y_j$, $j = 1, \dots, n$ be the observed data. Measurement error
is assumed to be white and additive with zero mean and variance proportional
to activities divided by the length of time frames (these assumptions arise
from the physical characterization of the \pet system of interest; alternative
specifications would be possible and appropriate for other situations).
Recall equation~\eqref{eq:CT},
\begin{gather*}
  C_T(t_j;\phi_{1:m},\theta_{1:m}) =
  \sum_{i=1}^m \phi_i\int_0^{t_j}C_P(s)e^{-\theta_i(t_j-s)} \intd s \\
  y_j = C_T(t_j;\phi_{1:m},\theta_{1:m}) +
  \sqrt{\frac{C_T(t_j;\phi_{1:m},\theta_{1:m})} {t_j - t_{j-1}}}
  \varepsilon_j,
\end{gather*}
where $m = 1, 2$, or~$3$ is the number of tissue compartments, $t_0 = 0$,
and $\varepsilon_j$'s are identically independently distributed random
variables with mean zero. It is usually assumed that $\varepsilon_j$'s have
a normal distribution. It is demonstrated below that there is evidence that
a $t$ distribution better fits the observed data. We consider two error
structures:
\begin{align*}
  \varepsilon_j &\sim \mathcal{N}(0,\lambda^{-1})
  &\text{Normally-distributed errors} \\
  \varepsilon_j &\sim \mathcal{T}(0,\tau,\nu)
  &\text{$t$-distributed errors},
\end{align*}
where $\mathcal{N}(0,\lambda^{-1})$ is the normal distribution with mean zero
and variance $\sigma^2$, and $\mathcal{T}(0,\tau,\nu)$ is the Student $t$
distribution with location zero, scale $\tau$, and degrees of freedom $\nu$.
