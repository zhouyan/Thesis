\chapter{vSMC: A C++ Library for Parallel Sequential Monte Carlo}
\label{cha:vSMC: A C++ Library for Parallel Sequential Monte Carlo}

The vSMC library \cite{software:VSMC} was developed during the research to
assist the implementation of various \smc and other algorithms, including but
not limited to the implementation of the illustrative and performance
comparison examples in previous chapters. It evolves into a sophisticated C++
framework for implementing \smc algorithms on both sequential and parallel
hardware.

A more comprehensive (but also much more technical) tutorial of the library
can be found in \cite{software:VSMC}. In this chapter, we provide a higher
level overview of the library structure and usage. In section~\ref{sec:A
  minimal example} we use a minimal example to demonstrate the most basic
usage of the library. Section~\ref{sec:Extensions} we discuss how one can
write simple extensions of the library to assist novel \smc related
researches. In the field of \smc, and computational statistics in general,
researchers are actively developing new algorithms. It is unlikely to have any
existing software to provide directly functionality of the newly developed
algorithms. One of the purpose of vSMC is to make it easier to develop new
\smc related algorithms on top of the presented framework, and to free
researchers from developing entirely new software to satisfy their particular
needs. In section~\ref{sec:Performance benchmark} we compare the performance
of different parallel programming models that can be used with vSMC. We also
compare the performance of vSMC against other similar softwares.

It shall be noted that, vSMC may be more ``developer friendly'' than ``user
friendly''. It does not provide \bugs-like interface for the user. However,
the more proficient one is at C++, vSMC may be used in a more flexible and
productive way than other softwares. There is always a trade-off between
easy of use and the productivity, flexibility and extensibility of a given
software work.

\section{A minimal example}
\label{sec:A minimal example}

In this section we use a minimal example to demonstrate an \smc algorithm
implemented with vSMC.

\subsection{Model and algorithm}
\label{sub:Model and algorithm}

Consider the problem of simulating a Normal distribution. Of course, for this
toy example, a direct simulation is trivial. However, for the purpose of
demonstration, we construct an \smc sampler on for sequence of distributions,
$\{\pi_t\}_{t=0}^T$ with
\begin{equation}
  \pi_t(x) = \pi_0(x)^{1 - \alpha(t/T)}\pi_T(x)^{\alpha(t/T)}
\end{equation}
where $\pi_0$ is the initial distribution and $\pi_T$ is our target
distribution. The mapping $\alpha:[0,1]\to[0,1]$ is strictly monotonic
increasing with $\alpha(0) = 0$ and $\alpha(1) = 1$. For simplicity, we choose
$\pi_0$ also be a Normal distribution. It is trivial to see that all the
intermediate distributions are all Normal. Let $\mu_0$ and $\sigma_0^2$ denote
the mean and variance of $\pi_0$, respectively; and let $\mu_T$ and
$\sigma_T^2$ denote the mean and variance of $\pi_T$, respectively. Then
$\pi_t$ has mean and variance, denoted by $\mu_t$ and $\sigma_t^2$,
\begin{align}
  \mu_t &= \frac{(1 - \alpha_t)\sigma_T^2\mu_0 + \alpha_t\sigma_0^2\mu_T}
  {(1 - \alpha_t)\sigma_T^2 + \alpha_t\sigma_0^2} \label{eq:mini mu_t}\\
  \sigma_t^2 &= \frac{\sigma_0^2\sigma_T^2}
  {(1 - \alpha_t)\sigma_T^2 + \alpha_t\sigma_0^2} \label{eq:mini var_t}
\end{align}
where $\alpha_t = \alpha(t/T)$. Again, for demonstration purpose, at each
iteration, instead of simulating from the Normal distribution directly, we use
a Normal random walk, with a kernel invariant to $\pi_t$, to move the
particles. The (unnormalized) incremental weights are thus,
\begin{equation}
  w_t(X_{t-1}^{(i)}, X_t^{(i)}) =
  \frac{\phi(X_{t-1}^{(i)},\mu_t,\sigma_t^2}
  {\phi(X_{t-1}^{(i)},\mu_{t-1},\sigma_{t-1}^2}
\end{equation}
where $\phi(x,\mu,\sigma^2)$ is the density function of Normal distribution
with mean $\mu$ and variance $\sigma^2$.

\subsection{Implementation}
\label{sub:Implementation}

Headers and definitions,
\begin{cppcode}
#define BASE_INIT    vsmc::InitializeSEQ
#define BASE_MOVE    vsmc::MoveSEQ
#define BASE_MONITOR vsmc::MonitorEvalSEQ

#include <vsmc/core/sampler.hpp>
#include <vsmc/smp/adapter.hpp>
#include <vsmc/smp/state_matrix.hpp>
#include <vsmc/mrw/normal.hpp>
#include <vsmc/smp/backend_seq.hpp>
\end{cppcode}


Classes used in this example,
\begin{cppcode}
using vsmc::Sampler;
using vsmc::Particle;
using vsmc::StateMatrix;
using vsmc::InitializeAdapter;
using vsmc::MonitorEvalAdapter;
using vsmc::SingleParticle;
using vsmc::ConstSingleParticle;
using vsmc::NormalRW;
using vsmc::cxx11::normal_distribution;
\end{cppcode}


The state type,
\begin{cppcode}
typedef StateMatrix<vsmc::RowMajor, 1, double> mini;
\end{cppcode}

The main function,
\begin{cppcode}
int main ()
{
  const size_t N = 1000;
  const size_t T = 100;
  const double mu0 = -5;
  const double sd0 = 10;
  const double muT = 5;
  const double sdT = 1;

  Sampler<mini> sampler(N);

  sampler.init(InitializeAdapter<mini, BASE_INIT>(
        [=] (SingleParticle<mini> sp) -> size_t {
          normal_distribution<> rnorm(mu0, sd0);
          sp.state(0) = rnorm(sp.rng());
          return 1;
        }));

  sampler.move(mini_move(mu0, sd0, muT, sdT, T), false);

  sampler.monitor("moments", 2,
      MonitorEvalAdapter<mini, BASE_MONITOR>(
        [] (size_t, size_t,
            ConstSingleParticle<mini> csp, double *res) {
          res[0] = csp.state(0);
          res[1] = res[0] * res[0];
        }));

  sampler.initialize().iterate(T);
  std::ofstream sampler_file;
  sampler_file.open("mini_sampler");
  sampler_file << sampler << std::endl;

  return 0;
}
\end{cppcode}


Move particles,
\begin{cppcode}
class mini_move : public BASE_MOVE<mini>
{
  public :

  mini_move (
      double mu0, double sd0,
      double muT, double sdT, size_t T) :
    mu0_(mu0), sd0_(sd0), muT_(muT), sdT_(sdT), T_(T),
    var0_(sd0 * sd0), varT_(sdT * sdT) {}

  void pre_processor (size_t, Particle<mini> &particle)
  {
    inc_weight_.resize(particle.size());
  }

  size_t move_state (size_t iter, SingleParticle<mini> sp)
  {
    inc_weight_[sp.id()] =
      (alpha(iter) - alpha(iter - 1)) * (
          normal_log_pdf(sp.state(0), muT_, sdT_) -
          normal_log_pdf(sp.state(0), mu0_, sd0_));

    NormalRW normal_rw;
    double m = mu(iter);
    double s = sd(iter);
    return normal_rw(sp.state(0), s, sp.rng(),
        [=] (double val) {
          return normal_log_pdf(val, m, s);
        });
  }

  void post_processor (size_t, Particle<mini> &particle)
  {
    particle.weight_set().add_log_weight(inc_weight_.begin());
  }

  private :

  double mu0_, sd0_, muT_, sdT_, T_, var0_, varT_;
  std::vector<double> inc_weight_;

  double alpha (size_t t) const;
  double mu (size_t t) const;
  double sd (size_t t) const;
};
\end{cppcode}


\section{Extensions}
\label{sec:Extensions}

\section{Performance benchmark}
\label{sec:Performance benchmark}

