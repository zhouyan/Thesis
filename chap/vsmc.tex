\chapter{\protect\vsmc: A C++ Library for Parallel SMC}
\label{cha:vSMC: A C++ Library for Parallel SMC}

The \vsmc library \cite{software:VSMC} was developed during the research to
assist the implementation of various \smc and other algorithms, including but
not limited to the implementation of the illustrative and performance
comparison examples in previous chapters. It evolves into a sophisticated \cpp
framework for implementing \smc algorithms on both sequential and parallel
hardware.

A more comprehensive (but also much more technical) tutorial of the library
can be found in \cite{software:VSMC}. In this chapter, we provide a higher
level overview of the library structure and usage. In section~\ref{sec:A
  minimal example} we use a minimal example to demonstrate the most basic
usage of the library. Section~\ref{sec:Extensions} we discuss how one can
write simple extensions of the library to assist novel \smc related
researches. In the field of \smc, and computational statistics in general,
researchers are actively developing new algorithms. It is unlikely to have any
existing software to provide directly functionality of the newly developed
algorithms. One of the purpose of \vsmc is to make it easier to develop new
\smc related algorithms on top of the presented framework, and to free
researchers from developing entirely new software to satisfy their particular
needs. In section~\ref{sec:Performance benchmark} we compare the performance
of different parallel programming models that can be used with \vsmc. We also
compare the performance of \vsmc against other similar softwares.

It shall be noted that, \vsmc may be more ``developer friendly'' than ``user
friendly''. It does not provide \bugs-like interface for the user. However,
the more proficient one is at \cpp, \vsmc may be used in a more flexible and
productive way than other softwares. There is always a trade-off between
easy of use and the productivity, flexibility and extensibility of a given
software work.

\section{Using the \protect\vsmc library}
\label{sec:Using the vSMC library}

\subsection{Overview}

The \vsmc library makes use of \cpp's template generic programming to
implement general \smc algorithms. This library is formed by a few major
modules listed below. Some features not included in these modules are
introduced later in context.
\begin{description}
  \item[Core] The highest level of abstraction of \smc samplers. Users
    interact with classes defined within this module to create and manipulate
    general \smc samplers. Classes in this module include \code{Sampler},
    \code{Particle} and others. These classes use user defined callback
    functions or callable objects, such as functors, to perform problem
    specific operations, such as updating particle values and weights. This
    module is documented in Section~\ref{sub:Core module}.
  \item[Symmetric Multiprocessing (\smp)] This is the form of computing most
    people use everyday, including multiprocessor workstations, multicore
    desktops and laptops. Classes within this module make it possible to write
    generic operations which manipulate a single particle that can be applied
    either sequentially or in parallel through various parallel programming
    models. A method defined through classes of this module can be used by
    \code{Sampler} as callback objects. This module is documented in
    section~\ref{sub:SMP module}.
  \item[Message Passing Interface] \mpi is the \emph{de facto} standard
    for parallel programming on distributed memory architectures. This module
    enables users to adapt implementations of algorithms written for the \smp
    module such that the same sampler can be parallelized using \mpi. In
    addition, when used with the \smp module, it allows easy implementation of
    hybrid parallelization such as \mpi/\openmp.
  \item[\opencl] This module is similar to the two above except it eases the
    parallelization through \opencl, such as for the purpose of General
    Purpose \gpu Programming (\gpgpu). \opencl is a framework for writing
    programs that can be execute across heterogeneous platforms. \opencl
    programs can run on either \cpu{}s or \gpu{}s.
\end{description}
We will introduce the core and \smp modules in detail. The other two will
requires substantial knowledge of the underlying programming models, and it is
beyond the scope of this chapter to give them proper introduction. However,
examples and performance comparison will be given in context later.

\subsection{Obtaining and installation}

\vsmc is a header only library. There is practically no installation step. The
library can be downloaded from \url{http://zhouyan.github.io/vSMC/vSMC.zip}.
After downloading and unpacking, one can start using \vsmc by ensuring that
the compiler can find the headers inside the \code{include} directory. To
permanently install the library in a system directory, one can simply copy the
contents of the \code{include} directory to an appropriate place.

Alternatively, one can use the \cmake (2.8 or later) configuration script and
obtain the source by Git. On a Unix-like system (such as Mac OS X, BSD, Linux
and others),
\begin{minted}{sh}
git clone git://github.com/zhouyan/\vsmc.git
cd vSMC
git submodule init
git submodule update
mkdir build
cd build
cmake .. -DCMAKE_BUILD_TYPE=Release
make install
\end{minted}

\subsection{Documentation}

To build the reference manual, one need Doxygen, version~1.8.3 or later.
Continuing the last step (still in the \code{build} directory), invoking
\code{make docs} will create a \code{doc} directory inside \code{build}, which
contains the \html references. Alternatively the reference manual can also be
found on \url{http://zhouyan.github.io/vSMC/doc/html/index.html}. It is beyond
the scope of this paper to document every feature of the \vsmc library. In many
places we will refer to this reference manual for further information.

\subsection{Third-party dependencies}

\vsmc uses Random123 \cite{Salmon:2011um} counter-based \rng for random number
generating. For an \smc sampler with $N$ particles, \vsmc constructs $N$
(statistically) independent \rng streams. It is possible to use millions of
such streams without a huge memory footprint or other performance penalties.
Since each particle has its own independent \rng stream, it frees users from
many thread-safety and statistical independence considerations. It is highly
recommended that the users install this library. Within \vsmc, these \rng
streams are wrapped under \cppoo \rng engines, and can be replaced by other
compatible \rng engines seamlessly. Users only need to be familiar with
classes defined in \cppoo{} \code{<random>} or their Boost equivalents to use
these \rng streams. See the documentation of the corresponding libraries for
details, as well as examples in Section~\ref{sec:A minimal example}.

The other third-party dependency is the Boost library. Version 1.49 or later
is required. However, this is actually optional provided that proper \cppoo
features are available in the standard library, for example using \clang with
\libcpp The \cppoo headers of concern are \code{<functional>} and
\code{<random>}. To instruct \vsmc to use the standard library headers instead
of falling back to the Boost library, one needs to define the configuration
macros before including any \vsmc headers. For example,
\begin{cppcode}
clang++ -std=c++11 -stdlib=\libcpp     \
    -DVSMC_HAS_CXX11LIB_FUNCTIONAL=1  \
    -DVSMC_HAS_CXX11LIB_RANDOM=1      \
    -o prog prog.cpp
\end{cppcode}
tells the library to use \cppoo{} \code{<functional>} and \code{<random>}.
The availability of these headers are also checked by the \cmake configuration
script.

\subsection{Compiler support}

\vsmc has been tested with recent versions of \clang, \gcc, \icpc and \msvc.
\vsmc can optionally use some \cppoo features to improve performance and
usability. In particular, as mentioned before, \vsmc can use \cppoo standard
libraries instead of the Boost library. At the time of writing, \clang with
\libcpp has the most comprehensive support of \cppoo with respect to standard
compliant and feature completion. \gcc~4.8, \msvc~2012 and \icpc~2013 also
have very good \cppoo support. Note that, provided the Boost library is
available, all \cppoo language and library features are optional. \vsmc can be
used with any \cppne conforming compilers.

\section{The \protect\vsmc library}
\label{sec:The vSMC library}

\subsection{Core module}
\label{sub:Core module}

The core module abstracts general \smc samplers. \smc samplers can be viewed
as formed by a few concepts regardless of the specific problems. The following
is a list of the most commonly seen components of \smc samplers and their
corresponding \vsmc abstractions.
\begin{itemize}
  \item A collection of all particle state values, namely
    $\{X_t^{(i)}\}_{i=1}^N$. In \vsmc, users need to define a class, say
    \code{T}, to abstract this concept. We refer to this as the \emph{value
      collection}. We will slightly abuse the generic name \code{T} in this
    paper. Whenever a template parameter is mentioned with the name \code{T},
    it always refers to such a value collection type unless stated otherwise.
  \item A collection of all particle state values and their associated
    weights. This is abstracted by a \code{Particle<T>} object. We refer to
    this as the \emph{particle collection}. A \code{Particle<T>} object has
    three primary sub-objects. One is the above type \code{T} value collection
    object. Another is an object that abstracts weights
    $\{W_t^{(i)}\}_{i=1}^N$. By default this is a \code{WeightSet} object. The
    last is a collection of \rng streams, one for each particle. By default
    this is an \code{RngSet} object.
  \item Operations that perform tasks common to all samplers to these
    particles, such as resampling. These are the member functions of
    \code{Particle<T>}.
  \item A sampler that updates the particles (state values and weights) using
    user defined callbacks. This is a \code{Sampler<T>} object.
  \item Monitors that record the importance sampling estimates of certain
    functions defined for the values when the sampler iterates. These are
    \code{Monitor<T>} objects. A monitor for the estimates of $E[h(X_t)]$
    computes $h(X_t^{(i)})$ for each $i = 1,\dots,N$. The function value
    $h(X_t)$ is allowed to be a vector.
\end{itemize}
Note that within the core module, all operations are applied to
\code{Particle<T>} objects, that is $\{W_t^{(i)},X_t^{(i)}\}_{i=1}^N$, instead
of a single particle. Later we will see how to write operations that can be
applied to individual particles and can be parallelized easily.

\subsubsection{Program structures}

A \vsmc program usually consists of the following tasks.
\begin{itemize}
  \item Define a value collection type \code{T}.
  \item Constructing a \code{Sampler<T>} object.
  \item Configure the behavior of initialization and updating by adding
    callable objects to the sampler object.
  \item Optionally add monitors.
  \item Initialize and iterate the sampler.
  \item Retrieve the outputs, estimates and other informations.
\end{itemize}
In this section we document how to implement each of these tasks. Within the
\vsmc library, all public classes, namespaces and free functions, are declared
in the namespace \code{vsmc}.

\subsubsection{The value collection}

The template parameter \code{T} is a user defined type that abstracts the
value collection. \vsmc does not restrict how the values shall be actually
stored. They can be stored in memory, spread among nodes of a cluster, in \gpu
memory or even in a database. However this kind of flexibility comes with a
small price. The value collection does need to fulfill two requirements. We
will see later that for most common usage, \vsmc provides readily usable
implementations, on top of which users can create problem specific classes.

First, the value collection class \code{T} has to provide a constructor of the
form
\begin{cppcode}
T (SizeType N)
\end{cppcode}
where \code{SizeType} is some integer type. Since \vsmc allows one to allocate
the states in any way suitable, one needs to provide this constructor which
\code{Particle<T>} can use to allocate them.

Second, the class has to provide a member function named \code{copy} that
copies each particle according to replication numbers given by a resampling
algorithm. For the same reason as above \vsmc has no way to know how it can
extract and copy a single particle when it is doing the resampling. The
signature of this member function may look like
\begin{cppcode}
template <typename SizeType>
void copy (std::size_t N, const SizeType *copy_from);
\end{cppcode}
The pointer \code{copy_from} points to an array that has $N$ elements,
where $N$ is the number of particles. After calling this member function, the
value of particle \code{i} shall be copied from the particle
\code{j = copy_from[i]}. In other words, particle \code{i} is a child of
particle \code{copy_from[i]}, or \code{copy_from[i]} is the parent of particle
\code{i}. If a particle \code{j} shall remain itself, then
\code{copy_from[j] == j}. How the values are actually copied is user defined.
Note that, the member function can take other forms, as usual when using \cpp
template generic programming.
The actual type of the pointer \code{copy_from}, \code{SizeType}, is
\code{Particle<T>::size_type}, which depends on the type \code{T}. For
example, define the member function as the following is also allowed,
\begin{cppcode}
void copy (int N, const std::size_t *copy_from);
\end{cppcode}
provided that \code{Particle<T>::size_type} is indeed \code{std::size_t}.
However, writing it as a member function template releases the users from
finding the actual type of pointer \code{copy_from} and the sometimes
troubling forward declaration issues. Will not elaborate such more technical
issues further.

\subsubsection{Constructing a sampler}

Once the value collection class \code{T} is defined. One can start
constructing \smc samplers. For example, the following line creates a sampler
with $N$ particles
\begin{cppcode}
Sampler<T> sampler(N);
\end{cppcode}
The number of particles is the only mandatory argument of the constructor.
There are two optional parameters. The complete signature of the constructor
is,
\begin{cppcode}
explicit Sampler<T>::Sampler (Sampler<T>::size_type N,
        ResampleScheme scheme = Stratified, double threshold = 0.5);
\end{cppcode}
The \code{scheme} parameter is self-explanatory. \vsmc provides a few built-in
resampling schemes; see the reference manual for a list of them. User defined
resampling algorithms can also be used. See the reference manual for details.
The \code{threshold} is the threshold of $\ess/N$ below which a resampling
will be performed. It is obvious that if $\text{\code{threshold}}\ge1$ then
resampling will be performed at each iteration. If
$\text{\code{threshold}}\le0$ then resampling will never be performed. Both
parameters can be changed later. However the size of the sampler can never be
changed after the construction.

\subsubsection{Initialization and updating}

All the callable objects that initialize, move and weight the particle
collection can be added to a sampler through a set of member functions. All
these objects operate on the \code{Particle<T>} object. Because \vsmc
allows one to manipulate the particle collection as a whole, in principle many
kinds of parallelization are possible.

To set an initialization method, one need to implement a function with the
following signature,
\begin{cppcode}
std::size_t init_func (Particle<T> &particle, void *param);
\end{cppcode}
or a class with \code{operator()} properly defined, such as
\begin{cppcode}
struct init_class
{ std::size_t operator() (Particle<T> &particle, void *param); };
\end{cppcode}
They can be added to the sampler through
\begin{cppcode}
sampler.init(init_func);
\end{cppcode}
or
\begin{cppcode}
sampler.init(init_class());
\end{cppcode}
respectively. \cppoo{} \code{std::function} or its Boost equivalent
\code{boost::function} can also be used. For example,
\begin{cppcode}
std::function<std::size_t (Particle<T> &, void *)> init_obj(init_func);
sampler.init(init_obj);
\end{cppcode}

The addition of updating methods is more flexible. There are two kinds of
updating methods. One is simply called \code{move} in \vsmc, and is performed
before the possible resampling at each iteration. These moves usually perform
the updating of the weights among other tasks. The other is called
\code{mcmc}, and is performed after the possible resampling. They are often
\mcmc type moves. Multiple \code{move}'s or \code{mcmc}'s are also allowed. In
fact a \vsmc sampler consists of a queue of \code{move}'s and a queue of
\code{mcmc}'s. The \code{move}'s in the queue can be changed through
\code{Sampler<T>::move},
\begin{cppcode}
Sampler<T> &move (const Sampler<T>::move_type &new_move, bool append);
\end{cppcode}
If \code{append == true} then \code{new_move} is appended to the existing
(possibly empty) queue. Otherwise, the existing queue is cleared and
\code{new_move} is added. The member function returns a reference to the
updated sampler. For example, the following move,
\begin{cppcode}
std::size_t move_func (std::size_t iter, Particle<T> &particle);
\end{cppcode}
can be added to a sampler by
\begin{cppcode}
sampler.move(move_func, false);
\end{cppcode}
This will clear the (possibly empty) existing queue of \code{move}'s and set a
new one. To add multiple moves into the queue,
\begin{cppcode}
sampler.move(move1, true).move(move2, true).move(move3, true);
\end{cppcode}
Objects of class type with \code{operator()} properly defined can also be
used, similarly to the initialization method. The queue of \code{mcmc}'s can
be used similarly. See the reference manual for other methods that can be used
to manipulate these two queues.

In principle, one can combine all moves into a single move. However, sometimes
it is more natural to think of a queue of moves. For instance, if a
multi-block Metropolis random walk consists of kernels $K_1$ and $K_2$, then
one can implement each of them as functions, say \code{mcmc_k1} and
\code{mcmc_k2}, and add them to the sampler sequentially,
\begin{cppcode}
sampler.mcmc(mcmc_k1, true).mcmc(mcmc_k2, true);
\end{cppcode}
Then at each iteration, they will be applied to the particle collection
sequentially in the order in which they are added.

\subsubsection{Running the algorithm, monitoring and outputs}

Having set all the operations, one can initialize and iterate the sampler by
calling
\begin{cppcode}
sampler.initialize((void *)param);
sampler.iterate(iter_num);
\end{cppcode}
The \code{param} argument to \code{initialize} is optional, with \code{NULL}
as its default. This parameter is passed to the user defined \code{init_func}.
The \code{iter_num} argument to \code{iterate} is also optional; the default
is $1$.

Before initializing the sampler or after a certain time point, one can add
monitors to the sampler. The concept is similar to \bugs's \code{monitor}
statement, except it does not monitor the individual values but rather the
importance sampling estimates. Consider approximating $\Exp[h(X_t)]$, where
$h(X_t) = (h_1(X_t),\dots,h_m(X_t))$ is an $m$-vector function. The importance
sampling estimate can be obtained by $AW$ where $A$ is an $N$ by $m$ matrix
where $A(i,j) = h_j(X_t^{(i)})$ and $W = (W_t^{(i)},\dots,W_t^{(N)})^T$ is the
$N$-vector of normalized weights. To compute this importance sampling
estimate, one need to define the following evaluation function (or a class
with \code{operator()} properly defined),
\begin{cppcode}
void monitor_eval (std::size_t iter, std::size_t m,
        const Particle<T> &particle, double *res)
\end{cppcode}
and add it to the sampler by calling,
\begin{cppcode}
sampler.monitor("some.user.chosen.variable.name", m, monitor_eval);
\end{cppcode}
When the function \code{monitor_eval} is called, \code{iter} is the iteration
number of the sampler, \code{m} is the same value as the one the user passed
to \code{Sampler<T>::monitor}; and thus one does not need global variable or
other similar techniques to access this value. The output pointer \code{res}
points to an $N \times m$ output array of row major order. That is, after the
calling of the function, \code{res[i * dim + j]} shall be $h_j(X_t^{(i)})$.

After each iteration of the sampler, the importance sampling estimate will be
computed automatically. See the reference manual for various ways to retrieve
the results. Usually it is sufficient to output the sampler by
\begin{cppcode}
std::ofstream output("file.name");
output << sampler << std::endl;
\end{cppcode}
where the output file will contain the importance sampling estimates among
other things. Alternatively, one can use the \code{Monitor<T>::record} member
function to access specific historical results. See the reference manual for
details of various overloaded version of this member function.

A reference to the value collection \code{T} object can be retrieved through
the \code{Particle<T>::value} member function. \code{Particle<T>} objects
manage the weights through a weight set object, which by default is of type
\code{WeightSet}. The \code{Particle<T>::weight_set} member function returns a
reference to this weigh set object. A user defined weight set class that
abstracts $\{W_t^{(i)}\}_{i=1}^N$ can also be used. The details involve some
more advanced \cpp template techniques and are documented in the reference
manual. One possible reason for replacing the default is to provide special
memory management of the weight set. For example, the \mpi module provides a
special weight set class that manages weights across multiple nodes and
perform proper normalization, computation of \ess, and other tasks.

The default \code{WeightSet} object provides some ways to retrieve weights.
Here we document some of the most commonly used. See the reference manual for
details of others. The weights can be accessed one by one, for example,
\begin{cppcode}
Particle<T> &particle = sampler.particle();
double w_i     = particle.weight_set().weight(i);
double log_w_i = particle.weight_set().log_weight(i);
\end{cppcode}
One can also read all weights into a container, for example,
\begin{cppcode}
std::vector<double> w(particle.size());
particle.weight_set().read_weight(w.begin());
double *lw = new double[particle.size()];
particle.weight_set().read_log_weight(lw);
\end{cppcode}
Note that these member functions accept general output iterators.

\subsubsection{Implementing initialization and updating}

So far we have only discussed how to add initialization and updating objects
to a sampler. To actually implement them, one writes callable objects that
operate on the \code{Particle<T>} object. For example, a move can be
implemented through the following function as mentioned before,
\begin{cppcode}
std::size_t move_func (std::size_t iter, Particle<T> &particle);
\end{cppcode}
Inside the body of this function, one can change the value by manipulating the
object through the reference returned by \code{particle.value()}.

When using the default weight set class, the weights can be updated through a
set of member functions of \code{WeightSet}. For example,
\begin{cppcode}
std::vector<double> weight(particle.size());
particle.weight_set().set_equal_weight();
particle.weight_set().set_weight(weight.begin());
particle.weight_set().mul_weight(weight.begin());
particle.weight_set().set_log_weight(weight.begin());
particle.weight_set().add_log_weight(weight.begin());
\end{cppcode}
The \code{set_equal_weight} member function sets all weights to be equal. The
\code{set_weight} and \code{set_log_weight} member functions set the values of
weights and logarithm weights, respectively. The \code{mul_weight} and
\code{add_log_weight} member functions multiply the weights or add to the
logarithm weights by the given values, respectively. All these member functions
accept general input iterators as their arguments.

One important thing to note is that, whenever one of these member functions is
called, both the weights and logarithm weights will be re-calculated,
normalized, and the \ess will be updated. The reason for not allowing changing
a single particle's weight is that, in a multi-threading environment, it is
possible for one to change one weight in one thread, and obtain another in
another thread without proper normalizing. Conceptually, changing one weight
actually changes all weights.

\subsubsection{Generating random numbers}

The \code{Particle<T>} object has a sub-object, a collection of \rng engines
that can be used with \cppoo{} \code{<random>} or Boost distributions. For
each particle \code{i}, one can obtain an engine that produces an \rng stream
independent of others by
\begin{cppcode}
particle.rng(i);
\end{cppcode}
To generate distribution random variates, one can use the
\cppoo{} \code{<random>} library. For example,
\begin{cppcode}
std::normal_distribution<double> rnorm(mean, sd);
double r = rnorm(particle.rng(i));
\end{cppcode}
or use the Boost library,
\begin{cppcode}
boost::random::normal_distribution<double> rnorm(mean, sd);
double r = rnorm(particle.rng(i));
\end{cppcode}
\vsmc itself also makes use of \cppoo{} \code{<random>} or Boost depending
on the value of the configuration macro \code{VSMC_HAS_CXX11LIB_RANDOM}.
Though the user is free to choose which one to use in their own code, there is
a convenient alternative. For each class defined in \cppoo{} \code{<random>},
it is imported to the \code{vsmc::cxx11} namespace. Therefore one can use
\begin{cppcode}
cxx11::normal_distribution<double> rnorm(mean, sd);
\end{cppcode}
while the underlying implementation of \code{normal_distribution} can be
either \cppoo standard library or Boost. The benefit is that if one needs to
develop on multiple platforms, and only some of them support \cppoo and some
of them have the Boost library, then only the configure macro
\code{VSMC_HAS_CXX11LIB_RANDOM} needs to be changed. This can be configured
through \cmake and other build systems. Of course, one can also use an
entirely different \rng system than those provided by \vsmc.

\subsection{SMP module}
\label{sub:SMP module}

\subsubsection{The value collection}

Many typical problems' value collections can be viewed as a matrix of certain
type. For example, a simple particle filter whose state is a vector of length
\code{Dim} and type \code{double} can be viewed as an $N$ by \code{Dim} matrix
where $N$ is the number of particles. A trans-dimensional problem can use an
$N$ by $1$ matrix whose type is a user defined class, say \code{StateType}.
For this kind of problems, \vsmc provide a class template
\begin{cppcode}
template <MatrixOrder Order, std::size_t Dim, typename StateType>
class StateMatrix;
\end{cppcode}
which provides the constructor and the \code{copy} member function required by
the core module interface, as well as methods for accessing individual values.
The first template parameter (possible value \code{RowMajor} or
\code{ColMajor}) specifies how the values are ordered in memory. Usually one
shall choose \code{RowMajor} to optimize data access. The second template
parameter is the number of variables, an integer value no less than $1$ or the
special value \code{Dynamic}, in which case \code{StateMatrix} provides a
member function \code{resize_dim} such that the number of variables can be
changed at runtime. The third template parameter is the type of the state
values.

Each particle's state is thus a vector of length \code{Dim}, indexed from
\code{0} to \code{Dim - 1}. To obtain the value at position \code{pos} of the
vector of particle \code{i}, one can use one of the following member
functions,
\begin{cppcode}
StateBase<RowMajor, Dim, StateType> value(N);
StateType val = value.state(i, pos);
StateType val = value.state(i, Position<Pos>());
StateType val = value.state<Pos>(i);
\end{cppcode}
where \code{Pos} is a compile time constant expression whose value is the same
as \code{pos}, assuming the position is known at compile time. One can also
read all values. To read the variable at position \code{pos},
\begin{cppcode}
std::vector<StateType> vec(value.size());
value.read_state(pos, vec.begin());
\end{cppcode}
Or one can read all values through an iterator,
\begin{cppcode}
std::vector<StateType> mat(Dim * value.size());
value.read_state_matrix(ColMajor, mat.first());
\end{cppcode}
Alternatively, one can also read all values through an iterator which points
to iterators,
\begin{cppcode}
std::vector<std::vector<StateType> > mat(Dim);
for (std::size_t i = 0; i != Dim; ++i)
    mat[i].resize(value.size());
std::vector<std::vector<StateType>::iterator> iter(Dim);
for (std::size_t i = 0; i != Dim; ++i)
    iter[i] = mat[i].begin();
value.read_state_matrix(iter.first());
\end{cppcode}

If the compiler support \cppoo{} \code{<tuple>}, \vsmc also provides a
\code{StateTuple} class template, which is similar to \code{StateMatrix}
except that the types of values do not have to be the same for each variable.
This is similar to \rlang's \code{data.frame}. For example, suppose each
particle's state is formed by two \code{double}'s, an \code{int} and a user
defined type \code{StateType}, then the following constructs a value
collection using \code{StateTuple},
\begin{cppcode}
StateTuple<ColMajor, double, double, int, StateType> value(N);
\end{cppcode}
And there are a few ways to access the state values, similar to
\code{StateMatrix},
\begin{cppcode}
double x0 = value.state(i, Position<0>());
int x2 = value.state<2>(i);
std::vector<StateType> vec(value.size());
state.read_state(Position<3>(), vec.begin());
\end{cppcode}
See the reference manual for details.

\subsubsection{A single particle}

For a \code{Particle<T>} object, one can construct a \code{SingleParticle<T>}
object that abstracts one of the particle from the collection. For example,
\begin{cppcode}
Particle<T> particle(N);
SingleParticle<T> sp(i, &particle);
\end{cppcode}
create a \code{SingleParticle<T>} object corresponding to the particle at
position \code{i}. There are a few member functions of
\code{SingleParticle<T>} that makes access to individual particles easier than
through the interface of \code{Particle<T>}. Firt \code{sp.id()} returns the
value of the argument \code{i} in the above code that created this
\code{SingleParticle<T>} object. In addition, \code{sp.rng()} is equivalent to
\code{particle.rng(i)}. Also \code{sp.particle()} returns a constant reference
to the \code{Particle<T>} object. And \code{sp.particle_ptr()} returns a
pointer to such a constant \code{Particle<T>} object. Note that, one cannot
get write access to a \code{Particle<T>} object through interface of
\code{SingleParticle<T>}. Instead, one can only get write access to a single
particle. For example, If \code{T} is a \code{StateMatrix} instantiation or
its derived class, then \code{sp.state(pos)} is equivalent to
\code{particle.value().state(i, pos)} and the reference it returns is mutable.
See the reference manual for more informations on the interface of the
\code{SingleParticle} class template. \code{SingleParticle<T>} objects are
usually not constructed by users, but rather by the libraries' other classes
in the \smp module, and passed to user defined functions, as we will see very
soon.

\subsubsection{Sequential and parallel implementations}

Once we have the \code{SingleParticle<T>} concept, we are ready to introduce
how to write implementations of \smc algorithms that manipulate a single
particle and can be applied to all particles in parallel or sequentially. For
sequential implementations, this can be done through five base classes,
\begin{cppcode}
template <typename BaseState> class StateSEQ;
template <typename T, typename D = Virtual> class InitializeSEQ;
template <typename T, typename D = Virtual> class MoveSEQ;
template <typename T, typename D = Virtual> class MonitorEvalSEQ;
template <typename T, typename D = Virtual> class PathEvalSEQ;
\end{cppcode}
The template parameter \code{BaseState} needs to satisfy the general value
collection requirements in addition to a \code{copy_particle} member function,
for example, \code{StateMatrix}. Other base classes expect \code{T} to satisfy
general value collection requirements. The details of all these class
templates can be found in the reference manual. Here we use the
\code{MoveSEQ<T>} class as an example to illustrate their usage. Recall that
\code{Sampler<T>} expect a callable object which has the following signature
as a move,
\begin{cppcode}
std::size_t move_func (std::size_t iter, Particle<T> &particle);
\end{cppcode}
For the purpose of illustration, the type \code{T} is defined as,
\begin{cppcode}
typedef StateMatrix<RowMajor, 1, double> T;
\end{cppcode}
Here is a typical example of implementation of such a function,
\begin{cppcode}
std::size_t move_func (std::size_t iter, Particle<T> &particle)
{
    std::vector<double> inc_weight(particle.size());
    for (Particle<T>::size_type i = 0; i != particle.size(); ++i) {
        cxx11::normal_distribution<double> rnorm(0, 1);
        particle.value().state(i, 0) = rnorm(particle.rng(i));
        inc_weight[i] = cal_inc_weight(particle.value().state(i, 0);
    }
    particle.weight_set().add_log_weight(inc_weight.begin());
}
\end{cppcode}
where \code{cal_inc_weight} is some function that calculates the logarithm
incremental weights. As we see, there are three main parts of a typical move.
First, we allocate a vector \code{inc_weight}. Second, we generate normal
random variates for each particle and calculate the incremental weights. This
is done through a \code{for} loop. Third, we add the logarithm incremental
weights. The first and the third are \emph{global} operations while the second
is \emph{local}. The first and the third are often optional and absent. The
local operation is also usually the most computational intensive part of \smc
algorithms and can benefit the most from parallelizations.

With \code{MoveSEQ<T>}, one can derive from this class, which defines
the \code{operator()} required by the core module interface,
\begin{cppcode}
std::size_t operator() (std::size_t iter, Particle<T> &particle);
\end{cppcode}
and customize what this operator does by defining one or more of the following
three member functions, corresponding to the three parts, respectively,
\begin{cppcode}
void pre_processor (std::size_t iter, Particle<T> &particle);
std::size_t move_state (std::size_t iter, SingleParticle<T> sp);
void post_processor (std::size_t iter, Particle<T> &particle);
\end{cppcode}
For example,
\begin{cppcode}
#include <vsmc/smp/backend_seq.hpp>

class move : public MoveSEQ<T>
{
    public :

    void pre_processor (std::size_t iter, Particle<T> &particle)
    {inc_weight_.resize(particle.size());}

    std::size_t move_state (std::size_t iter, SingleParticle<T> sp)
    {
        cxx11::normal_distribution<double> rnorm(0, 1);
        sp.state(0) = rnorm(sp.rng());
        inc_weight_[sp.id()] = cal_inc_weight(sp.state(0));
    }

    void post_processor (std::size_t iter, Particle<T> &particle)
    {particle.weight_set().add_log_weight(inc_weight_.begin());}

    private :

    std::vector<double> inc_weight_;
};
\end{cppcode}
The \code{operator()} of \code{MoveSEQ<T>} is equivalent to the single
function implementation as shown before.

In the simplest case, \code{MoveSEQ} only takes away the loop around Part 2.
However if one implement the move in such a way, and then replace
\code{MoveSEQ} with \code{MoveOMP}, the changing of the base class name causes
\vsmc to use \openmp to parallelize the loop. For example, one can declare the
class as
\begin{cppcode}
#include <vsmc/smp/backend_omp.hpp>
class move : public MoveOMP<T>;
\end{cppcode}
and use exactly the same implementation as before. Now when
\code{move::operator()} is called, it will be parallelized by \openmp. Other
backends are available in case \openmp is not available. Among them there are
\cilk and \tbb. In addition to these well known parallelization programming
models, \vsmc also has its own implementation using \cppoo{} \code{<thread>}.
There are other backends documented in the reference manual. To use any of
these parallelization, all one need to do is to change a few base class names.
In practice, one can use conditional compilation, for example, to use a
sequential implementation or a \openmp parallelized one, we can write,
\begin{cppcode}
#ifdef USE_SEQ
#include <vsmc/smp/backend_seq.hpp>
#define BASE_MOVE MoveSEQ
#endif

#ifdef USE_OMP
#include <vsmc/smp/backend_omp.hpp>
#define BASE_MOVE MoveOMP
#endif

class move : public BASE_MOVE<T>;
\end{cppcode}
And we can compile the same source into different samplers with
\code{Makefile} rules such as the following,
\begin{cppcode}
prog-seq : prog.cpp
        $(CXX) $(CXXFLAGS) -DUSE_SEQ -o prog-seq prog.cpp
prog-omp : prog.cpp
        $(CXX) $(CXXFLAGS) -DUSE_omp -o prog-omp prog.cpp
\end{cppcode}
Or one can configure the source file through a build system such as \cmake,
which can also determine which programming model is available on the system.

\subsubsection{Adapter}

Sometimes, the cumbersome task of writing a class to implement a move and
other operations can out weight the power we gain through object oriented
programming. For example, a simple \code{move_state}-like function is all that
we need to get the work done. In this case, one can create a move through the
\code{MoveAdapter}. For example, say we have implemented
\begin{cppcode}
std::size_t move_state (std::size_t iter, SingleParticle<T> sp);
\end{cppcode}
as a function. Then one can create a callable object acceptable to
\code{Sampler<T>::move} through
\begin{cppcode}
MoveAdapter<T, MoveSEQ>  move_obj(move_state);
MoveAdapter<T, MoveSTD>  move_obj(move_state);
MoveAdapter<T, MoveTBB>  move_obj(move_state);
MoveAdapter<T, MoveCILK> move_obj(move_state);
MoveAdapter<T, MoveOMP>  move_obj(move_state);
sampler.move(move_obj, false);
\end{cppcode}
These are respectively, sequential, \cppoo{} \code{<thread>}, \tbb, \cilk,
and \openmp implementations. The first template parameter is the type of value
collection and the second is the name of the base class template. Actually,
the \code{MoveAdapter}'s constructor accepts two more optional arguments, the
\code{pre_processor} and the \code{post_processor}, corresponding to the other
two aforementioned member functions. Similar adapters for the other three base
classes also exist.

Another scenario where an adapter is desired is that which backend to use
needs be decided at runtime. The above simple adapters can already be used for
this purpose. In addition, another form of the adapter is as the following,
\begin{cppcode}
class move;
MoveAdapter<T, MoveTBB, move> move_obj;
sampler.move(move_obj, false);
\end{cppcode}
where the class \code{move} has the same definition as before but it no longer
derives from any base class. The class \code{move} is required to have a
default constructor, a copy constructor and an assignment operator.

\subsection{Thread-safety and scalability considerations}
\label{sub:Thread-safety and scalability considerations}

When implementing parallelized \smc algorithms, issues such as thread-safety
cannot be avoided even though the \vsmc library hides most parallel constructs
from the user.

Classes in the \vsmc library usually guarantee that their member functions are
thread-safe in the sense that calling the same member function on different
objects at the same time from different threads is safe. However, calling
mutable member functions on the same object from different threads is usually
not safe. Calling immutable member functions is generally safe. There are a
few exceptions,
\begin{itemize}
  \item The constructors of \code{Particle} and \code{Sampler} are not
    thread-safe. Therefore if one need to construct multiple \code{Sampler}
    from different threads, a mutex protection is needed. However, subsequent
    member function calls on the constructed objects are thread-safe according
    to the above rules.
  \item Member functions that concern a single particle are generally
    thread-safe in the sense that one can call them on the same object from
    different threads as long as they are called for different particles. For
    example \code{Particle::rng} and \code{StateMatrix::state} are
    thread-safe.
\end{itemize}
In general, one can safely manipulate different individual particles from
different threads, which is the minimal requirement for scalable
parallelization. But one cannot manipulate the whole particle collection from
different threads, for example \code{WeightSet::set_log_weight}.

User defined callbacks shall generally follow the same rules. For example, for
a \code{MoveOMP} subclass, \code{pre_processor} and \code{post_processor} does
not have be thread-safe, but \code{move_state} needs to be. In general, avoid
write access to memory locations shared by all particles from
\code{move_state} and other similar member functions. One needs to take some
extra care when using third-party libraries. For example, in our example
implementation of the \code{move} class, the \code{rnorm} object, which is
used to generate Normal random variates, is defined within \code{move_state}
instead of being a class member data even though it is created with the same
parameters for each particle. This is because the call \code{rnorm(sp.rng())}
is not thread-safe in some implementations, for example, when using the Boost
library.

For scalable performance, certain practices should be avoided when
implementing member functions such as \code{move_state}. For example, dynamic
memory allocation is usually lock-based and thus should be avoided.
Alternatively one can use a scalable memory allocator such as the one provided
by \tbb. In general, in functions such as \code{move_state}, one should avoid
using locks to guarantee thread-safety, which can be a bottleneck to parallel
performance.


\section{A minimal example}
\label{sec:A minimal example}

In this section we use a minimal example to demonstrate an \smc algorithm
implemented with \vsmc.

\subsection{Model and algorithm}
\label{sub:Model and algorithm}

Consider the problem of simulating a Normal distribution. Of course, for this
toy example, a direct simulation is trivial. However, for the purpose of
demonstration, we construct an \smc sampler on for sequence of distributions,
$\{\pi_t\}_{t=0}^T$ with
\begin{equation}
  \pi_t(x) = \pi_0(x)^{1 - \alpha(t/T)}\pi_T(x)^{\alpha(t/T)}
\end{equation}
where $\pi_0$ is the initial distribution and $\pi_T$ is our target
distribution. The mapping $\alpha:[0,1]\to[0,1]$ is strictly monotonic
increasing with $\alpha(0) = 0$ and $\alpha(1) = 1$. For simplicity, we choose
$\pi_0$ also be a Normal distribution. It is trivial to see that all the
intermediate distributions are all Normal. Let $\mu_0$ and $\sigma_0^2$ denote
the mean and variance of $\pi_0$, respectively; and let $\mu_T$ and
$\sigma_T^2$ denote the mean and variance of $\pi_T$, respectively. Then
$\pi_t$ has mean and variance, denoted by $\mu_t$ and $\sigma_t^2$,
\begin{align}
  \mu_t &= \frac{(1 - \alpha_t)\sigma_T^2\mu_0 + \alpha_t\sigma_0^2\mu_T}
  {(1 - \alpha_t)\sigma_T^2 + \alpha_t\sigma_0^2} \label{eq:mini mu_t}\\
  \sigma_t^2 &= \frac{\sigma_0^2\sigma_T^2}
  {(1 - \alpha_t)\sigma_T^2 + \alpha_t\sigma_0^2} \label{eq:mini var_t}
\end{align}
where $\alpha_t = \alpha(t/T)$. Again, for demonstration purpose, at each
iteration, instead of simulating from the Normal distribution directly, we use
a Normal random walk, with a kernel invariant to $\pi_t$, to move the
particles. The (unnormalized) incremental weights are thus,
\begin{equation}
  w_t(X_{t-1}^{(i)}, X_t^{(i)}) =
  \frac{\phi(X_{t-1}^{(i)},\mu_t,\sigma_t^2}
  {\phi(X_{t-1}^{(i)},\mu_{t-1},\sigma_{t-1}^2}
\end{equation}
where $\phi(x,\mu,\sigma^2)$ is the density function of Normal distribution
with mean $\mu$ and variance $\sigma^2$.

\subsection{Implementation}
\label{sub:Implementation}

Headers and definitions,
\begin{cppcode}
#define BASE_INIT    vsmc::InitializeSEQ
#define BASE_MOVE    vsmc::MoveSEQ
#define BASE_MONITOR vsmc::MonitorEvalSEQ

#include <vsmc/core/sampler.hpp>
#include <vsmc/smp/adapter.hpp>
#include <vsmc/smp/state_matrix.hpp>
#include <vsmc/mrw/normal.hpp>
#include <vsmc/smp/backend_seq.hpp>
\end{cppcode}


Classes used in this example,
\begin{cppcode}
using vsmc::Sampler;
using vsmc::Particle;
using vsmc::StateMatrix;
using vsmc::InitializeAdapter;
using vsmc::MonitorEvalAdapter;
using vsmc::SingleParticle;
using vsmc::ConstSingleParticle;
using vsmc::NormalRW;
using vsmc::cxx11::normal_distribution;
\end{cppcode}


The state type,
\begin{cppcode}
typedef StateMatrix<vsmc::RowMajor, 1, double> mini;
\end{cppcode}

The main function,
\begin{cppcode}
int main ()
{
  const size_t N = 1000;
  const size_t T = 100;
  const double mu0 = -5;
  const double sd0 = 10;
  const double muT = 5;
  const double sdT = 1;

  Sampler<mini> sampler(N);

  sampler.init(InitializeAdapter<mini, BASE_INIT>(
        [=] (SingleParticle<mini> sp) -> size_t {
          normal_distribution<> rnorm(mu0, sd0);
          sp.state(0) = rnorm(sp.rng());
          return 1;
        }));

  sampler.move(mini_move(mu0, sd0, muT, sdT, T), false);

  sampler.monitor("moments", 2,
      MonitorEvalAdapter<mini, BASE_MONITOR>(
        [] (size_t, size_t,
            ConstSingleParticle<mini> csp, double *res) {
          res[0] = csp.state(0);
          res[1] = res[0] * res[0];
        }));

  sampler.initialize().iterate(T);
  std::ofstream sampler_file;
  sampler_file.open("mini_sampler");
  sampler_file << sampler << std::endl;

  return 0;
}
\end{cppcode}


Move particles,
\begin{cppcode}
class mini_move : public BASE_MOVE<mini>
{
  public :

  mini_move (
      double mu0, double sd0,
      double muT, double sdT, size_t T) :
    mu0_(mu0), sd0_(sd0), muT_(muT), sdT_(sdT), T_(T),
    var0_(sd0 * sd0), varT_(sdT * sdT) {}

  void pre_processor (size_t, Particle<mini> &particle)
  {
    inc_weight_.resize(particle.size());
  }

  size_t move_state (size_t iter, SingleParticle<mini> sp)
  {
    inc_weight_[sp.id()] =
      (alpha(iter) - alpha(iter - 1)) * (
          normal_log_pdf(sp.state(0), muT_, sdT_) -
          normal_log_pdf(sp.state(0), mu0_, sd0_));

    NormalRW normal_rw;
    double m = mu(iter);
    double s = sd(iter);
    return normal_rw(sp.state(0), s, sp.rng(),
        [=] (double val) {
          return normal_log_pdf(val, m, s);
        });
  }

  void post_processor (size_t, Particle<mini> &particle)
  {
    particle.weight_set().add_log_weight(inc_weight_.begin());
  }

  private :

  double mu0_, sd0_, muT_, sdT_, T_, var0_, varT_;
  std::vector<double> inc_weight_;

  double alpha (size_t t) const;
  double mu (size_t t) const;
  double sd (size_t t) const;
};
\end{cppcode}


\section{Extensions}
\label{sec:Extensions}

\section{Performance benchmark}
\label{sec:Performance benchmark}

