\documentclass[11pt, fontset=Minion, showoverfull, toclevel=2,
bib, biblatexstyle=numeric, mintcode, minted=cache]{marticle}
\addbibresource{thesis.bib}

\title{Sequential Monte Carlo for Bayesian Model Comparison}
\author{Yan Zhou}
\date{Estimated submission date: October 2013}
\let\firstyear\iftrue  \let\endfirstyear\fi
\let\secondyear\iftrue \let\endsecondyear\fi
\let\thirdyear\iffalse \let\endthirdyear\fi

\def\DeclareAbbr#1#2{\def#1{\textsc{#2}\xspace}}
\DeclareAbbr{\ess}{ess}
\DeclareAbbr{\aic}{aic}
\DeclareAbbr{\bic}{bic}
\DeclareAbbr{\dic}{dic}
\DeclareAbbr{\fic}{fic}
\DeclareAbbr{\fpe}{fpe}
\DeclareAbbr{\gic}{gic}
\DeclareAbbr{\mcmc}{mcmc}
\DeclareAbbr{\mdl}{mdl}
\DeclareAbbr{\mle}{mle}
\DeclareAbbr{\nic}{nic}
\DeclareAbbr{\nls}{nls}
\DeclareAbbr{\pet}{pet}
\DeclareAbbr{\rjmcmc}{rjmcmc}
\DeclareAbbr{\smc}{smc}
\DeclareAbbr{\tic}{tic}
\DeclareAbbr{\unix}{unix}

\def\bugs{\textsc{\sffamily bugs}\xspace}
\def\jags{\textsc{\sffamily jags}\xspace}

\def\aicc{\hbox{\aic{}\ensuremath{_c}}\xspace}

\def\kl{Kullback-Leibler divergence\xspace}
\def\mha{Metropolis-Hastings algorithm\xspace}

\def\dkl{D_{\mathrm{KL}}}

\def\bth{{\mathbfit{\theta}}}
\def\bTh{{\mathbfit{\Theta}}}

\def\hbth{\widehat{\bth}}
\def\tbth{\widetilde{\bth}}

\def\xm{(x^1,\dots,x^m)}
\def\Xm{(X^1,\dots,X^m)}
\def\bthm{(\bth^1,\dots,\bth^m)}
\def\bThm{(\bTh^1,\dots,\bTh^m)}
\def\vec#1#2{#1_1,\dots,#1_{#2}}
\def\svec#1#2{#1^1,\dots,#1^{#2}}
\def\mcvec#1#2{#1^{(1)},\dots,#1^{(#2)}}

\def\calGa{\mathcal{Ga}}
\def\calK{\mathcal{K}}
\def\calLN{\mathcal{LN}}
\def\calL{\mathcal{La}}
\def\calM{\mathcal{M}}
\def\calN{\mathcal{N}}

\def\bbN{\mathbb{N}}
\def\bbR{\mathbb{R}}

\def\bfX{\mathbfit{X}}
\def\bfx{\mathbfit{x}}
\def\bfY{\mathbfit{Y}}
\def\bfy{\mathbfit{y}}
\def\bfu{\mathbfit{u}}


\def\finish#1{(Estimated completion date: #1 2013)}

\begin{document}

\maketitle

\section{Introduction \finish{October}}

Model comparison for the purpose of selection, averaging and validation is a
problem found throughout statistics and related areas. It has been studies
since the very early days of statistics. Bayesian model comparison has
attracted substantial interest and has seen extensive applications in the past
three decades. However, for some application areas where a Bayesian approach
is appealing, much difficulties, mainly computationally, still remain.

Bayesian modeling for non-trivial problems are usually solved through
simulation techniques. Markov chain Monte Carlo methods has been the dominant
methods until recent development of population based Monte Carlo algorithms.
Among them sequential Monte Carlo (\smc) and other related algorithms have
attracted considerable interest and development in the past decade. However,
their applications for the purpose of Bayesian model comparison are less well
studied in the literature.

The current work advocates Bayesian model comparison for applications where
such approaches were previously difficult due to computational complexity.
Further, we develop novel sequential Monte Carlo techniques for the purpose of
Bayesian model comparison.

\section{Model selection (Finished)}

In this chapter various model selection techniques were reviewed. First the
approaches based on information criteria, such as \aic and \bic are reviewed
briefly. Though their limitations are well known, they are still widely used
in many areas due to its computational simplicity. It is followed by a review
of the Bayesian comparison methodology.

\subsection{Information criteria based model selection}

Though not directly related to Bayesian model comparison, however these
methods are widely used in some applications, including one detailed in this
thesis later. Therefore a brief review of some non-Bayesian methods are also
reviewed in this section.

\subsection{Bayesian model comparison}

\section{Monte Carlo methods (Finished)}

In this chapter, an extensive literature review of Monte Carlo methods is
provided.

Monte Carlo methods aim to simulate samples from a distribution of interest,
say $\pi$, or provide large sample approximations of some quantities related
to the distributions such as expectations of some $\pi$-integrable functions.

We first review the importance sampling. Though its applications are quite
limited due to well known limitations. It is essential for more advanced
techniques such as sequential Monte Carlo, which will be the central topic of
the current work.

Next various Markov chain Monte Carlo (\mcmc) algorithms are reviewed. Though
it well known that these techniques often cannot explore high dimensional
multimodal distributions efficiently, they are the building blocks of many
advanced Monte Carlo algorithms.

It is followed by a review of the sequential Monte Carlo literature. Some
related methods, including those combining both \smc and \mcmc are also
reviewed in this section.

\subsection{Classical Monte Carlo}

\subsection{Importance sampling}

\subsection{Markov chain Monte Carlo}

\subsubsection{Discrete time Markov chains}

\subsubsection{Metropolis-Hastings algorithm}

\subsubsection{Gibbs sampling}

\subsubsection{Reversible jump Markov chain Monte Carlo}

\subsubsection{Population \mcmc}

\section{Sequential Monte Carlo for Bayesian model comparison (Finished)}

Sequential Monte Carlo methods are a family of algorithms that combines
importance sampling and resampling to allows one to obtains samples from
complex distributions iteratively. These algorithms are well known for its
ability to explore high dimensional multimodal spaces. In this chapter, we
apply them for the purpose of Bayesian model comparison.

First standard techniques are reviewed in the context of Bayesian model
comparison. Next the combination of \smc and path sampling are discussed.

It follows a few novel extensions to existing practices. In particular, some
adaptive methods are developed.

We provide a few illustrative examples, including the \pet compartmental
model in the last chapter. It will be shown that \smc compares favorable to
other state-of-the-art Monte Carlo methods for the purpose of Bayesian model
comparison.

Last, some more practical issues of the implementation of the \smc algorithms
are developed. During a research progress, a C++ library is developed for the
purpose of parallel implementation of \smc algorithms. It is briefly discussed
in the end of this chapter.

\subsection{Methodology}

\subsection{Path sampling with \smc samplers}

\subsection{Extensions}

\subsection{Examples}

\section{vSMC: A C++ library for parallel implementation of \smc and
  related algorithms (Finished)}

A C++ library was developed along with the research. These development
provided tools for implementing the illustrative examples in the next chapter.
They are also tools for more general use. In this chapter, this software
framework is introduced. It will be shown how general and some more
specialized \smc samplers as well as some related algorithms can be
implemented within this framework with little efforts.

\section{Conclusions \finish{October}}

\printbibliography[heading=reference]

\end{document}
