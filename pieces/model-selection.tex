\chapter{Model selection}
\label{cha:Model selection}

Model selection is a problem found through out statistics and related
disciplines. A number of approaches of model selection has been developed
through the history of statistics. Each has its strength and drawbacks. There
is no universally best model selection strategy. As
\cite[][chap.~4]{Myers:1990wt} pointed out, different model selection
strategies have different purposes. Though he made the statement in the
context of regression analysis, it is much true in more general situations.
Selecting a suitable model selection strategy for a particular application is
perhaps a more interesting and difficult problem than any specific method
itself. We discuss more of this through out this chapter.

It is beyond the scope of this thesis to give a comprehensive systematic
review of the topic of model selection. We will review some of the most used
in the practice of modern statistics. Instead of grouping the methods reviewed
according to their origins (e.g., Bayesian theory or information theory), we
structure this chapter as the following. Section~\ref{sec:Consistency and
  efficiency} discusses two aspects of model selection methods in general, the
consistency and efficiency. We consider them to be two more important
properties of a given method. They provide practical guidelines for how to
select a model selection strategy. In the sections that follow, instead of
diving into more technical details of each method, such as rigorous
mathematical presentations of their derivations, we aim to give intuitive
discussions of these two aspects of each method. Section~\ref{sec:Information
  criteria} reviews a few methods that take the form of information criteria,
such as \aic and \bic. In section~\ref{sec:Cross-validation} the
cross-validation method is reviewed. In section~\ref{sec:Minimum description
  length} the minimum description length (\mdl) method is reviewed. The last,
in section~\ref{sec:Bayesian model comparison} the Bayesian framework for the
purpose of model comparison is discussed, while its computation is reviewed in
the next chapter. The rational behind this structure is that, though some
methods have different origins, for instance, \bic can trace its development
back to the Bayesian framework (section~\ref{sec:Bayesian model comparison})
instead of the information theory as \aic, it is more often compared to
\aic and its variants in practice instead of other ``pure'' Bayesian approach,
which are usually based on Monte Carlo approximations of Bayes factor or
posterior model probabilities. In addition, \aic and \bic are frequently
compared and studied together in literature in the past.

In this thesis, we restrict the discussions to parametric models. In addition,
we only consider methods that can be applied in general situations, in
contrast to those that can only be used for certain fields. For example, the
Hannan-Quinn \cite{Hannan:1979us} criteria for autoregression analysis is not
considered though it has been widely used in that specific field.
\sidenote{I think HQ can be used outside autoregression, though I haven't seen
  it yet. This may not be a proper example here}

\section{Consistency and efficiency}
\label{sec:Consistency and efficiency}

Two properties of a model selection method are of particular interest for the
purpose of this thesis, the consistency and efficiency of a given method.

By consistency, we mean a model selector shall pickup the true model if it
exists and is among the candidate models. If we only assume that there is a
true data generating mechanism, which may not be among the candidate models,
then the model selector shall choose the model that is closest to this true
model given enough data. There are different ways of measuring this
``closeness''. The \kl \cite{Kullback:1951va} between the true model and the
chosen model is the most commonly used. When more than one model that minimize
the \kl, then consistency means that the model selection methods shall choose
the model with the least complexity, usually meaning the shortest parameter
vector. Sometime this model is also called the \emph{parsimony} model. More
formal treatment can be found in \cite{Sin:1996vs} and \cite[][sec.~4.1
and~4.3]{Claeskens:2008tq} for the particular case of information criteria
that take the form of penalized likelihood. Note that, without a proper nested
relationship between models, the definition of consistency can be unclear in
general.

Efficiency is a less precisely defined term. The primary goal of statistical
modeling is often not the model selection itself, rather the inference and
prediction using the selected model. By efficiency, we means how well the
chosen model perform for the purpose of estimation and prediction. This is
often measured with mean squared error (\mse) or squared prediction error.
More formally, an efficient information criteria will select a model such that
when the sample size goes to infinity, the minimal \mse (or other criteria)
among all candidate models is reached by the chosen model. It shall be noted
that, the efficiency of model selection methods can only be studied in
context. A method that is efficient for a linear regression is not necessarily
efficient for a time series study.

It is clear that consistency is more important when one does assume that a
true model exits and identifying it is important. Efficiency is more important
if the purpose of model selection is finding the system that approximates the
reality best. It shall not be a surprise that these two factors are often of
conflict. In fact, in \cite{Yang:2005vj}, \sidenote{What Yang really said is
  that the selector is not minmax optimal with expected \mse as the risk. I
  intend not to introduce more rigorous mathematical representation here.
  Otherwise the this chapter needs a whole new level of technical details
  which I don't see particularly useful. I am not sure if my description here
  is close enough to the more formal presentation} it was shown that for
regression models with the form $Y_i = f(x_i) + \varepsilon_i$, under some
rather mild assumptions of the models, any consistent model selection method
cannot be efficient in the sense that the worst mean square error will be
higher than a non-consistent selector that minimize this quantity. Therefore,
depending on the purpose of model selection, ``identifying the true model''
may not be the most important thing to consider.

\section{Information criteria}
\label{sec:Information criteria}

In this section, we review various model selection methods that take the form
of information criteria. Among them, the Akaike's information criterion (\aic)
is perhaps the most widely used, and discussed in section~\ref{sub:Akaike's
  information criterion}. It is followed by discussions of Bayesian
information criterion (\bic). The first three all select a single model as the
best model while \fic has a different approach, which allows different model
to be selected for the purpose of estimating different parameters.

Though these methods have been developed with different background and
focuses. A common theme is that they tend to balance the goodness-of-fit with
complexity of the chosen model. This is most obvious in the cases of \aic and
\bic, which act as penalized log-likelihood criteria. More specifically, for a
given model with parameter vector $\theta$ and density $f(\cdot;\theta)$,
these two information criteria takes the form,
\begin{equation}
  \text{\ic} = -2\ell_n(\hat\theta) + p
  \label{eq:ic}
\end{equation}
where $\ell_n(\theta)$ is the log-likelihood for data $\data =
(y_1,\dots,y_n)^T$ given parameter $\theta$ and $\hat\theta$ is the \mle given
the data $\data$. For \iid samples, $\ell(\theta) = \sum_{i=1}^n\log
f(y_i;\theta)$. There are other model selection methods use criteria with the
same form as above. In \cite{Sin:1996vs}, a throughout treatment of asymptotic
properties of such information criteria can be found.

\subsection{Akaike's information criterion}
\label{sub:Akaike's information criterion}

As we will see later, \aic is built upon the relation between \kl and the
maximum likelihood estimator (\mle). Given data $\data = (y_1,\dots,y_n)^T$,
the general formula of \aic for a candidate parametric model $\calM$, with
parameter vector $\theta$ and density $f(\cdot;\theta)$ is,
\begin{equation}
  \text{\aic} = -2\ell_n(\hat\theta) + 2k
  \label{eq:aic}
\end{equation}
where $\hat\theta$ is the \mle under model $\calM$ and $k$ is the number
parameters estimated, i.e., the length of $\theta$. The model with the
smallest \aic value is selected. It is first introduced by Akaike in a series
of papers \cite{Akaike:1973uc, Akaike:1974ih, Akaike:1977ul}. It can be shown
that \aic is a large sample approximation to the expected \kl
\cite{Akaike:1973uc,Bozdogan:1987wy}. This point will be discussed later in
this section. From a likelihood principle perspective, \aic acts as a balance
between good fit (high log-likelihood) and complexity in the sense that the
dimension of the parameter vector is used as a penalty term in the formula.
The \aic method aims to find models that have fewer parameters while fitting
the data well.

\subsubsection{Maximum likelihood, Kullback-Leibler divergence and
  \protect\aic}
\label{ssub:Maximum likelihood, Kullback-Leiber divergence and aic}

The \aic strategy is not unique in using maximum likelihood as a measure of
goodness-of-fit. However, these methods differs in how the penalize the model
complexity. The \aic method is based on the idea of minimizing the
``distance'' between the chosen model and the true data generating density,
say $g(\cdot)$. There are several methods to measure the difference between
densities. However, the one most directly linked to the maximum likelihood
estimator is the \kl, defined as,
\begin{equation}
  D_{\text{\textsc{kl}}}(g, f(\cdot;\theta))
  = \int g(y)\log\frac{g(y)}{f(y;\theta)}\intd y
  = \int g(y)\log g(y) - \Exp_g[\log f(Y;\theta)],
  \label{eq:kl}
\end{equation}
where the expectation is taken with respect to the true data density $g$,
assuming it exists. For \iid samples, applying the strong law of large
numbers, we have
\begin{equation}
  \frac{1}{n}\ell_n(\theta) \xrightarrow{\text{a.s.}}
  \Exp_g[\log f(Y;\theta)]
\end{equation}
and thus,
\begin{equation}
  \hat\theta \xrightarrow{\text{a.s.}} \tilde\theta
\end{equation}
where $\tilde\theta$ is the value of $\theta$ that minimize
$D_{\text{\textsc{kl}}}(g,f(\cdot;\theta))$ for a given model. The \aic
formula, equation~\eqref{eq:aic} is essentially an approximation to the
expectation of
\begin{equation}
  R_n = \Exp_g[\log f(Y;\hat\theta)],
\end{equation}
say, $Q_n = \Exp_{\hat\theta}[R_n]$, where the outer expectation is with
respect to the distribution of the \mle, $\hat\theta$, under the true data
density $g$, based on the above relation. In other words, \aic studies the
expected \kl of the candidate model relative to the true data generating
density when using the \mle of given data. A naive approximation of $Q_n$ is,
\begin{equation}
  \hat{Q}_n = \frac{1}{n} \sum_{i=1}^nf(y_i;\hat\theta).
\end{equation}
However, it can be shown that this approximation has a positive bias. With a
first order Taylor expansion (see e.g., \cite[][sec.~2.3]{Claeskens:2008tq}),
\begin{equation}
  \Exp[\hat{Q}_n - Q_n] \approx \frac{1}{n}\tr(I^{-1}\Sigma),
  \label{eq:aic Taylor}
\end{equation}
where
\begin{equation}
  I = \Exp_g[I(Y,\tilde\theta)]\quad\text{and}\quad
  \Sigma = \var_g U(Y,\tilde\theta)
  \label{eq:aic bias}
\end{equation}
with
\begin{align}
  I(y,\theta) &=
  -\frac{\partial^2\log f(y;\theta)}{\partial\theta\partial\theta^T}\\
  U(y,\theta) &=
  \frac{\partial\log f(y;\theta)}{\partial\theta},
\end{align}
namely the information matrix and the score function, respectively. In the
special case that $f(y;\theta) = g(y)$, $I$ becomes the Fisher information
matrix and $\tr(I^{-1}\Sigma) = k$. This leads to the \aic
formula~\eqref{eq:aic}, by multiplying both the approximation $\hat{Q}_n$ and
the expected bias by $n$.

The approximation in the expression~\eqref{eq:aic Taylor} has an error of
order $o(n^{-1})$. For small sample situations, this approximation may be too
inaccurate for practical use. A second order approximation which is more
suitable for small sample size is developed as the corrected \aic and will be
discussed later.

The \aic strategy also has obvious issues when $n$ becomes larger. The
log-likelihood function increases linear as $n$ increases while the penalize
term $2k$ is not affected. Therefore, the \aic strategy is going to select
more and more complex models when more data becomes available. This is partly
due to the inadequate of assumption that the approximation model is correct,
that is $f(y;\theta) = g(y)$. Therefore, \aic has a tendency of overfitting
and it is formally shown in \cite{Sin:1996vs} that it is not consistent in the
sense that when more than one models that minimize the \kl to the true data
density are presented, \aic does not necessarily choose the simplest model.

In many studies, it is found that \aic can be more efficient when compared to
other information criteria (such as \bic as we will see later). For example,
in \cite{Lee:2001tm} it was shown that \aic is efficient for selecting the
order of an autoregressive process, with efficiency measured by the mean
squared prediction error. Similar results is also found in \cite{Yang:2005vj}
for regression analysis.

These properties of consistency and efficiency of \aic are also applicable to
the two commonly used refinements discussed below.

\subsubsection{Refinement of \protect\aic}
\label{ssub:Refinement of aic}

Most refinement of \aic resolves around the improvement of the approximation
in~\eqref{eq:aic Taylor}, which uses a first order Taylor expansion, or more
accurate estimation of the bias term in terms of equations~\eqref{eq:aic
  bias}.

\paragraph{Corrected \protect\aic}

As shown in \cite{Sugiura:1978be}, the first order approximation can perform
poorly when the data size is small when compared to the number of parameters
to be estimated. A second order variant is derived in the same paper and
further studied by \cite{Hurvich:1989ev}, which led to a criterion that is
called \aic{}$_c$, the corrected \aic,
\begin{equation}
  \aic_c = -2\ell_n(\hat\theta) + \frac{2nk}{n-k-1}
\end{equation}
where $n$ is the sample size. It is clearly that the additional bias
correction is negligible if $n$ is large when compared to $k$, as
$\lim_{n\to\infty}2nk/(n-k-1) = 2k$, which is exactly the penalty term in the
original \aic formula. A rule of thumb, found in various source, is that
\aic{}$_c$ shall be used in place of \aic when $n/k\le40$, see e.g.,
\cite[][sec.~2.4]{Burnham:2002wc}.

It should be noted that \aic{}$_c$ is just one way to improve \aic for small
samples. In particular, it is derived in the case of a model with linear
structure and normal errors \cite{Hurvich:1989ev, Burnham:2002wc}. Under other
models, other form of improved \aic can be derived. However, this form has
also been adopted successfully in literature even in nonlinear non-Gaussian
cases, for example see \cite{Turkheimer:2003iy}.

\paragraph{Takeuchi's information criterion}

As mentioned before, the \aic's approximation of the bias term is based on the
assumption the candidate model is close enough to the true model. However,
this is often not the case in reality. A model-robust modification, shown by
Takeuchi in 1976 (and commonly termed \tic) uses the data to estimate the two
matrices in equations~\eqref{eq:aic bias}. More specifically, let $\hat{I}_n$
and $\hat\Sigma_n$ be the sample average and sample variance of $I(y,\theta)$
and $U(y,\theta)$, respectively. Then $\tr(\hat{I}_n\hat\Sigma_n)$ is used as
the penalize term instead of the length of the parameter vector.

\subsection{Bayesian information criterion}
\label{sub:Bayesian information criterion}

The Bayesian information criterion (\bic) was developed as an approximation to
the Bayes factor \cite{Schwarz:1978uv}. The \bic is defined as,
\begin{equation}
  \text{\bic} = -2\ell(\hat\theta) + k\log(n).
  \label{eq:bic}
\end{equation}
where $\ell(\hat\theta)$ is the log-likelihood function evaluated at the
\mle; $k$ is the number of parameters to be estimated and $n$ is the number of
observations. The \bic strategy choose the model with the smallest \bic value.
A full treatment of Bayesian model comparison, and Bayes factor in particular
is given in section~\ref{sec:Bayesian model comparison}. For the purpose of
this section, it is sufficient to mention that \bic aims to choose a model
that maximize the posterior probability of a model, say $M$, defined as,
\begin{equation}
  \Pr(M|\data) = \frac{\Pr(M)}{f(\data)}\int
  f(y|M,\theta)\pi(\theta|M)\intd\theta
\end{equation}
where $f(y|M,\data)$ is the likelihood function; $\pi(\theta|M)$ is a prior
distribution, which can be arbitrary yet critically influence the model
selection results; and $f(\data)$ is the unconditional likelihood of the data
and it is irrelevant here since it is a constant for all models. The \bic is
essentially a large sample approximation of the integration term, which is
termed the \emph{marginal likelihood}, as it is also equal to $\pi(\data|M)$.

It is clear from the formulation of \bic that the influence of the prior
distribution $\pi(\theta|M)$ is eliminated in this strategy. This also leads
to some argue that \bic is not a full fledged Bayesian approach. The Bayesian
model comparison approach will be discussed in section~\ref{sec:Bayesian model
  comparison}.

Similar to \aic, the \bic assumes that the sample size is large enough so that
it can approximate the marginal likelihood properly. In addition \bic also
assumes ``good behavior'' in the sense that the \mle is in the high posterior
probability region. As seen in its formulation, \bic eliminates the prior for
large data set. These assumptions restricted the use of \bic in many
situations. For example, see \cite{Berger:2001uy} for examples that the
irregularity of likelihood function causes the \bic method failed. Moreover,
in non-\iid situations, the definition of the parameter dimension $k$ is
ambiguous \cite{Spiegelhalter:1998uc, Kass:1995vb}. There are other criticism
of the \bic strategy. For example \cite[][sec.~7.2.3]{Robert:2007tc} argued
that the \bic strategy eliminated the subjective input into the Bayes modeling
since the value of \bic does not depend on the prior distribution. However
this is equally argued as an advantage of this strategy in the case that
priors, to which Bayes factors can be very sensitive, are hard to specify.

Though the original development of \bic \cite{Schwarz:1978uv} does not really
involves the consideration of \kl as \aic did. However, from an information
theory perspective, \bic enjoys greater consistency than \aic in many
applications. It was shown in \cite{Yang:2005vj} that, asymptotically (in the
sense that the sample size goes to infinity), \bic is able to choose the
simplest model that minimizing \kl between the true model and the candidate
model for regression analysis. In \cite{Sin:1996vs}, a more general result can
be found.

However, \bic also has its drawback. It may not be as efficient as \aic for
some applications. For example, in \cite{Lee2001} it was shown that for
autogressive process and some other time series applications, \bic is not
efficient in the sense that \bic may not choose the model that minimize the
prediction error. In \cite[][sec.~4.7]{Claeskens:2008tq}, it was also shown
that \bic is not efficient for regression variable selection.

\sidenote{Cross-validation still lacks some depth}

\section{Cross-validation}
\label{sec:Cross-validation}

Cross-validation is a model selection strategy that has long been practiced by
statisticians, especially for predictive studies. It was formalized by
\cite{Stone1974} and \cite{Geisser1975}. The idea is that the data is spitted
into two parts. The majority of the data are used for model fitting. And the
rest, called the \emph{left-out} data are predicted or estimated with
the fitted model and compared with the observations. The model that estimate
the left-out data best is the chosen model.

The most common implementation of this scheme is to left one data at a time.
Candidate models are fit with all but one data, and are then used to predict
or estimate the case that is left out. There are many ways of comparing the
models. One is related to \kl. Recall equation~\eqref{eq:kl}, where the term
$\Exp_g[\log f(Y;\theta)]$ is the one that dependent on the density $f$ and
hence the model. One way of approximate the quantity is,
\begin{equation}
  \frac{1}{n}\sum_{i=1}^n \log f(y_i;\hat\theta_i)
\end{equation}
where $\hat\theta_i$ is the \mle estimated with the data $y_i$ are left out
from the data $\data = (y_1,\dots,y_n)$. Then the model with the highest
value of the above quantity is chosen.

Intuitively, it measures how well the model fitted without $y_i$ predict or
estimate $Y_i$ using an approximation of (relative) \kl. The ``better'' model
has higher likelihood for each data with parameter set to the one estimated
without it. There are other ways of choosing model using the cross-validation
scheme. For example, the \tic approximation of \kl can be used instead of the
above.

\section{Bayesian model comparison}
\label{sec:Bayesian model comparison}

Bayes' theorem, in its simplest form is stated as below,
\begin{equation}
  p(H|\data) = \frac{p(\data|H)p(H)}{p(\data)} \label{eq:bayes}
\end{equation}
where $\data$ is the data and $H$ is a hypothesis. Like many other probability
theories, technically Bayes' theorem merely provides a method of accounting
for the uncertainty. There are different interpretations, rooted in the views
of probabilities. Two popular views. One is the \emph{frequentist} (or
\emph{statistical}) view among other so called \emph{objective} views. The
other is \emph{subjective} view. The frequentist view an \emph{event} as a
class of \emph{individual events} which are not only equally probable and
\emph{stochastically independent}. With a subjective view, the only thing
that matters is uncertainty and how to make informed and rational discussions
under uncertainty. Some, mostly from an objective perspective, has argued that
Bayesian modeling merely provides a convenient formulation in some situations.
From a subjective perspective, Bayesian statistics provides a rational way to
update one's \emph{personal belief} and to make rational decisions. There has
been debate over views of probability as long as statistics has been a
discipline. These topics, though will not be focused on in this thesis, are
important in both theory and practice. For example, it can affects the choice
of priors.

A throughout treatment of Bayesian modeling from a decision theory perspective
can be found in \cite{Robert:2007tc}. Formal mathematical representations can
also be found in \cite[][sec.~5.1 and sec.~6.1]{Bernardo:1994vd}. Notion of
rational decisions in the context of uncertainty was also made precise in the
form of axioms in \cite{DeFinetti:1974tg,DeFinetti:1975ua}. It is assumed that
a rational decision cannot be considered separately from rational beliefs. And
rational beliefs shall be built upon available information (the data in the
statistical term) and any personal preference input (the prior information in
Bayesian literatures).

\subsection{Mathematical foundations}
\label{sub:Mathematical foundations}

Consider a (possibly infinite) countable set of parametric models $\calM =
\{\calM_k\}_{k\in\calK}$. Under each model, the data is generated according to
a likelihood function $f(\data|\theta_k,\calM_k)$ where $\theta_k$ is the
parameter vector in the space $\Theta_k\subset\Real^{d_k}$, and $d_k$ is the
dimension. Within the Bayesian framework, a prior distribution is chosen for
these parameters conditional on the model, say $\pi(\theta_k|\calM_k)$. And
each model itself has a prior distribution $\pi(\calM_k)$. Therefore according
to Bayes' theorem, the posterior distribution of the parameters and the model,
conditional upon the data is given by the following density, defined on the
space $\bigcup_{k\in\calK}\{\calM_k\}\times\Theta_k$,
\begin{equation}
  \pi(\theta_k,\calM_k|\data) =
  \frac{f(\data|\theta_k,\calM_k)\pi(\theta_k|\calM_k)\pi(\calM_k)}{p(\data)}
  \label{eq:full posterior}
\end{equation}
where
\begin{align}
  p(\data) &= \sum_{k\in\calK}p(\data|\calM_k)\pi(\calM_k) \\
  p(\data|\calM_k) &=
  \int f(\data|\theta_k,\calM_k)\pi(\theta_k|\calM_k)\intd\theta_k
  \label{eq:marginal likelihood}
\end{align}
This is sometimes also termed the \emph{full posterior}. The within model
posterior distribution of the parameter is,
\begin{equation}
  \pi(\theta_k|\data,\calM_k) =
  \frac{f(\data|\theta_k,\calM_k)\pi(\theta_k|\calM_k)}{p(\data|\calM_k)}
  \label{eq:within posterior}
\end{equation}
The term $p(\data|\calM_k)$ is termed the \emph{marginal likelihood} or the
\emph{evidence} of the model.

From equation~\eqref{eq:full posterior}, it is clear that the model posterior
probability $\pi(\calM_k|\data)$ is a marginal of the full posterior,
\begin{equation}
  \pi(\calM_k|\data) =
  \frac{
    \pi(\calM_k)
    \int f(\data|\theta_k,\calM_k)\pi(\theta_k|\calM_k)\intd\theta_k
  }{
    \sum_{l\in\calK}\pi(\calM_l)
    \int f(\data|\theta_l,\calM_l)\pi(\theta_l|\calM_l)\intd\theta_l
  }.
  \label{eq:post model prob}
\end{equation}
Bayesian model choice problem mostly centers around the inference of this
posterior model probability. Many methods for computing this probability is
reviewed in chapter~\ref{cha:Monte Carlo Methods}. In the remaining of this
section, we assume that the calculation of required quantities will be
available.

Our aim is to choose the ``best'' model from the set $\calM$. There will
usually be actions taken after the model is chosen, for example inference of
the past or prediction of the future. It is the consequences of these actions
of interest instead of the chosen model itself. Therefore, from a
decision-theoretic perspective, the ``best'' model should maximize the utility
for some quality of interest. However, in practice it is common to ignore the
actions following the model selection and the sole interest is the true model,
say $\calM_t$. In this case, it is natural to define a zero-one utility
function, say $u(\calM_k,\calM_t)$,
\begin{equation}
  u(\calM_k,\calM_t) =
  \begin{cases}
    0, &\text{if } \calM_k = \calM_t,\\
    1  &\text{otherwise.}
  \end{cases}
\end{equation}
It is easy to see that the model $\calM_k$ that maximizes the expected utility
given data $\data$ is the model with the highest posterior probability
$\pi(\calM_k|\data)$, see \cite[][chap.~6]{Bernardo:1994vd}. Also see
\cite[][sec.~7.2.1]{Robert:2007tc} for an in depth discussion of the
difficulties of the Bayesian formulation in the model choice problem and the
reason of why such a maximum posterior probability approach is adapted. It
should be noted the above argument is only valid if the true model $\calM_t
\in \calM$. Otherwise, the utility is always zero for all models. In what
follows, we presume that our aim is to find the model with the highest
posterior probability.

It should be noted that the prior distribution $\pi(\theta_k|\calM_k)$ is
chosen by statisticians in the modeling process. In chapter~\ref{cha:Bayesian
  Model Comparison for Positron Emission Tomography Compartmental Model} we
will see an example of construction of well-informed priors for a realistic
application. The choice of the prior distribution is one of the most critical
and criticized part of Bayesian modeling. In principle, the prior distribution
shall represent the prior beliefs. That is, it represents how much is already
known about the data generating mechanism and what is believed about it before
the observation of the data, no more or no less. It shall not only describe
all knowledge already known, but also more importantly preserve all the
ignorance. If it contains more information than what is actually known, the
inference will be biased. In other words, one can always construct a prior
distribution such that inference will be biased towards one's preference,
which is not necessarily rational. The priors need to be considered carefully
on a per problem basis. Nonetheless there are a few classes of prior
distributions frequently used in the Bayesian modeling. It is often too
difficult to elicit a precise distribution from prior information. Therefore
it is necessary to make at least partially arbitrary choice of the prior
distribution \cite[][chap.~3]{Robert:2007tc}\cite{Kass:1995vb}.

One of the more important class of prior distribution is conjugate priors. A
conjugate prior, say $\pi(\theta_k|\calM_k)$, for a parametric model with
likelihood $f(\data|\theta_k,\calM_k)$, is one such that the posterior
$\pi(\theta_k|\data,\calM_K)$ belongs to the same family of distributions as
the prior. In \cite[][chap.~5]{Bernardo:1994vd} it was argued that the
conjugate prior reduced the input of prior information to only the choice of
parameter values and thus cannot be fully justified from a subjective
perspective. Though their mathematical simplicity make them attractive in
practice, for many applications of interest it is near impossible to find such
priors.

In situations where no or little prior information are available, the
so-called ``non-informative'' priors are often used. One choice is the
Jeffrey's priors \cite{Jeffreys:1946jf}, which has the form
\begin{equation}
  \pi(\theta_k|\calM_k)\propto\sqrt{I(\theta_k)}
\end{equation}
where
\begin{equation}
  I(\theta_k) = \int f(\data|\theta_k,\calM_k)\Round[Big]{
    -\frac{\partial^2}{\partial\theta_k^2}\log f(\data|\theta_k,\calM_k)}
  \intd\data,
\end{equation}
is the \emph{Fisher's information}. The basic idea behind Jeffrey's priors is
that the prior shall contain no more information than the observed data do.
Though intuitive, Jeffrey's priors derived for many models may be problemtic.
It is often the case that the derived Jeffrey's prior can be improper, which
makes the Bayesian model choice problem difficult to formulalize as there is
no unique scale for improper priors.

Another class non-informative priors is called \emph{reference priors}
introduced in \cite{Bernardo:1979uq}. Reference priors aims to derive priors
such that the distance between the posteriror and prior is maximized, usually
measured in terms of \kl \cite{Kullback:1951va}. In some sense, a reference
prior is the least informative prior. See \cite{Berger:1989vj, Berger:1992kf,
  Berger:1992wo} and \cite[][sec.~5.4]{Bernardo:1994vd} for more information
on this class of priors.

\subsection{Bayes factor}
\label{sub:Bayes factor}

When the model set $\calM$ is finite, we can find the model with the highest
posterior probability by compare models pairwise. To compare the posterior
probabilities of two models, say $\calM_{k_1}$ and $\calM_{k_2}$, one only
need to compute their ratio. Recall the expression~\eqref{eq:post model prob},
the ratio can be written as,
\begin{equation}
  \frac{\pi(\calM_{k_1}|\data)}{\pi(\calM_{k_2}|\data)}
  = \frac{\pi(\calM_{k_1})}{\pi(\calM_{k_2})} \frac
  {\int f(\data|\theta_{k_1},\calM_{k_1})\pi(\theta_{k_1}|\calM_{k_1})
      \intd\theta_k}
  {\int f(\data|\theta_{k_2},\calM_{k_2})\pi(\theta_{k_2}|\calM_{k_2})
      \intd\theta_k}
  = \frac{\pi(\calM_{k_1})}{\pi(\calM_{k_2})}B_{12},
  \label{eq:posterior odd}
\end{equation}
where
\begin{equation}
  B_{12} = \frac
  {\int f(\data|\theta_{k_1},\calM_{k_1})\pi(\theta_{k_1}|\calM_{k_1})
      \intd\theta_k}
  {\int f(\data|\theta_{k_2},\calM_{k_2})\pi(\theta_{k_2}|\calM_{k_2})
      \intd\theta_k}
    = \frac{p(\data|\calM_{k_1})}{p(\data|\calM_{k_2})}
  \label{eq:bayes factor}
\end{equation}
is called the Bayes factor. Equation~\eqref{eq:posterior odd} states how the
prior odds ratio is transformed into the posterior odds ratio by the Bayes
factor \cite{Kass:1995vb}. The Bayes factor is the principle tool for
Bayesian model comparison and model selection. As it is made clear in the
above equation, to compute the Bayes factor, all that needs to be done is the
computation of the marginal likelihood for each model $\calM_k\in\calM$,
\begin{equation}
  p(\data|\calM_k) =
  \int f(\data|\theta_k,\calM_k)\pi(\theta_k|k)\intd\theta_k.
\end{equation}
It is obvious that $B_{ij} = B_{ik} B_{kj}$, and thus the Bayes
factor approach is equivalent to choosing the model with the highest marginal
likelihood, as long as the model set $\calM$ is finite.

Though intuitively the influence of priors can be eliminated given enough
data. One formal assertion will be seen later in the derivation of Bayesian
information criterion (section~\ref{ssub:Bayesian information criterion}). In
the particular problem of Bayesian model comparison and selection, it shall be
noted that Bayesian model comparison is more sensitive to the choice of prior
than point or interval estimations in the sense that more data are need to
eliminate the influence of priors when it comes to the value of Bayes factor
than a posterior interval \cite{Kass:1993vy, Kass:1995vb}.

It is also important to evaluate the Bayes factor over a range of possible
priors to assess the sensitivity issues. This is often computational expensive
since many high dimensional integrations are required. When there is enough
information to construct parametric priors, it is possible to alter the values
of parameters and recompute the Bayes factor \cite{McCulloch:1991hj}. In
general situations, a less computational expensive method is to use the
Laplace approximation to compare the Bayes factor using different priors
\cite{Kass:1992tz}. It is also proposed in the literature to use the maximum
of Bayes factor (and thus the maximal evidence against a model) to evaluate
the sensitivity problem \cite{Berger:1987iq}.
\sidenote{They did say that, on top of pp.~782. Kass 1995. I added a half
  sentence to make the idea more clearly}

The calculation of the Bayes factor can be made exact using analytical results
only occasionally. In most applications of interest, approximations has to be
used. Two approaches are widely used. On is using Monte Carlo approximations.
Another is based on the asymptotic behavior of the Bayes factor. A few of the
later is reviewed in the remaining of this section.
