\documentclass[11pt, bib, fontset = Lucida]{marticle}

\def\mse{\textsc{mse}\xspace}
\def\pet{\textsc{pet}\xspace}
\def\mcmc{\textsc{mcmc}\xspace}
\def\pmcmc{\textsc{pmcmc}\xspace}
\def\rjmcmc{\textsc{rjmcmc}\xspace}
\def\rjsmc{\textsc{rjsmc}\xspace}
\def\smc{\textsc{smc}\xspace}
\def\by{\mathbfit{y}}
\def\bu{\mathbfit{u}}
\def\K{\mathcal{K}}
\def\kmax{k_{\mathrm{max}}}

\addbibresource{library.bib}

\begin{document}
\title{RJMCMC and trans-dimensional SMC implementation}
\author{Yan Zhou}
\maketitle

\section{Algorithm and model}

The trans-dimensional \smc algorithm, (\rjsmc for short since it involves a
reversible jump as well), is different from the standard \smc by the posterior
distribution and the type of \mcmc moves between iterations. The posterior is
now
\begin{equation}
  \pi(\bftheta,k|\by) \propto p(k)\pi(\bftheta|k)f(\by|\bftheta,k)
\end{equation}
which is defined on the joint space $\cup_{k\in\K}(\bfTheta_k\times\{k\})$,
where $\K$ is the model space and $k$ is the model indicator. In this
particular example, the space $\K$ is defined as a set of integers
$\{1,\dots,\kmax\}$ and the prior of $k$ is uniform on this set. The value of
$\kmax$ is set to $100$ in the implementation. Another choice of the prior of
$k$ is a Poisson distribution. The moves between iterations are constructed
with one local move, which does not change the dimension and is the same as in
the standard \smc; and two reversible jump moves. The first is a split or
combine move. The second is a birth or death move.

The \rjmcmc algorithm can be viewed as a special case of \rjsmc. That is,
without resampling, and every intermediate distribution is the posterior, then
a \rjsmc system with $n$ particles can be viewed as $n$ independent \rjmcmc
chains. Therefore we will not make distinction between these two algorithms in
the following discussion, except for those places that such a distinction is
important.

The model is the same as in \textcite{DelMoral:2006hc}, and therefore is
slightly different from \textcite{Richardson:1997ea} in the sense that there
is no allocation parameter involved (in other words it is integrated out from
\textcite{Richardson:1997ea}'s model) and there is no prior for the scale
parameter of the Gamma prior for the precision parameter. This setting
simplifies things considerably.

As in \textcite{Richardson:1997ea} the components are labeled such that the
means of components are strictly increasing. That is, for components $j_1 <
j_2$, we have $\mu_{j_1} < \mu_{j_2}$.

\subsection{Formulation of algorithms}

To summary the above discussion, we give the explicit formulas of the
algorithm in \textcite{Richardson:1997ea} with adjustment to the
parameterization without allocation parameters. As before, we denote the
likelihood by $L(\by|\bftheta,k)$, the prior without the adjacency condition
by $\pi(\bftheta|k)$ and prior for $k$ by $p(k)$ (since it is uniform we will
omit this one henceforth). The first two are exactly the same as in standard
\smc. With the adjacency condition, the prior is thus $k!\pi(\bftheta|k)$.
The parameter $\bftheta = \{w_j, \mu_j, \lambda_j\}_{j=1}^k$, and all weights,
means and precisions share common priors. That is, the weights $\{w_j\}$
follow a symmetric Dirichlet distribution with parameter $\delta = 1$; the
means $\{\mu_j\}$ follow a normal distribution with mean $\eta$ and variance
$\sigma^2$; the precisions $\{\lambda_j\}$ follow a Gamma distribution with
shape $\kappa$ and scale $\chi$. The prior parameters are estimated from the
data in the same way as in \textcite{DelMoral:2006hc}.

\subsubsection{Local moves}

The local move is a three-blocks Metropolis-Hastings random walk. Assume that
the particle is at state $\bftheta$ with $k$ components. First the means
$\{\mu_j\}$ are updated with a normal random walk. Second, the precisions
$\{\lambda_j\}$ are updated with a log-normal random walk, or in other words a
normal random walk on the logarithm scale. Last, the weights $\{w_j\}$ are
updated with a normal random walk on logit scale.

\subsubsection{Split and combine moves}

Again, assume the particle is at state $\bftheta$ with $k$ components, that
is, $\bftheta = \{w_j, \mu_j, \lambda_j\}_{j=1}^k$. We first chose split with
probability $b_k$, and thus combine with probability $d_k = 1 - b_k$.  We set
$b_1 = 1$, $b_{\kmax} = 0$ and $b_k = 0.5$ for $1 < k < \kmax$. If a combine
move is chosen, then we randomly select two adjacent components, say $i_1 = i$
and $i_2 = i + 1$. We propose to combine them into a new component $i$ and
move to the new state $\bftheta^*$ by the following transformation,
\begin{align*}
  (w_j^*, \mu_j^*, \lambda_j^*) &= (w_j, \mu_j, \lambda_j)
  && \quad\text{for }j < i, \\
  (w_j^*, \mu_j^*, \lambda_j^*) &= (w_{j+1}, \mu_{j+1}, \lambda_{j+1})
  && \quad\text{for }j > i,
\end{align*}
and
\begin{equation}
  \left.
  \begin{aligned}
    w_i^* &= w_{i_1} + w_{i_2} \\
    w_i^*\mu_i^* &= w_{i_1}\mu_{i_1} + w_{i_2}\mu_{i_2} \\
    w_i^*({\mu_i^*}^2 + {\lambda_i^*}^{-1}) &=
    w_{i_1}(\mu_{i_1}^2 + \lambda_{i_1}^{-1}) +
    w_{i_2}(\mu_{i_2}^2 + \lambda_{i_2}^{-1})
  \end{aligned}
  \qquad\right\}\label{eq:combine_trans}
\end{equation}
If a split move is chosen, then we randomly select one component, say
component $i$, and propose to split it into two component labeled with $i_1 =
i$ and $i_2 = i + 1$, and move the state $\bftheta$ to the new state
$\bftheta^*$,
\begin{align*}
  (w_j^*, \mu_j^*, \lambda_j^*) &= (w_j, \mu_j, \lambda_j)
  && \quad\text{for }j < i, \\
  (w_j^*, \mu_j^*, \lambda_j^*) &= (w_{j-1}, \mu_{j-1}, \lambda_{j-1})
  && \quad\text{for }j > i + 1,
\end{align*}
and
\begin{equation}
  \left.
  \begin{aligned}
    w_{i_1}^* &= u_1w_i & w_{i_2}^* &= (1 - u_1)w_i \\
    \mu_{i_1}^* &=
    \mu_i - u_2\sqrt{\frac{w_{i_2}^*}{w_{i_1}^*\lambda_i}} &
    \mu_{i_2}^* &=
    \mu_i + u_2\sqrt{\frac{w_{i_1}^*}{w_{i_2}^*\lambda_i}} \\
    \frac{1}{\lambda_{i_1}^*} &=
    u_3(1 - u_2^2)\frac{w_i}{w_{i_1}^*}\frac{1}{\lambda_i} &
    \frac{1}{\lambda_{i_2}^*} &=
    (1 - u_3)(1 - u_2^2)\frac{w_i}{w_{i_2}^*}\frac{1}{\lambda_i}\\
  \end{aligned}
  \qquad\right\}\label{eq:split_trans}
\end{equation}
where $u_1\sim\mathrm{Beta}(2,2)$, $u_2\sim\mathrm{Beta}(2,2)$ and
$u_2\sim\mathrm{Beta}(1,1)$ and $\mathrm{Beta}(a,b)$ is the Beta distribution.
It is easy to check that the split transformation~\eqref{eq:split_trans}
satisfies the relations in the combine transformation~\eqref{eq:combine_trans}
with suitable substitution of notations.

In the split move, if $\mu_{i_1}^* < \mu_{i - 1}$ or $\mu_{i_2}^* > \mu_{i +
1}$, i.e., the adjacency condition is not met, the proposal is rejected.
Otherwise, it is accepted with probability $\min\{1, A_s\}$, where
\begin{align}
  A_s =& \frac{(k + 1)!}{k!}
  \frac{\pi(\bftheta^*|k + 1)L(\by|\bftheta^*,k + 1)}
  {\pi(\bftheta|k)L(\by|\bftheta,k)} \notag\\
  &\times\frac{d_{k+1}}{b_k}
  \frac{1}{f_{2,2}(u_1)f_{2,2}(u_2)f_{1,1}(u_3)} \notag\\
  &\times\frac{w_i\Abs{\mu_{i_1}^* - \mu_{i_2}^*}}
  {u_2(1 - u_2^2)u_3(1 - u_3)}
  \frac{\lambda_{i_1}^*\lambda_{i_2}^*}{\lambda_i}
\end{align}
where $f_{a,b}(\cdot)$ is the density of $\mathrm{Beta}(a,b)$ distribution.
The term $d_{k+1}/b_k$ is the odd of propose a combine move from $\bftheta^*$
back to $\bftheta$ to the split move. The last line is the Jacobian of the
transformation from $(\bftheta,\bu)$ to $\bftheta^*$ with $\bu = (u_1, u_2,
u_3)$. In the combine move, we accept the proposed new state with probability
$\min\{1,A_s^{-1}\}$ with suitable substitutions of notations.

\subsubsection{Birth and death moves}

As before suppose the particle is at state $\bftheta$ with $k$ components.  We
choose birth or death with the same probabilities $b_k$ and $d_k$ as in the
split and combine moves. If a birth move is chosen, then we generate a new
component with
\begin{equation}
  w_{k+1}^* \sim \mathrm{Beta}(1, k),\qquad
  \mu_{k+1}^* \sim \mathrm{Normal}(\eta,\sigma^2),\qquad
  \lambda_{k+1}^* \sim \mathrm{Gamma}(\kappa, \chi).
\end{equation}
To make space for the new component we rescale the old weights $\{w_j\}$ to
$\{w_j^*\}$ with $w_j^* = w_j(1 - w_{k+1}^*)$. The last, we rearrange the
components to met the adjacency condition and obtain the new state
$\bftheta^*$. When a death move is chosen, we randomly select a component and
propose to delete it. The accept probability of accepting the birth move is
$\min\{1,A_b\}$ with
\begin{align}
  A_s =& \frac{(k + 1)!}{k!}
  \frac{\pi(\bftheta^*|k + 1)L(\by|\bftheta^*,k + 1)}
  {\pi(\bftheta|k)L(\by|\bftheta,k)} \notag\\
  &\times\frac{d_{k+1}}{b_k}
  \frac{1}{f_{1,k}(w_{k+1}^*)}(1-w_{k+1}^*)^k.
\end{align}
And the probability of accepting the death move is thus $\min\{1,A_b^{-1}\}$
with suitable substitutions of notations.

\printbibliography[heading = reference]

\end{document}
