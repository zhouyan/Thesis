\documentclass[11pt, linespread = 1.1, lines=40, lineratio = 30, bib,
fontset = Minion]{marticle}

\def\mse{\textsc{mse}\xspace}
\def\pet{\textsc{pet}\xspace}
\def\mcmc{\textsc{mcmc}\xspace}
\def\pmcmc{\textsc{pmcmc}\xspace}
\def\rjmcmc{\textsc{rjmcmc}\xspace}
\def\rjsmc{\textsc{rjsmc}\xspace}
\def\smc{\textsc{smc}\xspace}
\def\btheta{\mathbfit{\theta}}
\def\bTheta{\mathbfit{\Theta}}
\def\by{\mathbfit{y}}
\def\bu{\mathbfit{u}}
\def\K{\mathcal{K}}
\def\kmax{k_{\mathrm{max}}}

\addbibresource{library.bib}

\begin{document}
  \title{RJMCMC and trans-dimensional SMC implementation}
  \author{Yan Zhou}
  \maketitle

  \section{Algorithm and model}

  The trans-dimensional \smc algorithm, (\rjsmc for short since it involves a
  reversible jump as well), is different from the standard \smc by the
  posterior distribution and the type of \mcmc moves between iterations. The
  posterior is now
  \begin{equation}
    \pi(\btheta,k|\by) \propto p(k)\pi(\btheta|k)f(\by|\btheta,k)
  \end{equation}
  which is defined on the joint space $\cup_{k\in\K}\bTheta_k\times\K$, where
  $\K$ is the model space and $k$ is the model indicator. The space of model
  $\K$ is defined as a set of integers $\{1,\dots,\kmax\}$ and the prior of
  $k$ is uniform on this set. The value of $\kmax$ is set to $100$ in the
  implementation. Another choice of the prior on $k$ is Poisson distribution.
  The moves between iterations are constructed with one local move, which does
  not change the dimension and two reversible jump moves. The first is a split
  or combine move. The second is a birth or death move.

  The \rjmcmc algorithm can be viewed as a special case of \rjsmc. That is,
  without resampling, and every intermediate distribution is the posterior,
  then a \rjsmc system with $n$ particles can be viewed as $n$ independent
  \rjmcmc chains. Therefore we will not make distinction between these two
  algorithms in the following discussion, except for those places that such
  distinction is important.

  The model is the same as in \textcite{DelMoral:2006hc}, and therefore is
  slightly different from \textcite{Richardson:1997ea} in the sense that there
  is no allocation parameter involved (in other words it is integrated out
  from \textcite{Richardson:1997ea}'s model) and there is no prior for the
  scale parameter of the Gamma prior for the precision parameter of means.
  This setting simplifies things considerably.

  The model in \textcite{Richardson:1997ea} restricted the labeling of
  components such that the means of components are strictly increasing. That
  is, for components $j_1 < j_2$, we have $\mu_{j_1} < \mu_{j_2}$. This is
  similar to what we did for the \pet model, though with an identical prior
  for each component, they can use an order statistic to implement such
  restriction. However, in the \smc literature, for example both
  \textcite{DelMoral:2006hc} and \textcite{Jasra:2007in}, such restriction was
  not implied. In fact, they considered the identifiability problem as one of
  the benchmarks that measured the algorithms' ability to explore the whole
  space efficiently. For the start, the model with adjacency condition as in
  \textcite{Richardson:1997ea} was implemented. This difference in modeling
  will change the prior by a factor $k!$. But it also changes the parameter
  space, as now the marginal likelihood is the integration over one of the
  $1/k!$ identical spaces, thus the estimation of Bayes factor shall be
  unchanged compared to the unordered modeling. A simple interpretation is
  that labeling the components with an adjacency condition does not make one
  model preferable compared to another.

  The difference in parameterization implies that, in addition to a slightly
  different local move, the birth death move is also slightly different. With
  the allocation parameter, there are the ``empty components'' from which no
  data comes and the death move is only proposed for such component since the
  new born component in the birth move will be a such one. In our case, we
  don't formula the likelihood as conditional on the allocation parameters,
  therefore the death move is proposed for any random chosen component.

  In the next section, we will attempt to implement the reversible jump for
  the unordered case. It will not be very different but it seems there are
  some issues to address in order to maintain the reversibility.

  \subsection{Summary and formulation}

  To summary the above discussion, we give the explicit formulas of the
  algorithm in \textcite{Richardson:1997ea} with adjustment to the
  parameterization without allocation parameters. As before, we denote the
  likelihood by $L(\by|\btheta,k)$, the prior without the adjacency condition
  by $\pi(\btheta|k)$ and prior for $k$ by $p(k)$ (since it is uniform we will
  omit this one henceforth). The first two are exactly the same as in standard
  \smc. With the adjacency condition, the prior is thus $k!\pi(\btheta|k)$.
  The parameter $\btheta = \{w_j, \mu_j, \lambda_j\}_{j=1}^k$, and all
  weights, means and precisions share common priors. That is, the weights
  $\{w_j\}$ follow a symmetric Dirichlet distribution with parameter $\delta =
  1$; the means $\{\mu_j\}$ follow a normal distribution with mean $\eta$ and
  variance $\sigma^2$; the precisions $\{\lambda_j\}$ follow a Gamma
  distribution with shape $\kappa$ and scale $\chi$ (that is mean
  $\kappa\chi$. I always found that half authors use scale and the other half
  use rate, but more than often both of them use the same alphabets, usually
  $\alpha$ and $\beta$, which often gets tings confused. I will try to stick
  with the scale parameter so the text and code are always the same thing and
  hope I didn't use another notation before. Also we still reserve the letter
  $\alpha$ for the power of likelihood function)

  \subsubsection{The split and combine move}

  Say at state $\btheta$ with number of components $k$, we first chose split
  with probability $b_k$, and thus combine with probability $d_k = 1 - b_k$.
  We set $b_1 = 1$, $b_{\kmax} = 0$ and $b_k = 0.5$ for $1 < k < \kmax$. If a
  combine move is chosen, then we randomly select two adjacent components, say
  $i_1 = i$ and $i_2 = i + 1$. We propose to combine them into a new
  component $i$ and move to the new state $\theta'$,
  \begin{align*}
    (w\sprime{'}_j, \mu\sprime{'}_j, \lambda\sprime{'}_j) &=
    (w_j,     \mu_j,     \lambda_j) && \quad\text{for }j < i, \\
    %
    (w\sprime{'}_j, \mu\sprime{'}_j, \lambda\sprime{'}_j) &=
    (w_{j+1}, \mu_{j+1}, \lambda_{j+1}) && \quad\text{for }j > i,
  \end{align*}
  and
  \begin{equation}
    \left.\begin{aligned}
      w\sprime{'}_i &= w_{i_1} + w_{i_2} \\
      %
      w\sprime{'}_i \mu\sprime{'}_i &= w_{i_1}\mu_{i_1} + w_{i_2}\mu_{i_2} \\
      %
      w\sprime{'}_i({\mu\sprime{'}_i}^2 + {\lambda\sprime{'}_i}^{-1}) &=
      w_{i_1}(\mu_{i_1}^2 + \lambda_{i_1}^{-1}) +
      w_{i_2}(\mu_{i_2}^2 + \lambda_{i_2}^{-1})
    \end{aligned}\qquad\right\}\label{eq:combine_trans}
  \end{equation}
  If a split move is chosen, then we randomly select one component, say
  component $i$, and propose to split it into two component labeled with $i_1
  = i$ and $i_2 = i + 1$, and move the state $\theta$ to the new state
  $\theta'$,
  \begin{align*}
    (w\sprime{'}_j, \mu\sprime{'}_j, \lambda\sprime{'}_j) &=
    (w_j, \mu_j, \lambda_j) && \quad\text{for }j < i, \\
    %
    (w\sprime{'}_j, \mu\sprime{'}_j, \lambda\sprime{'}_j) &=
    (w_{j-1}, \mu_{j-1}, \lambda_{j-1}) && \quad\text{for }j > i + 1,
  \end{align*}
  and
  \begin{equation}
    \left.\begin{aligned}
      w\sprime{'}_{i_1}   &= u_1w_i &
      w\sprime{'}_{i_2}   &= (1 - u_1)w_i \\
      %
      \mu\sprime{'}_{i_1} &=
      \mu_i - u_2\sqrt{\frac{w\sprime{'}_{i_2}}{w\sprime{'}_{i_1}\lambda_i}} &
      \mu\sprime{'}_{i_2} &=
      \mu_i + u_2\sqrt{\frac{w\sprime{'}_{i_1}}{w\sprime{'}_{i_2}\lambda_i}} \\
      %
      \frac{1}{\lambda\sprime{'}_{i_1}} &=
      u_3(1 - u_2^2)\frac{w_i}{w\sprime{'}_{i_1}}\frac{1}{\lambda_i} &
      \frac{1}{\lambda\sprime{'}_{i_2}} &=
      (1 - u_3)(1 - u_2^2)\frac{w_i}{w\sprime{'}_{i_2}}\frac{1}{\lambda_i}\\
    \end{aligned}\qquad\right\}\label{eq:split_trans}
  \end{equation}
  where $u_1\sim\mathrm{Beta}(2,2)$, $u_2\sim\mathrm{Beta}(2,2)$ and
  $u_2\sim\mathrm{Beta}(1,1)$ and $\mathrm{Beta}(a,b)$ is the Beta
  distribution. It is easy to check that they are indeed bijection
  transformations.

  In the split move, if $\mu\sprime{'}_{i_1} < \mu_{i - 1}$ or
  $\mu\sprime{'}_{i_2} > \mu_{i + 1}$, i.e., the adjacency condition is not
  met, the proposal is rejected. Otherwise, it is accepted with probability
  $\min\{1, A\}$, where
  \begin{align}
    A =& \frac{(k + 1)!}{k!}
    \frac{\pi(\btheta'|k + 1)L(\by|\btheta',k + 1)}
    {\pi(\btheta|k)L(\by|\btheta,k)} \notag\\
    &\times\frac{d_{k+1}}{b_k}
    \frac{1}{g_{2,2}(u_1)g_{2,2}(u_2)g_{1,1}(u_3)} \notag\\
    &\times\frac{w_i\Abs{\mu\sprime{'}_{i_1} - \mu\sprime{'}_{i_2}}}
    {u_2(1 - u_2^2)u_3(1 - u_3)}
    \frac{\lambda\sprime{'}_{i_1}\lambda\sprime{'}_{i_2}}{\lambda_i}
  \end{align}
  where $g_{a,b}(\cdot)$ is the density of Beta distribution. The term
  $d_{k+1}/b_k$ is the odd of propose a combine move from $\btheta'$ back to
  $\btheta$ to the split move. The last line is the Jacobian of
  the transformation from $(\btheta,\bu)$ to $\btheta'$ with $\bu = (u_1, u_2,
  u_3)$. I haven't done all the calculation to verify this result. What I did
  is taking the results from \textcite{Richardson:1997ea} and taking our
  transformation as formed by a chain of three transformations, one transform
  from precision $\lambda_i$ to variance $\sigma_i^2$, the second make the
  transformation as in the paper, and last transform back to our parameters.
  So by multiplying two additional easy to compute Jacobian, I got the above
  result.

  In the combine move, we accept the proposed new state with probability
  $\min\{1,A^{-1}\}$ with suitable substitutions of notations.

  \printbibliography

\end{document}
